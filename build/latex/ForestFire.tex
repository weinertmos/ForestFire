%% Generated by Sphinx.
\def\sphinxdocclass{report}
\documentclass[letterpaper,10pt,english]{sphinxmanual}
\ifdefined\pdfpxdimen
   \let\sphinxpxdimen\pdfpxdimen\else\newdimen\sphinxpxdimen
\fi \sphinxpxdimen=49336sp\relax

\usepackage[margin=1in,marginparwidth=0.5in]{geometry}
\usepackage[utf8]{inputenc}
\ifdefined\DeclareUnicodeCharacter
  \DeclareUnicodeCharacter{00A0}{\nobreakspace}
\fi
\usepackage{cmap}
\usepackage[T1]{fontenc}
\usepackage{amsmath,amssymb,amstext}
\usepackage{babel}
\usepackage{times}
\usepackage[Bjarne]{fncychap}
\usepackage{longtable}
\usepackage{sphinx}

\usepackage{multirow}
\usepackage{eqparbox}

% Include hyperref last.
\usepackage{hyperref}
% Fix anchor placement for figures with captions.
\usepackage{hypcap}% it must be loaded after hyperref.
% Set up styles of URL: it should be placed after hyperref.
\urlstyle{same}

\addto\captionsenglish{\renewcommand{\figurename}{Fig.\@ }}
\addto\captionsenglish{\renewcommand{\tablename}{Table }}
\addto\captionsenglish{\renewcommand{\literalblockname}{Listing }}

\addto\extrasenglish{\def\pageautorefname{page}}

\setcounter{tocdepth}{2}



\title{ForestFire Documentation}
\date{Dec 22, 2017}
\release{1.1.5}
\author{Marlon Weinert}
\newcommand{\sphinxlogo}{}
\renewcommand{\releasename}{Release}
\makeindex

\begin{document}

\maketitle
\sphinxtableofcontents
\phantomsection\label{\detokenize{index::doc}}



\chapter{\_\_init\_\_}
\label{\detokenize{index:init}}\label{\detokenize{index:module-ForestFire}}\label{\detokenize{index:forestfire}}\index{ForestFire (module)}
\sphinxstyleemphasis{ForestFire} is a Python tool that aims to enhance the performance of machine learning algorithms. 
It utilises the Random Forest algorithm - which is itself a machine learning technique - to determine the 
importance of features in a given set of data and make new predictions which featuresets are most 
likely to yield the best results. 
After building a Random Forest only the most promising feature sets are presented to the machine learning algorithm to gain a better result. 
The Random Forest is burnt down and a new one is grown until the defined maximum number of forests is reached.
The results can be compared against random search.

The value of \sphinxstyleemphasis{ForestFire} lies in the selection of a {\hyperref[\detokenize{Overview:term-feature-set}]{\sphinxtermref{\DUrole{xref,std,std-term}{feature set}}}} that - when computed by the designated {\hyperref[\detokenize{Overview:mla}]{\sphinxcrossref{\DUrole{std,std-ref}{MLA}}}} - yields
better results than using all of the features or a random selection of features.

\sphinxstyleemphasis{ForestFire} is most usefull in data sets with a number of features greater than 10 where a single run of
a {\hyperref[\detokenize{Overview:mla}]{\sphinxcrossref{\DUrole{std,std-ref}{MLA}}}} has a high computational cost. In such data sets the problem arises that some features are
more significant than the rest.
Others may even distort the performance of the underlying {\hyperref[\detokenize{Overview:mla}]{\sphinxcrossref{\DUrole{std,std-ref}{MLA}}}} in a negative fashion. 
With a rising number of features the number of possible combinations (= feature sets) emerges and converges towards infinity.
In those cases ForestFire can help to choose those feature sets that are most promising to yield good results.
By predicting the performance of new feature sets according to their importance in a Random Forest built 
from previous runs it is more likely to find a feature set with a higher performance after a shorter period 
of time than randomly choosing new feature sets.

\sphinxstylestrong{Possible benefits:}
\begin{itemize}
\item {} 
Increase overall precision (higher accuracy / lower Error Rate)

\item {} 
Reduce overall computational cost (Finding a good solution earlier)

\item {} 
Gain knowledge about importance of single features

\end{itemize}


\section{How to use}
\label{\detokenize{index:how-to-use}}
In order to use \sphinxstyleemphasis{ForestFire} it is required to provide raw data in the form of two numpy arrays:
\begin{itemize}
\item {} 
\sphinxstylestrong{X.npy} - contains the values of the features for each data set

\item {} 
\sphinxstylestrong{y.npy} - contains the corresponding performance of those feature sets as a single value

\end{itemize}

The {\hyperref[\detokenize{Overview:mla}]{\sphinxcrossref{\DUrole{std,std-ref}{MLA}}}} and the way the raw data is split are configured in two seperate files:
\begin{itemize}
\item {} 
{\hyperref[\detokenize{Importing_Data:import-data}]{\sphinxcrossref{\DUrole{std,std-ref}{import\_data.py}}}} - X and y are loaded from the numpy files in the same folder. 
It is possible (yet not required) to apply data splitting methods here and return the train and test data sets.

\item {} 
{\hyperref[\detokenize{Generate_Database:compute}]{\sphinxcrossref{\DUrole{std,std-ref}{compute.py}}}} - Set up the {\hyperref[\detokenize{Overview:mla}]{\sphinxcrossref{\DUrole{std,std-ref}{MLA}}}} that you want to supply with promising selections of 
feature sets generated by \sphinxstyleemphasis{ForestFire}.

\end{itemize}

After \sphinxstyleemphasis{ForestFire} is supplied with the raw Data in X.npy and y.npy (import\_data.py)
and the designated {\hyperref[\detokenize{Overview:mla}]{\sphinxcrossref{\DUrole{std,std-ref}{MLA}}}} (compute.py) the default setup is complete. 
By executing \sphinxstylestrong{run\_ForestFire.py} the tool can be started with default or adjusted hyperparameters.

\sphinxstyleemphasis{ForestFire} will execute an initial \sphinxstyleemphasis{n\_start} number of {\hyperref[\detokenize{Overview:mla}]{\sphinxcrossref{\DUrole{std,std-ref}{MLA}}}} runs to set up an internal database. 
From this database single Decision Trees are built and grouped into a Random Forest. 
The Random Forest is evaluated to determine the importance of each feature.
\sphinxstyleemphasis{ForestFire} will next predict the performance of possible new feature sets (chosen both randomly and deliberately).
The two feature sets with the highest predicted performance (for mean and for variance) are selected, computed by the
original {\hyperref[\detokenize{Overview:mla}]{\sphinxcrossref{\DUrole{std,std-ref}{MLA}}}} and their result is added to the database. 
The Random Forest is burnt down and a new one is built, taking into account the two newly generated data points. 
A total number of n\_forests is built.
\sphinxstyleemphasis{ForestFire} will print the current best 5 feature sets as soon as a new top 5 feature set emerges.
After all Random Forests are built, the results are stored in descending order both as a .txt file and a .npy file.

In \sphinxstyleemphasis{Demo mode}, the performance of \sphinxstyleemphasis{ForestFire} is compared to randomly picking new featuresets.
This can be used to make sure that the algorithm does not only exploit local maxima, but keeps exploring the 
whole solution space.
The results can be plotted.

Quickstart: \href{https://github.com/weinertmos/ForestFire}{Clone Repository} and run ForestFire-master/Source/ForestFire/run\_ForestFire.py


\chapter{Using ForestFire}
\label{\detokenize{index:get-started}}\label{\detokenize{index:using-forestfire}}

\section{Overview}
\label{\detokenize{Overview:overview}}\label{\detokenize{Overview::doc}}\label{\detokenize{Overview:id1}}
ForestFire takes as input raw data and an unsupervised machine learning algorithm provided by the user and tries to optimize the feature selection for the given classifier.

It uses Decision Trees and Random Forests to predict the performance of untested feature sets. Only those which are most likely to perform well are used during future runs of the {\hyperref[\detokenize{Overview:term-machine-learning-algorithm}]{\sphinxtermref{\DUrole{xref,std,std-term}{Machine Learning Algorithm}}}}.


\subsection{Utilized Modules}
\label{\detokenize{Overview:utilized-modules}}
The following Modules are imported during the execution of ForestFire:

\begin{sphinxVerbatim}[commandchars=\\\{\}]
\PYG{c+c1}{\PYGZsh{}  Imports}
\PYG{k+kn}{import} \PYG{n+nn}{numpy} \PYG{k}{as} \PYG{n+nn}{np}
\PYG{k+kn}{import} \PYG{n+nn}{matplotlib}\PYG{n+nn}{.}\PYG{n+nn}{pyplot} \PYG{k}{as} \PYG{n+nn}{plt}
\PYG{k+kn}{from} \PYG{n+nn}{sklearn} \PYG{k}{import} \PYG{n}{preprocessing}
\end{sphinxVerbatim}


\subsection{Abbreviations}
\label{\detokenize{Overview:abbreviations}}\phantomsection\label{\detokenize{Overview:dt}}\begin{description}
\item[{DT}] \leavevmode
{\hyperref[\detokenize{Overview:term-decision-tree}]{\sphinxtermref{\DUrole{xref,std,std-term}{Decision Tree}}}}

\end{description}
\phantomsection\label{\detokenize{Overview:rf}}\begin{description}
\item[{RF}] \leavevmode
{\hyperref[\detokenize{Overview:term-random-forest}]{\sphinxtermref{\DUrole{xref,std,std-term}{Random Forest}}}}

\end{description}
\phantomsection\label{\detokenize{Overview:mla}}\begin{description}
\item[{MLA}] \leavevmode
{\hyperref[\detokenize{Overview:term-machine-learning-algorithm}]{\sphinxtermref{\DUrole{xref,std,std-term}{Machine Learning Algorithm}}}}

\end{description}


\subsection{Glossary}
\label{\detokenize{Overview:glossary}}\begin{description}
\item[{branch\index{branch|textbf}}] \leavevmode\phantomsection\label{\detokenize{Overview:term-branch}}
Junction in a {\hyperref[\detokenize{Overview:term-decision-tree}]{\sphinxtermref{\DUrole{xref,std,std-term}{Decision Tree}}}}. Each {\hyperref[\detokenize{Overview:term-node}]{\sphinxtermref{\DUrole{xref,std,std-term}{node}}}} has a true and a false branch leading away from it.

\item[{Decision Tree\index{Decision Tree|textbf}}] \leavevmode\phantomsection\label{\detokenize{Overview:term-decision-tree}}
Consists of at least one {\hyperref[\detokenize{Overview:term-node}]{\sphinxtermref{\DUrole{xref,std,std-term}{node}}}} and represents a treelike structure that can be used for classification of new observations

\item[{feature\index{feature|textbf}}] \leavevmode\phantomsection\label{\detokenize{Overview:term-feature}}
Unique property of the raw data set.
Typically all entries in a specific column of the raw data.

\item[{feature set\index{feature set|textbf}}] \leavevmode\phantomsection\label{\detokenize{Overview:term-feature-set}}
Combination of several single features.
At least one, at most all of the available features.
Used to present the {\hyperref[\detokenize{Overview:term-machine-learning-algorithm}]{\sphinxtermref{\DUrole{xref,std,std-term}{Machine Learning Algorithm}}}} with a selection of features on which it performs with better results.
Synonym to {\hyperref[\detokenize{Overview:term-observation}]{\sphinxtermref{\DUrole{xref,std,std-term}{Observation}}}}.

\item[{ForestFire\index{ForestFire|textbf}}] \leavevmode\phantomsection\label{\detokenize{Overview:term-forestfire}}
Subject of this documentation.
Tool that can imporove performance and efficiency of {\hyperref[\detokenize{Overview:term-machine-learning-algorithm}]{\sphinxtermref{\DUrole{xref,std,std-term}{MLAs}}}}.

\item[{leaf\index{leaf|textbf}}] \leavevmode\phantomsection\label{\detokenize{Overview:term-leaf}}
Last point of a {\hyperref[\detokenize{Overview:term-branch}]{\sphinxtermref{\DUrole{xref,std,std-term}{branch}}}} in a :term{}`Decision Tree{}`

\item[{Machine Learning Algorithm\index{Machine Learning Algorithm|textbf}}] \leavevmode\phantomsection\label{\detokenize{Overview:term-machine-learning-algorithm}}
Specified by the user in {\hyperref[\detokenize{Generate_Database:compute}]{\sphinxcrossref{\DUrole{std,std-ref}{Generate Database}}}}.
Can basically be any existing unsupervised machine learning algorithm that classifies the raw data.
Results can be improved by {\hyperref[\detokenize{Overview:term-forestfire}]{\sphinxtermref{\DUrole{xref,std,std-term}{ForestFire}}}}

\item[{node\index{node|textbf}}] \leavevmode\phantomsection\label{\detokenize{Overview:term-node}}
A point in a {\hyperref[\detokenize{Overview:term-decision-tree}]{\sphinxtermref{\DUrole{xref,std,std-term}{Decision Tree}}}} where a decision is made (either true or false)

\item[{Observation\index{Observation|textbf}}] \leavevmode\phantomsection\label{\detokenize{Overview:term-observation}}
Synonym to {\hyperref[\detokenize{Overview:term-feature-set}]{\sphinxtermref{\DUrole{xref,std,std-term}{feature set}}}}.

\item[{pruning\index{pruning|textbf}}] \leavevmode\phantomsection\label{\detokenize{Overview:term-pruning}}
Cutting back {\hyperref[\detokenize{Overview:term-branch}]{\sphinxtermref{\DUrole{xref,std,std-term}{branches}}}} of a {\hyperref[\detokenize{Overview:term-decision-tree}]{\sphinxtermref{\DUrole{xref,std,std-term}{Decision Tree}}}} with little information gain.
See {\hyperref[\detokenize{DT:prune}]{\sphinxcrossref{\DUrole{std,std-ref}{prune}}}}

\item[{Random Forest\index{Random Forest|textbf}}] \leavevmode\phantomsection\label{\detokenize{Overview:term-random-forest}}
Cumulation of {\hyperref[\detokenize{Overview:term-decision-tree}]{\sphinxtermref{\DUrole{xref,std,std-term}{Decision Trees}}}} that can be used for classification of new observations

\item[{Raw data set\index{Raw data set|textbf}}] \leavevmode\phantomsection\label{\detokenize{Overview:term-raw-data-set}}
Data set that is provided by the user in {\hyperref[\detokenize{Importing_Data:import-data}]{\sphinxcrossref{\DUrole{std,std-ref}{Import Data}}}}.
The {\hyperref[\detokenize{Overview:mla}]{\sphinxcrossref{\DUrole{std,std-ref}{MLA}}}} will run on this data set.

\item[{Synonyms\index{Synonyms|textbf}}] \leavevmode\phantomsection\label{\detokenize{Overview:term-synonyms}}\begin{itemize}
\item {} 
Feature Importance = Feature Probability

\end{itemize}

\end{description}


\subsection{References}
\label{\detokenize{Overview:references}}

\subsection{About the author}
\label{\detokenize{Overview:about-the-author}}
Marlon Weinert is currently graduating from the University of Stuttgart, Germany.
After finishing a B.Sc. degree in Renewable Energies in 2015 he enroled in the M.Sc. program for Energy Technology.
ForestFire is his first python and machine learning project.


\section{Import Data}
\label{\detokenize{Importing_Data:import-data}}\label{\detokenize{Importing_Data::doc}}\label{\detokenize{Importing_Data:id1}}
corresponding file: \href{https://github.com/weinertmos/ForestFire/blob/master/source/ForestFire/import\_data.py}{import\_data.py}

In this step the raw data is imported.
It must consist of two numpy arrays \sphinxstylestrong{X} and \sphinxstylestrong{y} which are located in the same directory as \sphinxstyleemphasis{import\_data.py}.
\sphinxstylestrong{X} contains the data sets in rows and the features in columns.
For example, X{[}0:12{]} is the value of the 13th feature in the first data set.
\sphinxstylestrong{y} contains the corresponding result for all data sets in a single column.
It must be of the same length as X.
For example y{[}19{]} is the result of the 20th data set.

After loading the data apply how it should be splitted into train and test data sets and set \sphinxstylestrong{X\_train / X\_test and y\_train / y\_test} accordingly.

\begin{sphinxadmonition}{note}{Note:}
The train/test split in \sphinxstyleemphasis{import\_data.py} will only be done once!
Use it if a fix split is desired.
If a split should be done in every future calculation (e.g. with shufflesplit),
set \sphinxstylestrong{X = X\_test = X\_train and y= y\_test = y\_train} and configure the splitting routine
in the next step ({\hyperref[\detokenize{Generate_Database:compute}]{\sphinxcrossref{\DUrole{std,std-ref}{Generate Database}}}}).
\end{sphinxadmonition}

\begin{sphinxadmonition}{important}{Important:}
\sphinxstylestrong{Functions used in this chapter}
\index{import\_data() (in module ForestFire.import\_data)}

\begin{fulllineitems}
\phantomsection\label{\detokenize{Importing_Data:ForestFire.import_data.import_data}}\pysiglinewithargsret{\sphinxcode{ForestFire.import\_data.}\sphinxbfcode{import\_data}}{}{}
import the raw data from two numpy arrays.

Import raw data from two numpy arrays X.npy and y.npy. 
Set how train and test data are to be split for fix splits.
Returns train/test splits as well as number of features.

Returns:
\begin{itemize}
\item {} 
X\_test \{np.array\} -- result training data

\item {} 
X\_train \{np.array\} -- feature training data

\item {} 
y\_test \{np.array\} -- result test data

\item {} 
y\_train \{np.array\} -- result training data

\end{itemize}

\end{fulllineitems}

\end{sphinxadmonition}


\section{Generate Database}
\label{\detokenize{Generate_Database:generate-database}}\label{\detokenize{Generate_Database:compute}}\label{\detokenize{Generate_Database::doc}}
corresponding file: \href{https://github.com/weinertmos/ForestFire/blob/master/source/ForestFire/compute.py}{compute.py}

In this step the underlying machine learning algorithm can be configured from scratch or inserted from an existing file.
Required imports can be put at the top of the file.
The default algorithm can be replaced.
As inputs the train / test split data from {\hyperref[\detokenize{Importing_Data:import-data}]{\sphinxcrossref{\DUrole{std,std-ref}{Import Data}}}} can be used.

\begin{sphinxadmonition}{note}{Note:}
If no train / test split has been configured in {\hyperref[\detokenize{Importing_Data:import-data}]{\sphinxcrossref{\DUrole{std,std-ref}{Import Data}}}} it has to be done here.
\end{sphinxadmonition}

The result of the {\hyperref[\detokenize{Overview:mla}]{\sphinxcrossref{\DUrole{std,std-ref}{MLA}}}} is stored in the variable \sphinxstyleemphasis{score} and returned to the main file.

\begin{sphinxadmonition}{important}{Important:}
\sphinxstylestrong{Functions used in this chapter}
\index{compute() (in module ForestFire.compute)}

\begin{fulllineitems}
\phantomsection\label{\detokenize{Generate_Database:ForestFire.compute.compute}}\pysiglinewithargsret{\sphinxcode{ForestFire.compute.}\sphinxbfcode{compute}}{\emph{X\_train}, \emph{y\_train}, \emph{mask\_sub\_features}, \emph{X\_test}, \emph{y\_test}}{}
Computes a new dataset for the Random Forest with the underlying machine learning algorithm.

Configure your machine learning algorithm here.
Add imports at the top of the file.
If no train / test split is done during import, X\_train and X\_test are equal (y\_train and y\_test as well).
In this case define your own splits with your machine learning algorithm.

Arguments:
\begin{itemize}
\item {} 
X\_train \{np.array\} -- feature training data

\item {} 
y\_train \{np.array\} -- result training data

\item {} 
mask\_sub\_features \{np.array\} -- feature set = dedicated part of all features

\item {} 
X\_test \{np.array\} -- result training data

\item {} 
y\_test \{np.array\} -- result test data

\end{itemize}
\begin{description}
\item[{Returns:}] \leavevmode
score \{np.float64\} -- score of the selected feature set

\end{description}

\end{fulllineitems}

\index{gen\_database() (in module ForestFire.Main)}

\begin{fulllineitems}
\phantomsection\label{\detokenize{Generate_Database:ForestFire.Main.gen_database}}\pysiglinewithargsret{\sphinxcode{ForestFire.Main.}\sphinxbfcode{gen\_database}}{\emph{n\_start}, \emph{X}, \emph{y}, \emph{X\_test}, \emph{y\_test}}{}
Runs the underlying {\hyperref[\detokenize{Overview:mla}]{\sphinxcrossref{\DUrole{std,std-ref}{MLA}}}} \sphinxstyleemphasis{n\_start} times to generate a database from which Random Forests can be built.
\begin{description}
\item[{Arguments:}] \leavevmode\begin{itemize}
\item {} 
n\_start \{int\} -- number of times the underlying {\hyperref[\detokenize{Overview:mla}]{\sphinxcrossref{\DUrole{std,std-ref}{MLA}}}} is executed

\item {} 
X \{numpy.array\} -- raw data

\item {} 
y \{numpy.array\} -- raw data

\item {} 
X\_test \{numpy.array\} -- test data

\item {} 
y\_test \{numpy.array\} -- test data

\end{itemize}

\item[{Returns:}] \leavevmode
{[}numpy.array{]} -- data set containing feature sets and corresponding results

\end{description}

\end{fulllineitems}

\end{sphinxadmonition}


\section{Execution}
\label{\detokenize{execution:execution}}\label{\detokenize{execution::doc}}\label{\detokenize{execution:id1}}
corresponding file: \href{https://github.com/weinertmos/ForestFire/blob/master/source/ForestFire/run\_ForestFire.py}{run\_ForestFire.py}

After {\hyperref[\detokenize{Importing_Data:import-data}]{\sphinxcrossref{\DUrole{std,std-ref}{importing the raw data}}}} and {\hyperref[\detokenize{Generate_Database:compute}]{\sphinxcrossref{\DUrole{std,std-ref}{configuring the MLA}}}}, ForestFire can be executed.


\subsection{Hyperparameters}
\label{\detokenize{execution:hyperparameters}}\label{\detokenize{execution:id2}}
There is a number of hyperparameters that can be changed or left at default:

\begin{sphinxVerbatim}[commandchars=\\\{\}]
\PYG{c+c1}{\PYGZsh{} Hyperparameters \PYGZsh{}}

\PYG{c+c1}{\PYGZsh{} number of runs before building first Random Forest = number of data points in first RF; minimum = 4, default = 50}
\PYG{c+c1}{\PYGZsh{} adjust according to computational capabilities and demands of the underlying machine learning algorithm}
\PYG{n}{n\PYGZus{}start} \PYG{o}{=} \PYG{l+m+mi}{30}  \PYG{c+c1}{\PYGZsh{} default = 30e}
\PYG{c+c1}{\PYGZsh{} if pruning is greater than zero, branches of a Decision Tree will be pruned proportional to pruning value; default = 0}
\PYG{c+c1}{\PYGZsh{} advanced parameter. If set too high, all trees will be cut down to stumps. Increase carefully. Start with values between 0 and 1.}
\PYG{n}{pruning} \PYG{o}{=} \PYG{l+m+mf}{0.3}
\PYG{c+c1}{\PYGZsh{} minimum percentage of Datasets that is used in RF generation; default = 0.2}
\PYG{n}{min\PYGZus{}data} \PYG{o}{=} \PYG{l+m+mf}{0.1}
\PYG{c+c1}{\PYGZsh{} number of forests; minimum=1;  default = 25}
\PYG{c+c1}{\PYGZsh{} adjust according to computational capabilities. For each forest two new computational runs are done. default = 20}
\PYG{n}{n\PYGZus{}forests} \PYG{o}{=} \PYG{l+m+mi}{85}
\end{sphinxVerbatim}

These parameters should be chosen according to computational demand of the {\hyperref[\detokenize{Overview:mla}]{\sphinxcrossref{\DUrole{std,std-ref}{MLA}}}}.
It makes sense to start with a small number of runs and increase it carefully.
Pruning is an advanced parameter.
If it is set to high, every single branch will be cut and only a tree stump with a single node is left.
If this parameter is used at all it should be incremented carefully to find a good balance between merging branches and keeping the tree significant.

The following parameters can be left at default since they adapt to the raw data automatically.
But changing them can tweak the performance.

\begin{sphinxVerbatim}[commandchars=\\\{\}]
\PYG{c+c1}{\PYGZsh{} number of trees that stand in a forest; min = 3; default = number of features * 3}
\PYG{n}{n\PYGZus{}trees} \PYG{o}{=} \PYG{l+s+s1}{\PYGZsq{}}\PYG{l+s+s1}{default}\PYG{l+s+s1}{\PYGZsq{}}
\PYG{c+c1}{\PYGZsh{} number of deliberately chosen feature sets that get predicted in each forest; default = n\PYGZus{}trees * 5}
\PYG{n}{n\PYGZus{}configs\PYGZus{}biased} \PYG{o}{=} \PYG{l+s+s1}{\PYGZsq{}}\PYG{l+s+s1}{default}\PYG{l+s+s1}{\PYGZsq{}}
\PYG{c+c1}{\PYGZsh{} number of randomly chosen feature sets that get predicted in each forest; default = n\PYGZus{}configs\PYGZus{}biased * 0.2}
\PYG{n}{n\PYGZus{}configs\PYGZus{}unbiased} \PYG{o}{=} \PYG{l+s+s1}{\PYGZsq{}}\PYG{l+s+s1}{default}\PYG{l+s+s1}{\PYGZsq{}}
\PYG{c+c1}{\PYGZsh{} sets how aggressively the feature importance changes; default = 0.25}
\PYG{c+c1}{\PYGZsh{} higher values will increase pressure on how often promising features will be selected.}
\PYG{c+c1}{\PYGZsh{} advanced parameter, adjust carefully. If set too high the risk of runnning into local extrema rises.}
\PYG{n}{multiplier\PYGZus{}stepup} \PYG{o}{=} \PYG{l+s+s1}{\PYGZsq{}}\PYG{l+s+s1}{default}\PYG{l+s+s1}{\PYGZsq{}}
\PYG{c+c1}{\PYGZsh{} number of recent forests that are taken into acount for generating probability of the chosen feature sets default = 0.01}
\PYG{n}{seen\PYGZus{}forests} \PYG{o}{=} \PYG{l+s+s1}{\PYGZsq{}}\PYG{l+s+s1}{default}\PYG{l+s+s1}{\PYGZsq{}}
\PYG{c+c1}{\PYGZsh{} the chosen feature sets default = 4}

\PYG{c+c1}{\PYGZsh{} weight of the mean in calculating the new probability for selecting future feature sets; default = 0.2}
\PYG{n}{weight\PYGZus{}mean} \PYG{o}{=} \PYG{l+s+s1}{\PYGZsq{}}\PYG{l+s+s1}{default}\PYG{l+s+s1}{\PYGZsq{}}
\PYG{c+c1}{\PYGZsh{} weight of the gradient in calculating the new probability for selecting future feature sets; default = 0.8}
\PYG{n}{weight\PYGZus{}gradient} \PYG{o}{=} \PYG{l+s+s1}{\PYGZsq{}}\PYG{l+s+s1}{default}\PYG{l+s+s1}{\PYGZsq{}}

\PYG{c+c1}{\PYGZsh{} which scoring metric should be used in the Decision Tree (available: entropy, giniimpurity and variance); default = entropy}
\PYG{c+c1}{\PYGZsh{} select variance for numerical values in y only}
\PYG{n}{scoref} \PYG{o}{=} \PYG{l+s+s1}{\PYGZsq{}}\PYG{l+s+s1}{entropy}\PYG{l+s+s1}{\PYGZsq{}}
\PYG{c+c1}{\PYGZsh{} set random seed for repeatabilit; comment out if no repeatability is required; default = 1}
\PYG{n}{np}\PYG{o}{.}\PYG{n}{random}\PYG{o}{.}\PYG{n}{seed}\PYG{p}{(}\PYG{l+m+mi}{10}\PYG{p}{)}
\end{sphinxVerbatim}


\subsection{Demo Mode \& Plot}
\label{\detokenize{execution:demo-mode-plot}}
In order to compare and plot the performance of ForestFire vs. a randomized search there are two more hyperparameters that can be used:

\begin{sphinxVerbatim}[commandchars=\\\{\}]

\PYG{c+c1}{\PYGZsh{} if true a comparison between the Random Forest driven Search and a random search is done}
\PYG{n}{demo\PYGZus{}mode} \PYG{o}{=} \PYG{k+kc}{True}
\PYG{c+c1}{\PYGZsh{} decide if at the end a plot should be generated , only valid in demo mode}
\end{sphinxVerbatim}

This mode can be usefull when trying to make sure that ForestFire doesn't get caught in a local extremum.
In general ForestFire should always find solutions that are at least as good as a random search - otherwise there is no sense in using it at all - or better.
If that's not the case it might be ``stuck'' at a dominant feature set that seems to perform well, but there are even better feature sets that never get chosen.


\subsection{Output}
\label{\detokenize{execution:output}}
By Executing \href{https://github.com/weinertmos/ForestFire/blob/master/source/ForestFire/run\_ForestFire.py}{run\_ForestFire.py} the algorithm starts.
When a new feature set with good performance (top 5) is found, the current 5 best feature sets and the according performance are printed to the console.
For each feature either 1 or 0 is displayed.
1 means that the underlying {\hyperref[\detokenize{Overview:mla}]{\sphinxcrossref{\DUrole{std,std-ref}{MLA}}}} did ``see'' the feature, 0 means this feature was left out

Naturally in the first runs there will be more new best feature sets.
The longer the algorithm continues the harder it gets to find better values.

The importance of a feature can be interpreted by looking at the feature sets that had the best results.
If for example a feature is included in all best feature sets it has a high importance.
If on the other hand a feature is never included, this indicates that the feature is either not important or is even a distortion to the {\hyperref[\detokenize{Overview:mla}]{\sphinxcrossref{\DUrole{std,std-ref}{MLA}}}}.

After all Random Forests are built, the results are stored both as a .txt (human readable) and a .npy (binary, for further use with python) file.
In the results file the rows contain all feature set combinations calculated by the {\hyperref[\detokenize{Overview:mla}]{\sphinxcrossref{\DUrole{std,std-ref}{MLA}}}}.
The last column contains the corresponding results to the preceding information wether a feature has been selected in the particular run.
Example: {[}1 0 0 1 0.9432{]} means that feature 1 and 4 were presented to the {\hyperref[\detokenize{Overview:mla}]{\sphinxcrossref{\DUrole{std,std-ref}{MLA}}}} whereas feature 2 and 3 were not. The result corresponding result is 94.32\%.


\subsubsection{Example}
\label{\detokenize{execution:example}}
A generic output (with demo mode on) can look like this:

\begin{sphinxVerbatim}[commandchars=\\\{\}]
\PYG{n}{Starting} \PYG{n}{ForestFire}
\PYG{n}{Loading} \PYG{n}{Raw} \PYG{n}{Data}
\PYG{n}{setting} \PYG{n}{Hyperparameters}
\PYG{n}{Generate} \PYG{n}{Data} \PYG{n}{Base} \PYG{k}{for} \PYG{n}{Random} \PYG{n}{Forest}
\PYG{n}{Starting} \PYG{n}{ForestFire}

\PYG{n}{Building} \PYG{n}{Random} \PYG{n}{Forest} \PYG{n}{Nr}\PYG{o}{.} \PYG{l+m+mi}{1}
\PYG{n}{wrongs}\PYG{p}{:} \PYG{l+m+mi}{9}\PYG{o}{/}\PYG{l+m+mi}{39}
\PYG{n+nb}{max} \PYG{n}{Probability}\PYG{p}{:} \PYG{k+kc}{None}
\PYG{n}{picked} \PYG{n}{biased} \PYG{n}{feature} \PYG{n+nb}{set} \PYG{k}{for} \PYG{n}{mean}
\PYG{n}{picked} \PYG{n}{biased} \PYG{n}{feature} \PYG{n+nb}{set} \PYG{k}{for} \PYG{n}{var}
\PYG{n}{found} \PYG{n}{new} \PYG{n}{best} \PYG{l+m+mi}{5} \PYG{n}{feature} \PYG{n}{sets}\PYG{p}{:} \PYG{p}{[}\PYG{p}{[} \PYG{l+m+mf}{1.}          \PYG{l+m+mf}{1.}          \PYG{l+m+mf}{1.}          \PYG{l+m+mf}{1.}          \PYG{l+m+mf}{1.}          \PYG{l+m+mf}{0.}          \PYG{l+m+mf}{1.}
   \PYG{l+m+mf}{1.}          \PYG{l+m+mf}{0.}          \PYG{l+m+mf}{1.}          \PYG{l+m+mf}{0.}          \PYG{l+m+mf}{1.}          \PYG{l+m+mf}{0.}          \PYG{l+m+mf}{0.74}      \PYG{p}{]}
 \PYG{p}{[} \PYG{l+m+mf}{1.}          \PYG{l+m+mf}{0.}          \PYG{l+m+mf}{1.}          \PYG{l+m+mf}{0.}          \PYG{l+m+mf}{0.}          \PYG{l+m+mf}{1.}          \PYG{l+m+mf}{1.}
   \PYG{l+m+mf}{1.}          \PYG{l+m+mf}{0.}          \PYG{l+m+mf}{0.}          \PYG{l+m+mf}{0.}          \PYG{l+m+mf}{0.}          \PYG{l+m+mf}{0.}
   \PYG{l+m+mf}{0.72666667}\PYG{p}{]}
 \PYG{p}{[} \PYG{l+m+mf}{0.}          \PYG{l+m+mf}{0.}          \PYG{l+m+mf}{0.}          \PYG{l+m+mf}{1.}          \PYG{l+m+mf}{1.}          \PYG{l+m+mf}{1.}          \PYG{l+m+mf}{1.}
   \PYG{l+m+mf}{1.}          \PYG{l+m+mf}{1.}          \PYG{l+m+mf}{1.}          \PYG{l+m+mf}{0.}          \PYG{l+m+mf}{1.}          \PYG{l+m+mf}{0.}          \PYG{l+m+mf}{0.71}      \PYG{p}{]}
 \PYG{p}{[} \PYG{l+m+mf}{1.}          \PYG{l+m+mf}{1.}          \PYG{l+m+mf}{1.}          \PYG{l+m+mf}{0.}          \PYG{l+m+mf}{1.}          \PYG{l+m+mf}{0.}          \PYG{l+m+mf}{1.}
   \PYG{l+m+mf}{1.}          \PYG{l+m+mf}{0.}          \PYG{l+m+mf}{1.}          \PYG{l+m+mf}{0.}          \PYG{l+m+mf}{0.}          \PYG{l+m+mf}{1.}
   \PYG{l+m+mf}{0.68666667}\PYG{p}{]}
 \PYG{p}{[} \PYG{l+m+mf}{0.}          \PYG{l+m+mf}{0.}          \PYG{l+m+mf}{1.}          \PYG{l+m+mf}{0.}          \PYG{l+m+mf}{1.}          \PYG{l+m+mf}{1.}          \PYG{l+m+mf}{0.}
   \PYG{l+m+mf}{1.}          \PYG{l+m+mf}{1.}          \PYG{l+m+mf}{1.}          \PYG{l+m+mf}{0.}          \PYG{l+m+mf}{1.}          \PYG{l+m+mf}{0.}
   \PYG{l+m+mf}{0.67666667}\PYG{p}{]}\PYG{p}{]}

\PYG{n}{Building} \PYG{n}{Random} \PYG{n}{Forest} \PYG{n}{Nr}\PYG{o}{.} \PYG{l+m+mi}{2}
\PYG{n}{wrongs}\PYG{p}{:} \PYG{l+m+mi}{2}\PYG{o}{/}\PYG{l+m+mi}{39}
\PYG{n+nb}{max} \PYG{n}{Probability}\PYG{p}{:} \PYG{k+kc}{None}
\PYG{n}{picked} \PYG{n}{biased} \PYG{n}{feature} \PYG{n+nb}{set} \PYG{k}{for} \PYG{n}{mean}
\PYG{n}{picked} \PYG{n}{unbiased} \PYG{n}{feature} \PYG{n+nb}{set} \PYG{k}{for} \PYG{n}{var}
\PYG{n}{found} \PYG{n}{new} \PYG{n}{best} \PYG{l+m+mi}{5} \PYG{n}{feature} \PYG{n}{sets}\PYG{p}{:} \PYG{p}{[}\PYG{p}{[} \PYG{l+m+mf}{1.}          \PYG{l+m+mf}{1.}          \PYG{l+m+mf}{1.}          \PYG{l+m+mf}{1.}          \PYG{l+m+mf}{1.}          \PYG{l+m+mf}{0.}          \PYG{l+m+mf}{1.}
   \PYG{l+m+mf}{1.}          \PYG{l+m+mf}{0.}          \PYG{l+m+mf}{1.}          \PYG{l+m+mf}{0.}          \PYG{l+m+mf}{1.}          \PYG{l+m+mf}{0.}          \PYG{l+m+mf}{0.74}      \PYG{p}{]}
 \PYG{p}{[} \PYG{l+m+mf}{1.}          \PYG{l+m+mf}{0.}          \PYG{l+m+mf}{1.}          \PYG{l+m+mf}{0.}          \PYG{l+m+mf}{0.}          \PYG{l+m+mf}{1.}          \PYG{l+m+mf}{1.}
   \PYG{l+m+mf}{1.}          \PYG{l+m+mf}{0.}          \PYG{l+m+mf}{0.}          \PYG{l+m+mf}{0.}          \PYG{l+m+mf}{0.}          \PYG{l+m+mf}{0.}
   \PYG{l+m+mf}{0.72666667}\PYG{p}{]}
 \PYG{p}{[} \PYG{l+m+mf}{1.}          \PYG{l+m+mf}{1.}          \PYG{l+m+mf}{1.}          \PYG{l+m+mf}{0.}          \PYG{l+m+mf}{1.}          \PYG{l+m+mf}{1.}          \PYG{l+m+mf}{1.}
   \PYG{l+m+mf}{1.}          \PYG{l+m+mf}{0.}          \PYG{l+m+mf}{1.}          \PYG{l+m+mf}{0.}          \PYG{l+m+mf}{0.}          \PYG{l+m+mf}{1.}
   \PYG{l+m+mf}{0.71333333}\PYG{p}{]}
 \PYG{p}{[} \PYG{l+m+mf}{0.}          \PYG{l+m+mf}{0.}          \PYG{l+m+mf}{0.}          \PYG{l+m+mf}{1.}          \PYG{l+m+mf}{1.}          \PYG{l+m+mf}{1.}          \PYG{l+m+mf}{1.}
   \PYG{l+m+mf}{1.}          \PYG{l+m+mf}{1.}          \PYG{l+m+mf}{1.}          \PYG{l+m+mf}{0.}          \PYG{l+m+mf}{1.}          \PYG{l+m+mf}{0.}          \PYG{l+m+mf}{0.71}      \PYG{p}{]}
 \PYG{p}{[} \PYG{l+m+mf}{1.}          \PYG{l+m+mf}{1.}          \PYG{l+m+mf}{1.}          \PYG{l+m+mf}{1.}          \PYG{l+m+mf}{1.}          \PYG{l+m+mf}{1.}          \PYG{l+m+mf}{1.}
   \PYG{l+m+mf}{1.}          \PYG{l+m+mf}{1.}          \PYG{l+m+mf}{1.}          \PYG{l+m+mf}{1.}          \PYG{l+m+mf}{0.}          \PYG{l+m+mf}{1.}          \PYG{l+m+mf}{0.7}       \PYG{p}{]}\PYG{p}{]}

   \PYG{o}{.}\PYG{o}{.}\PYG{o}{.}
   \PYG{o}{.}\PYG{o}{.}\PYG{o}{.}
   \PYG{o}{.}\PYG{o}{.}\PYG{o}{.}

\PYG{n}{Building} \PYG{n}{Random} \PYG{n}{Forest} \PYG{n}{Nr}\PYG{o}{.} \PYG{l+m+mi}{8}
\PYG{n}{wrongs}\PYG{p}{:} \PYG{l+m+mi}{4}\PYG{o}{/}\PYG{l+m+mi}{39}
\PYG{n+nb}{max} \PYG{n}{Probability}\PYG{p}{:} \PYG{l+m+mf}{0.133463620284}
\PYG{n}{raised} \PYG{n}{multiplier} \PYG{n}{to} \PYG{l+m+mf}{1.03}
\PYG{n}{picked} \PYG{n}{biased} \PYG{n}{feature} \PYG{n+nb}{set} \PYG{k}{for} \PYG{n}{mean}
\PYG{n}{picked} \PYG{n}{biased} \PYG{n}{feature} \PYG{n+nb}{set} \PYG{k}{for} \PYG{n}{var}
\PYG{n}{found} \PYG{n}{new} \PYG{n}{best} \PYG{l+m+mi}{5} \PYG{n}{feature} \PYG{n}{sets}\PYG{p}{:} \PYG{p}{[}\PYG{p}{[} \PYG{l+m+mf}{1.}          \PYG{l+m+mf}{0.}          \PYG{l+m+mf}{1.}          \PYG{l+m+mf}{1.}          \PYG{l+m+mf}{1.}          \PYG{l+m+mf}{1.}          \PYG{l+m+mf}{1.}
   \PYG{l+m+mf}{1.}          \PYG{l+m+mf}{1.}          \PYG{l+m+mf}{1.}          \PYG{l+m+mf}{1.}          \PYG{l+m+mf}{1.}          \PYG{l+m+mf}{1.}
   \PYG{l+m+mf}{0.76333333}\PYG{p}{]}
 \PYG{p}{[} \PYG{l+m+mf}{1.}          \PYG{l+m+mf}{0.}          \PYG{l+m+mf}{1.}          \PYG{l+m+mf}{1.}          \PYG{l+m+mf}{1.}          \PYG{l+m+mf}{1.}          \PYG{l+m+mf}{1.}
   \PYG{l+m+mf}{1.}          \PYG{l+m+mf}{1.}          \PYG{l+m+mf}{1.}          \PYG{l+m+mf}{1.}          \PYG{l+m+mf}{1.}          \PYG{l+m+mf}{1.}
   \PYG{l+m+mf}{0.76333333}\PYG{p}{]}
 \PYG{p}{[} \PYG{l+m+mf}{1.}          \PYG{l+m+mf}{0.}          \PYG{l+m+mf}{1.}          \PYG{l+m+mf}{1.}          \PYG{l+m+mf}{1.}          \PYG{l+m+mf}{1.}          \PYG{l+m+mf}{1.}
   \PYG{l+m+mf}{1.}          \PYG{l+m+mf}{1.}          \PYG{l+m+mf}{1.}          \PYG{l+m+mf}{1.}          \PYG{l+m+mf}{1.}          \PYG{l+m+mf}{1.}
   \PYG{l+m+mf}{0.76333333}\PYG{p}{]}
 \PYG{p}{[} \PYG{l+m+mf}{1.}          \PYG{l+m+mf}{1.}          \PYG{l+m+mf}{1.}          \PYG{l+m+mf}{1.}          \PYG{l+m+mf}{1.}          \PYG{l+m+mf}{1.}          \PYG{l+m+mf}{1.}
   \PYG{l+m+mf}{1.}          \PYG{l+m+mf}{1.}          \PYG{l+m+mf}{1.}          \PYG{l+m+mf}{1.}          \PYG{l+m+mf}{1.}          \PYG{l+m+mf}{1.}
   \PYG{l+m+mf}{0.74666667}\PYG{p}{]}
 \PYG{p}{[} \PYG{l+m+mf}{1.}          \PYG{l+m+mf}{1.}          \PYG{l+m+mf}{1.}          \PYG{l+m+mf}{1.}          \PYG{l+m+mf}{0.}          \PYG{l+m+mf}{1.}          \PYG{l+m+mf}{1.}
   \PYG{l+m+mf}{1.}          \PYG{l+m+mf}{1.}          \PYG{l+m+mf}{1.}          \PYG{l+m+mf}{1.}          \PYG{l+m+mf}{1.}          \PYG{l+m+mf}{1.}
   \PYG{l+m+mf}{0.74666667}\PYG{p}{]}\PYG{p}{]}

\PYG{n}{Building} \PYG{n}{Random} \PYG{n}{Forest} \PYG{n}{Nr}\PYG{o}{.} \PYG{l+m+mi}{9}
\PYG{n}{wrongs}\PYG{p}{:} \PYG{l+m+mi}{5}\PYG{o}{/}\PYG{l+m+mi}{39}
\PYG{n+nb}{max} \PYG{n}{Probability}\PYG{p}{:} \PYG{l+m+mf}{0.16963581418}
\PYG{n}{picked} \PYG{n}{biased} \PYG{n}{feature} \PYG{n+nb}{set} \PYG{k}{for} \PYG{n}{mean}
\PYG{n}{picked} \PYG{n}{biased} \PYG{n}{feature} \PYG{n+nb}{set} \PYG{k}{for} \PYG{n}{var}

\PYG{n}{Building} \PYG{n}{Random} \PYG{n}{Forest} \PYG{n}{Nr}\PYG{o}{.} \PYG{l+m+mi}{10}
\PYG{n}{wrongs}\PYG{p}{:} \PYG{l+m+mi}{2}\PYG{o}{/}\PYG{l+m+mi}{39}

\PYG{n+nb}{max} \PYG{n}{Probability}\PYG{p}{:} \PYG{l+m+mf}{0.130904237306}
\PYG{n}{raised} \PYG{n}{multiplier} \PYG{n}{to} \PYG{l+m+mf}{1.04}
\PYG{n}{picked} \PYG{n}{biased} \PYG{n}{feature} \PYG{n+nb}{set} \PYG{k}{for} \PYG{n}{mean}
\PYG{n}{picked} \PYG{n}{biased} \PYG{n}{feature} \PYG{n+nb}{set} \PYG{k}{for} \PYG{n}{var}

\PYG{n}{ForestFire} \PYG{n}{finished}

\PYG{n}{Generating} \PYG{n}{more} \PYG{n}{randomly} \PYG{n}{selected} \PYG{n}{feature} \PYG{n}{sets} \PYG{k}{for} \PYG{n}{comparison}
\PYG{n}{best} \PYG{l+m+mi}{5} \PYG{n}{feature} \PYG{n}{sets} \PYG{n}{of} \PYG{n}{random} \PYG{n}{selection}\PYG{p}{:} \PYG{p}{[}\PYG{p}{[} \PYG{l+m+mf}{1.}          \PYG{l+m+mf}{0.}          \PYG{l+m+mf}{1.}          \PYG{l+m+mf}{0.}          \PYG{l+m+mf}{0.}          \PYG{l+m+mf}{1.}          \PYG{l+m+mf}{1.}
   \PYG{l+m+mf}{1.}          \PYG{l+m+mf}{0.}          \PYG{l+m+mf}{0.}          \PYG{l+m+mf}{0.}          \PYG{l+m+mf}{0.}          \PYG{l+m+mf}{0.}
   \PYG{l+m+mf}{0.72666667}\PYG{p}{]}
 \PYG{p}{[} \PYG{l+m+mf}{1.}          \PYG{l+m+mf}{1.}          \PYG{l+m+mf}{1.}          \PYG{l+m+mf}{0.}          \PYG{l+m+mf}{0.}          \PYG{l+m+mf}{1.}          \PYG{l+m+mf}{0.}
   \PYG{l+m+mf}{1.}          \PYG{l+m+mf}{1.}          \PYG{l+m+mf}{0.}          \PYG{l+m+mf}{1.}          \PYG{l+m+mf}{1.}          \PYG{l+m+mf}{0.}
   \PYG{l+m+mf}{0.72333333}\PYG{p}{]}
 \PYG{p}{[} \PYG{l+m+mf}{0.}          \PYG{l+m+mf}{0.}          \PYG{l+m+mf}{0.}          \PYG{l+m+mf}{1.}          \PYG{l+m+mf}{1.}          \PYG{l+m+mf}{1.}          \PYG{l+m+mf}{1.}
   \PYG{l+m+mf}{1.}          \PYG{l+m+mf}{1.}          \PYG{l+m+mf}{1.}          \PYG{l+m+mf}{0.}          \PYG{l+m+mf}{1.}          \PYG{l+m+mf}{0.}          \PYG{l+m+mf}{0.71}      \PYG{p}{]}
 \PYG{p}{[} \PYG{l+m+mf}{1.}          \PYG{l+m+mf}{1.}          \PYG{l+m+mf}{0.}          \PYG{l+m+mf}{0.}          \PYG{l+m+mf}{0.}          \PYG{l+m+mf}{0.}          \PYG{l+m+mf}{1.}
   \PYG{l+m+mf}{1.}          \PYG{l+m+mf}{1.}          \PYG{l+m+mf}{0.}          \PYG{l+m+mf}{0.}          \PYG{l+m+mf}{1.}          \PYG{l+m+mf}{1.}
   \PYG{l+m+mf}{0.70333333}\PYG{p}{]}
 \PYG{p}{[} \PYG{l+m+mf}{1.}          \PYG{l+m+mf}{0.}          \PYG{l+m+mf}{1.}          \PYG{l+m+mf}{0.}          \PYG{l+m+mf}{0.}          \PYG{l+m+mf}{1.}          \PYG{l+m+mf}{1.}
   \PYG{l+m+mf}{1.}          \PYG{l+m+mf}{1.}          \PYG{l+m+mf}{1.}          \PYG{l+m+mf}{1.}          \PYG{l+m+mf}{0.}          \PYG{l+m+mf}{1.}
   \PYG{l+m+mf}{0.70333333}\PYG{p}{]}\PYG{p}{]}

\PYG{n}{Lowest} \PYG{n}{MSE} \PYG{n}{after} \PYG{l+m+mi}{50} \PYG{n}{random} \PYG{n}{SVM} \PYG{n}{runs}\PYG{p}{:} \PYG{l+m+mf}{0.726666666667}
\PYG{n}{Lowest} \PYG{n}{MSE} \PYG{n}{of} \PYG{n}{ForestFire} \PYG{n}{after} \PYG{l+m+mi}{30} \PYG{n}{initial} \PYG{n}{random} \PYG{n}{runs} \PYG{o+ow}{and} \PYG{l+m+mi}{20} \PYG{n}{guided} \PYG{n}{runs}\PYG{p}{:} \PYG{l+m+mf}{0.763333333333}
\PYG{n}{Performance} \PYG{k}{with} \PYG{n}{ForestFire} \PYG{n}{improved} \PYG{n}{by} \PYG{l+m+mf}{5.04587155963}\PYG{o}{\PYGZpc{}}
\PYG{n}{Execution} \PYG{n}{finished}

\PYG{n}{Found} \PYG{n}{Best} \PYG{n}{value} \PYG{k}{for} \PYG{n}{Random} \PYG{n}{Forest} \PYG{n}{Search} \PYG{n}{after} \PYG{l+m+mi}{30} \PYG{n}{initial} \PYG{n}{runs} \PYG{o+ow}{and} \PYG{l+m+mi}{11}\PYG{o}{/}\PYG{l+m+mi}{20} \PYG{n}{smart} \PYG{n}{runs}
\PYG{n}{Best} \PYG{n}{value} \PYG{k}{with} \PYG{n}{RF}\PYG{p}{:} \PYG{l+m+mf}{0.763333333333}

\PYG{n}{Found} \PYG{n}{Best} \PYG{n}{value} \PYG{k}{for} \PYG{n}{Random} \PYG{n}{Search} \PYG{n}{after} \PYG{l+m+mi}{18} \PYG{n}{random} \PYG{n}{runs}
\PYG{n}{Best} \PYG{n}{value} \PYG{k}{with} \PYG{n}{Random} \PYG{n}{Search}\PYG{p}{:} \PYG{l+m+mf}{0.726666666667}

\PYG{n}{Creating} \PYG{n}{Plots}

\PYG{p}{[}\PYG{n}{Finished} \PYG{o+ow}{in} \PYG{n}{xxx} \PYG{n}{s}\PYG{p}{]}
\end{sphinxVerbatim}

\sphinxstylestrong{Interpretation:}
\begin{quote}

In this example ForestFire was able to find the best solution of 76,3\% accuracy after 30 random and 11 guided runs.
Compared to random search accuracy could be improved by \textasciitilde{}5\%.
The best {\hyperref[\detokenize{Overview:mla}]{\sphinxcrossref{\DUrole{std,std-ref}{MLA}}}} run did ``see'' all features but the second.

Since Demo mode was turned on at the end two plots are produced:
\end{quote}
\begin{figure}[htbp]
\centering

\noindent\sphinxincludegraphics[scale=0.35]{{generic_run_current}.png}
\end{figure}
\begin{figure}[htbp]
\centering

\noindent\sphinxincludegraphics[scale=0.35]{{generic_run_all_time}.png}
\end{figure}

\begin{sphinxadmonition}{important}{Important:}
\sphinxstylestrong{Functions used in this chapter}
\index{main\_loop() (in module ForestFire.Main)}

\begin{fulllineitems}
\phantomsection\label{\detokenize{execution:ForestFire.Main.main_loop}}\pysiglinewithargsret{\sphinxcode{ForestFire.Main.}\sphinxbfcode{main\_loop}}{\emph{n\_start}, \emph{pruning}, \emph{min\_data}, \emph{n\_forests}, \emph{n\_trees}, \emph{n\_configs\_biased}, \emph{n\_configs\_unbiased}, \emph{multiplier\_stepup}, \emph{seen\_forests}, \emph{weight\_mean}, \emph{weight\_gradient}, \emph{scoref}, \emph{demo\_mode}, \emph{plot\_enable}}{}
Load raw data and Generate database for Random Forest. Iteratively build and burn down new Random Forests, predict the performance of new feature sets and compute two new feature sets per round.

Arguments:
\begin{itemize}
\item {} 
n\_start \{int\} -- number of runs before building first RF = number of data points in first RF; minimum = 4, default = 50

\item {} 
pruning \{float\} -- if greater than zero, branches of a Decision Tree will be pruned proportional to pruning value; default = 0

\item {} 
min\_data \{float\} -- minimum percentage of Datasets that is used in RF generation; default = 0.2

\item {} 
n\_forests \{int\} -- number of forests; minimum=1;  default = 25

\item {} 
n\_trees \{int\} -- \# number of trees that stand in a forest; min = 3; default = number of features x 3 x

\item {} 
n\_configs\_biased \{int\} -- \# number of deliberately chosen feature sets that get predicted in each forest; default = n\_trees x 5

\item {} 
n\_configs\_unbiased \{int\} -- \# number of randomly chosen feature sets that get predicted in each forest; default = n\_configs\_biased x0.2

\item {} 
multiplier\_stepup \{float\} -- \# sets how aggressively the feature importance changes; default = 0.25

\item {} 
seen\_forests \{int\} -- \# number of recent forests that are taken into acount for generating probability of the chosen feature sets default = 4

\item {} 
weight\_mean \{float\} -- \# weight of the mean in calculating the new probability for selecting future feature sets; default = 0.2

\item {} 
weight\_gradient \{bool\} -- \# weight of the gradient in calculating the new probability for selecting future feature sets; default = 0.8

\item {} 
scoref \{function\} -- \# which scoring metric should be used in the Decision Tree (available: entropy and giniimpurity); default = entropy

\item {} 
demo\_mode bool -- \# if true a comparison between the Random Forest driven Search and a random search is done

\item {} 
plot\_enable bool -- \# decide if at the end a plot should be generated , only possible in demo mode

\end{itemize}

\end{fulllineitems}

\end{sphinxadmonition}


\chapter{Building and Burning Random Forests}
\label{\detokenize{index:building-and-burning-random-forests}}\label{\detokenize{index:loop}}

\section{Decision Tree}
\label{\detokenize{DT::doc}}\label{\detokenize{DT:decision-tree}}\label{\detokenize{DT:singletree}}
corresponding file: \href{https://github.com/weinertmos/ForestFire/blob/master/source/ForestFire/Main.py}{Main.py}

The principle of building decision trees is based on the implementation of decision trees in \phantomsection\label{\detokenize{DT:id1}}{\hyperref[\detokenize{Overview:collective-intelligence}]{\sphinxcrossref{{[}Collective\_Intelligence{]}}}} by Toby Segaran.


\subsection{Base Class}
\label{\detokenize{DT:base-class}}
At the foundation of the ForestFire algorithm stands the {\hyperref[\detokenize{DT:decisionnode}]{\sphinxcrossref{\DUrole{std,std-ref}{decisionnode class}}}}.
It represents a node in a {\hyperref[\detokenize{Overview:term-decision-tree}]{\sphinxtermref{\DUrole{xref,std,std-term}{DT}}}} at which the decision is made into which branch (true or false) to proceed.
The whole tree is built up of nodes.
Each node itself can contain two more nodes - the true and false branch - which are themselves decisionnodes.
In this way a tree is constructed in which a set of data takes a certain path along the tree to get classified.
At each node it either enters the true or the false branch.
When a branch is reached with no further branches below, this is called a leaf node.
The leaf node contains the results which represent the classification a data set receives.
The results can be a single value - in this case the classification is 100\% this single value.
It can also consist of several values, e.g. value1 with 2 instances and value2 with 1 instance.
The result of this classification is ambiguous, so it is expressed as a probability: the classification is 1/3 value2 and 2/3 value1.


\subsection{Helper Functions}
\label{\detokenize{DT:helper-functions}}\label{\detokenize{DT:help}}
At each node two questions have to be answered:
\begin{itemize}
\item {} \begin{description}
\item[{By which feature (=column) should the next decision be made?}] \leavevmode
The feature that is chosen at the first node should be the one feature that separates the data set in the best possible way. Latter features are of less importance

\end{description}

\item {} 
By which value should the decision be made?

\end{itemize}

To answer those questions the data is iteratively split in every possible way.
This means it is split for every feature and within every feature it is split for every single value.

See {\hyperref[\detokenize{DT:divideset}]{\sphinxcrossref{\DUrole{std,std-ref}{divideset}}}}

Each of the resulting splits has to be evaluted with respect to ``how well'' the split separates the big list into two smaller lists.
For this three evaluation metrics can be chosen from:
\begin{itemize}
\item {} \begin{description}
\item[{{\hyperref[\detokenize{DT:giniimpurity}]{\sphinxcrossref{\DUrole{std,std-ref}{Gini Impurity}}}}}] \leavevmode
``Probability that a randomly placed item will be in the wrong category''

\end{description}

\item {} \begin{description}
\item[{{\hyperref[\detokenize{DT:entropy}]{\sphinxcrossref{\DUrole{std,std-ref}{Entropy}}}}}] \leavevmode
``How mixed is a list''

\end{description}

\item {} \begin{description}
\item[{{\hyperref[\detokenize{DT:variance}]{\sphinxcrossref{\DUrole{std,std-ref}{Variance}}}}}] \leavevmode
``How far apart do the numbers lie''

\end{description}

\end{itemize}

The evaluation metric returns the gini coefficient / entropy / variance of the list that it is presented with.
Both methods need information about how many unique elements are in one list.
See {\hyperref[\detokenize{DT:uniquecounts}]{\sphinxcrossref{\DUrole{std,std-ref}{uniquecounts}}}}.

After a tree is built its width and depth can be examined by {\hyperref[\detokenize{DT:getdepth}]{\sphinxcrossref{\DUrole{std,std-ref}{getdepth}}}} and {\hyperref[\detokenize{DT:getwidth}]{\sphinxcrossref{\DUrole{std,std-ref}{getwidth}}}}.
A tree's depth is the maximum number of decisions that can be made before reaching a leaf node plus 1 (A tree stump that has no branches by definition still has a depth of 1).
A tree's width is the number of leaves it contains, i.e. number of nodes that have entries in their results property.


\subsection{Building a tree}
\label{\detokenize{DT:building-a-tree}}
Starting with a root node and the whole provided data set the {\hyperref[\detokenize{DT:buildtree}]{\sphinxcrossref{\DUrole{std,std-ref}{buildtree}}}} function recursively
loops through the following steps and builds up the tree structure:
\begin{enumerate}
\item {} 
create a decisionnode

\item {} 
calculate score (entropy / gini coefficient / variance) of current list

\item {} 
divide list into every possible split

\item {} 
evaluate each split according to evaluation metric

\item {} 
split the list into true and false branches according to best evaluated split

\item {} 
If no split is better than the current list no split is performed and results are stored, tree is returned

\item {} 
If true and false branches are created, start at 1.

\end{enumerate}

An example tree can look like {\hyperref[\detokenize{DT:treeview}]{\sphinxcrossref{\DUrole{std,std-ref}{this}}}}.
The first node checks if the value of the third column is \textgreater{}= 21.
If yes it continues to the right and checks column 0 if the value is equal to `slashdot'.
If yes the prediction for the new data set will be 50\% None and 50\% Premium since both values have appeared 1 time during trainging/building of the tree.

If the value of column 0 is instead not equal to `slashdot', there is another query at the next node for colum 0 wether it is equal to `google' and so on.
\begin{figure}[htbp]
\centering
\capstart

\noindent\sphinxincludegraphics[scale=0.8]{{treeview}.jpg}
\caption{Treeview.jpg}\label{\detokenize{DT:treeview}}\label{\detokenize{DT:id2}}\end{figure}


\subsubsection{Pruning a tree}
\label{\detokenize{DT:pruning-a-tree}}
At the deeper levels of a tree there might be splits that further reduce the entropy / gini coefficient / variance of the data, but only to a minor degree.
These further splits are not productive since they make the tree more complex but yield only small improvements.
There are two ways of tackling this problem.

One is to stop splitting the data if the split does not produce a significant reduction in entropy / gini coefficient / variance.
The danger in doing this is that there is a possibility that at an even later split there might be a significant reduction, but the algorithm can not forsee this.
This would lead to an premature stop.

The better way of dealing with the subject of overly complex trees is {\hyperref[\detokenize{DT:prune}]{\sphinxcrossref{\DUrole{std,std-ref}{pruning}}}}.
The pruning approach builds up the whole complex tree and then starts from its leaves going up.
It takes a look at the information gain that is made by the preceding split.
If the gain is lower than a threshold specified by the \sphinxstyleemphasis{pruning} hyperparameter in {\hyperref[\detokenize{execution:execution}]{\sphinxcrossref{\DUrole{std,std-ref}{Execution}}}} it will reunite the two leaves into one single leaf.
This way no meaningful splits are abandoned but complexity can be reduced

In the {\hyperref[\detokenize{DT:treeview}]{\sphinxcrossref{\DUrole{std,std-ref}{above example tree}}}} the rightmost leaf is the only place where pruning might have hapenned.
Before pruning `None' and `Premium' could have been located in separate leaves.
If the information gain from splitting the two was below the defined threshold, those two leaves would get pruned into one single leaf.
Still, only by looking at the finished tree one cannot tell if the tree was pruned or if it has been built this way (meaning that already during building there was no benefit in creating another split).

\begin{sphinxadmonition}{warning}{Warning:}
By default pruning is disabled (set to 0).
A reasonable value for pruning depends on the raw data.
Observe the output for ``wrongs'' on the console.
By default it should be quite small (\textless{}10\% of the total number of trees at most).
Try a value for pruning between 0 and 1 and only increase above 1 if the ``wrongs'' output does not get too big.

A ``wrong'' tree is a tree ``stump'' consisting of only one node.
Such a tree has no informational benefit.

Being an advanced hyperparameter pruning can greatly improve overall results as well as the number of runs it takes to find a good result.
But it also increases the risk of getting stuck in a local extremum or ending up with a lot of tree `stumps' that are useless for further information retrieval.
\end{sphinxadmonition}


\subsection{Classifying new observations}
\label{\detokenize{DT:classifying-new-observations}}
After a {\hyperref[\detokenize{Overview:term-decision-tree}]{\sphinxtermref{\DUrole{xref,std,std-term}{DT}}}} is built new observations can be classified.
This process can vividly be explained by starting at the top node and asking a simple yes or no question about the corresponding feature and value that is stored in the node.
If the answer for the new observastion is yes, the path follows the true branch of the node.
In case of a negated answer the false branch is pursued.

See {\hyperref[\detokenize{DT:treeview}]{\sphinxcrossref{\DUrole{std,std-ref}{Tree Image}}}} as an example. Visually the true branch is on the right hand side of the parent node, the false branch on the left.

The classification of new data is done with the help of the {\hyperref[\detokenize{DT:classify}]{\sphinxcrossref{\DUrole{std,std-ref}{classify function}}}}.

\begin{sphinxadmonition}{note}{Note:}
{\hyperref[\detokenize{DT:classify}]{\sphinxcrossref{\DUrole{std,std-ref}{classify}}}} is also able to handle missing data entries.
In this case both branches are followed and the result is weighted according to the number of entries they contain.
Since the ForestFire algorithm produces its own database from the raw data and the underlying {\hyperref[\detokenize{Overview:mla}]{\sphinxcrossref{\DUrole{std,std-ref}{MLA}}}} it is made sure that there are always entries present and the case of missing entries does not come to pass.
\end{sphinxadmonition}


\subsection{Visualizing a tree}
\label{\detokenize{DT:visualizing-a-tree}}
The following functions are for debugging purposes only.

The structure of the tree can be output to the console with the help of {\hyperref[\detokenize{DT:printtree}]{\sphinxcrossref{\DUrole{std,std-ref}{printtree}}}}.

An image of the tree can be created with the {\hyperref[\detokenize{DT:drawtree}]{\sphinxcrossref{\DUrole{std,std-ref}{drawtree}}}} function.
It makes use of {\hyperref[\detokenize{DT:drawnode}]{\sphinxcrossref{\DUrole{std,std-ref}{drawnode}}}}.


\subsection{Storing the tree structure}
\label{\detokenize{DT:storing-the-tree-structure}}
To {\hyperref[\detokenize{RF:random-forest}]{\sphinxcrossref{\DUrole{std,std-ref}{grow a Random Forest from single Decision Trees}}}} there must be a way to store whole trees and their structure in an array.
Unlike {\hyperref[\detokenize{DT:printtree}]{\sphinxcrossref{\DUrole{std,std-ref}{printtree}}}} and {\hyperref[\detokenize{DT:drawtree}]{\sphinxcrossref{\DUrole{std,std-ref}{drawtree}}}} where the tree is printed / drawn recursively by looping through the nodes.

This is done with the help of {\hyperref[\detokenize{DT:pathgen}]{\sphinxcrossref{\DUrole{std,std-ref}{path\_gen}}}} and {\hyperref[\detokenize{DT:pathgen2}]{\sphinxcrossref{\DUrole{std,std-ref}{path\_gen2}}}}.
By examining the last column of the path matrix that is returned by {\hyperref[\detokenize{DT:pathgen}]{\sphinxcrossref{\DUrole{std,std-ref}{path\_gen}}}} all results of the different leaf nodes can be reached.

Another usefull function is {\hyperref[\detokenize{DT:checkpath}]{\sphinxcrossref{\DUrole{std,std-ref}{check\_path}}}}. It takes as input a tree and a result (typically extracted from a path matrix) and checks wether the result is in that tree. This way it is possible to move along the branches of a tree and at each node check if it (still) contains a certain result, e.g. the best result of the whole tree. This is used for determining the importance of features in the following chapter about {\hyperref[\detokenize{RF:random-forest}]{\sphinxcrossref{\DUrole{std,std-ref}{growing a Random Forest}}}}

\begin{sphinxadmonition}{important}{Important:}
\sphinxstylestrong{Functions used in this chapter}
\phantomsection\label{\detokenize{DT:decisionnode}}\index{decisionnode (class in ForestFire.Main)}

\begin{fulllineitems}
\phantomsection\label{\detokenize{DT:ForestFire.Main.decisionnode}}\pysiglinewithargsret{\sphinxstrong{class }\sphinxcode{ForestFire.Main.}\sphinxbfcode{decisionnode}}{\emph{col=-1}, \emph{value=None}, \emph{results=None}, \emph{tb=None}, \emph{fb=None}}{}
Base class that a decision tree is built of.
\begin{description}
\item[{Keyword Arguments:}] \leavevmode\begin{itemize}
\item {} 
col \{integer\} -- column number = decision criterium for splitting data (default: \{-1\})

\item {} 
value \{integer/float/string\} -- value by which data gets split (default: \{None\})

\item {} 
results \{integer/float/string\} -- if node is an end node (=leaf) it contains the results (default: \{None\})

\item {} 
tb \{decisionnode\} -- next smaller node containing the true branch (default: \{None\})

\item {} 
fb \{decisionnode\} -- next smaller node containing the false branch (default: \{None\})

\end{itemize}

\end{description}

\end{fulllineitems}

\phantomsection\label{\detokenize{DT:divideset}}\index{divideset() (in module ForestFire.Main)}

\begin{fulllineitems}
\phantomsection\label{\detokenize{DT:ForestFire.Main.divideset}}\pysiglinewithargsret{\sphinxcode{ForestFire.Main.}\sphinxbfcode{divideset}}{\emph{rows}, \emph{column}, \emph{value}}{}
splits a data set into two separate sets according to the column and the value that is passed into.

If value is a number the comparison is done with \textless{}= and \textgreater{}=.
If value is not a number the exact value is compared
\begin{description}
\item[{Arguments:}] \leavevmode\begin{itemize}
\item {} 
rows \{list\} -- data set that is split

\item {} 
column\{integer\} -- column by which data gets split

\item {} 
value \{number/string\} -- value by which data gets split

\end{itemize}

\item[{Returns:}] \leavevmode
{[}list{]} -- two listso

\end{description}

\end{fulllineitems}

\phantomsection\label{\detokenize{DT:giniimpurity}}\index{giniimpurity() (in module ForestFire.Main)}

\begin{fulllineitems}
\phantomsection\label{\detokenize{DT:ForestFire.Main.giniimpurity}}\pysiglinewithargsret{\sphinxcode{ForestFire.Main.}\sphinxbfcode{giniimpurity}}{\emph{rows}}{}
Probability that a randomly placed item will be in the wrong category

Calculates the probability of each possible outcome by dividing the number of times that outcome occurs
by the total number of rows in the set.
It then adds up the products of all these probabilities.
This gives the overall chance that a row would be randomly assigned to the wrong outcome.
The higher this probability, the worse the split.
\begin{description}
\item[{Returns:}] \leavevmode
float -- probability of being in the wrong category

\end{description}

\end{fulllineitems}

\phantomsection\label{\detokenize{DT:entropy}}\index{entropy() (in module ForestFire.Main)}

\begin{fulllineitems}
\phantomsection\label{\detokenize{DT:ForestFire.Main.entropy}}\pysiglinewithargsret{\sphinxcode{ForestFire.Main.}\sphinxbfcode{entropy}}{\emph{rows}}{}
Entropy is the sum of p(x)log(p(x)) across all the different possible results --\textgreater{} how mixed is a list

Funciton calculates the frequency of each item (the number of times it appears divided by the total number of rows)
and applies these formulas:
\begin{align*}\!\begin{aligned}
p(i) = frequency(outcome) = \dfrac{count(outcome)}{count(total rows)}\\
Entropy = \sum(p(i)) \cdot  \log(p(i)) \ for \ all \ outcomes\\
\end{aligned}\end{align*}
The higher the entropy, the worse the split.
\begin{description}
\item[{Arguments:}] \leavevmode
rows \{list\} -- list to evaluate

\item[{Returns:}] \leavevmode
{[}float{]} -- entropy of the list

\end{description}

\end{fulllineitems}

\phantomsection\label{\detokenize{DT:variance}}\index{variance() (in module ForestFire.Main)}

\begin{fulllineitems}
\phantomsection\label{\detokenize{DT:ForestFire.Main.variance}}\pysiglinewithargsret{\sphinxcode{ForestFire.Main.}\sphinxbfcode{variance}}{\emph{rows}}{}
Evaluates how close together numerical values lie

Calculates mean and variance for given list
\begin{align*}\!\begin{aligned}
mean = \dfrac{\sum(entries)}{number \ of \ entries}\\
variance = \sum(entry - mean) ^ 2\\
\end{aligned}\end{align*}\begin{description}
\item[{Arguments:}] \leavevmode
rows \{list\} -- list to evaluate

\item[{Returns:}] \leavevmode
number -- variance of the list

\end{description}

\end{fulllineitems}

\phantomsection\label{\detokenize{DT:uniquecounts}}\index{uniquecounts() (in module ForestFire.Main)}

\begin{fulllineitems}
\phantomsection\label{\detokenize{DT:ForestFire.Main.uniquecounts}}\pysiglinewithargsret{\sphinxcode{ForestFire.Main.}\sphinxbfcode{uniquecounts}}{\emph{rows}}{}
evaluate how many unique elements are in a given list
\begin{description}
\item[{Arguments:}] \leavevmode
rows \{list\} -- evaluated list

\item[{Returns:}] \leavevmode
integer -- number of unique elements

\end{description}

\end{fulllineitems}

\phantomsection\label{\detokenize{DT:getdepth}}\index{getdepth() (in module ForestFire.Main)}

\begin{fulllineitems}
\phantomsection\label{\detokenize{DT:ForestFire.Main.getdepth}}\pysiglinewithargsret{\sphinxcode{ForestFire.Main.}\sphinxbfcode{getdepth}}{\emph{tree}}{}
returns the maximum number of consecutive nodes
\begin{description}
\item[{Arguments:}] \leavevmode
tree \{decisionnode\} -- tree to examine

\item[{Returns:}] \leavevmode
number -- maximum number of consecutive nodes

\end{description}

\end{fulllineitems}

\phantomsection\label{\detokenize{DT:getwidth}}\index{getwidth() (in module ForestFire.Main)}

\begin{fulllineitems}
\phantomsection\label{\detokenize{DT:ForestFire.Main.getwidth}}\pysiglinewithargsret{\sphinxcode{ForestFire.Main.}\sphinxbfcode{getwidth}}{\emph{tree}}{}
returns the number of leaves = endnodes in the tree
\begin{description}
\item[{Arguments:}] \leavevmode
tree \{decisionnode\} -- tree to examine

\item[{Returns:}] \leavevmode
number -- number of endnodes

\end{description}

\end{fulllineitems}

\phantomsection\label{\detokenize{DT:buildtree}}\index{buildtree() (in module ForestFire.Main)}

\begin{fulllineitems}
\phantomsection\label{\detokenize{DT:ForestFire.Main.buildtree}}\pysiglinewithargsret{\sphinxcode{ForestFire.Main.}\sphinxbfcode{buildtree}}{\emph{rows}, \emph{scoref}}{}
recursively builds decisionnode objects that form a decision tree

At each node the best possible split is calculated (depending on the evaluation metric).
If no further split is neccessary the remaining items and their number of occurence
are written in the results property.
\begin{description}
\item[{Arguments:}] \leavevmode
rows \{list\} -- dataset from which to build the tree
scoref \{function\} -- evaluation metric (entropy / gini coefficient)

\item[{Returns:}] \leavevmode
decisionnode -- either two decisionnodes for true and false branch or one decisionnode with results (leaf node)

\end{description}

\end{fulllineitems}

\phantomsection\label{\detokenize{DT:prune}}\index{prune() (in module ForestFire.Main)}

\begin{fulllineitems}
\phantomsection\label{\detokenize{DT:ForestFire.Main.prune}}\pysiglinewithargsret{\sphinxcode{ForestFire.Main.}\sphinxbfcode{prune}}{\emph{tree}, \emph{mingain}}{}
prunes the leaves of a tree in order to reduce complexity

By looking at the information gain that is achieved by splitting data further and further and checking if
it is above the mingain threshold, neighbouring leaves can be collapsed to a single leaf.
\begin{description}
\item[{Arguments:}] \leavevmode
tree \{decisionnode\} -- tree that gets pruned
mingain \{number\} -- threshold for pruning

\end{description}

\end{fulllineitems}

\phantomsection\label{\detokenize{DT:printtree}}\index{printtree() (in module ForestFire.Main)}

\begin{fulllineitems}
\phantomsection\label{\detokenize{DT:ForestFire.Main.printtree}}\pysiglinewithargsret{\sphinxcode{ForestFire.Main.}\sphinxbfcode{printtree}}{\emph{tree}, \emph{indent=' `}}{}
prints out the tree on the command line
\begin{description}
\item[{Arguments:}] \leavevmode
tree \{decisionnode\} -- tree that gets printed

\end{description}

\end{fulllineitems}

\phantomsection\label{\detokenize{DT:drawtree}}\index{drawtree() (in module ForestFire.Main)}

\begin{fulllineitems}
\phantomsection\label{\detokenize{DT:ForestFire.Main.drawtree}}\pysiglinewithargsret{\sphinxcode{ForestFire.Main.}\sphinxbfcode{drawtree}}{\emph{tree}, \emph{jpeg='tree.jpg'}}{}
visualization of the tree in a jpeg
\begin{description}
\item[{Arguments:}] \leavevmode
tree \{decisionnode\} -- tree to draw

\item[{Keyword Arguments:}] \leavevmode
jpeg \{str\} -- Name of the .jpg (default: \{`tree.jpg'\})

\end{description}

\end{fulllineitems}

\phantomsection\label{\detokenize{DT:drawnode}}\index{drawnode() (in module ForestFire.Main)}

\begin{fulllineitems}
\phantomsection\label{\detokenize{DT:ForestFire.Main.drawnode}}\pysiglinewithargsret{\sphinxcode{ForestFire.Main.}\sphinxbfcode{drawnode}}{\emph{draw}, \emph{tree}, \emph{x}, \emph{y}}{}
Helper Function for drawtree, draws a single node
\begin{description}
\item[{Arguments:}] \leavevmode
draw \{img\} -- node to be drawn
tree \{decisionnode\} -- tree that the node belongs to
x \{number\} -- x location
y \{number\} -- y location

\end{description}

\end{fulllineitems}

\phantomsection\label{\detokenize{DT:classify}}\index{classify() (in module ForestFire.Main)}

\begin{fulllineitems}
\phantomsection\label{\detokenize{DT:ForestFire.Main.classify}}\pysiglinewithargsret{\sphinxcode{ForestFire.Main.}\sphinxbfcode{classify}}{\emph{observation}, \emph{tree}}{}
takes a new data set that gets classified and the tree that determines the classification and returns the estimated result.
\begin{description}
\item[{Arguments:}] \leavevmode
observation \{numpy.array\} -- the new data set that gets classified, e.g. test data set
tree \{decisionnode\} -- tree that observation gets classified in

\item[{Returns:}] \leavevmode
data -- expected result

\end{description}

\end{fulllineitems}

\phantomsection\label{\detokenize{DT:pathgen}}\index{path\_gen() (in module ForestFire.Main)}

\begin{fulllineitems}
\phantomsection\label{\detokenize{DT:ForestFire.Main.path_gen}}\pysiglinewithargsret{\sphinxcode{ForestFire.Main.}\sphinxbfcode{path\_gen}}{\emph{tree}}{}
Create a path Matrix which contains the structure of the tree. Calls path\_gen2 to do so.
\begin{description}
\item[{Arguments:}] \leavevmode
tree \{decisionnode\} -- tree of which the data structure is stored

\item[{Returns:}] \leavevmode
numpy.array -- data structure of the tree, NaN means there is no more branch

\end{description}

\end{fulllineitems}

\phantomsection\label{\detokenize{DT:pathgen2}}\index{path\_gen2() (in module ForestFire.Main)}

\begin{fulllineitems}
\phantomsection\label{\detokenize{DT:ForestFire.Main.path_gen2}}\pysiglinewithargsret{\sphinxcode{ForestFire.Main.}\sphinxbfcode{path\_gen2}}{\emph{tree}, \emph{width}, \emph{depth}, \emph{path}, \emph{z2}, \emph{z1}}{}
Create a path Matrix which contains the structure of the tree.

creates a matrix `path' that represents the structure of the tree and the decisions made at each node, last column contains the average MSE at that leaf
the sooner a feature gets chosen as a split feature the more important it is (the farther on the left it appears in path matrix)
order that leaves are written in (top to bottom): function will crawl to the rightmost leaf first (positive side), then jump back up one level and move one step to the left (loop)
\begin{description}
\item[{Arguments:}] \leavevmode
tree \{decisionnode\} -- tree of which the data structure is stored
width \{int\} -- width of the tree
depth \{int\} -- depth of the tree
path \{{[}type{]}\} -- current path matrix, gets updated during function calls
z2 \{int\} -- control variable for current depth
z1 \{int\} -- control variable for current width

\item[{Returns:}] \leavevmode
numpy.array -- the structure of the tree

\end{description}

\end{fulllineitems}

\phantomsection\label{\detokenize{DT:checkpath}}\index{check\_path() (in module ForestFire.Main)}

\begin{fulllineitems}
\phantomsection\label{\detokenize{DT:ForestFire.Main.check_path}}\pysiglinewithargsret{\sphinxcode{ForestFire.Main.}\sphinxbfcode{check\_path}}{\emph{tree}, \emph{result}}{}
Check if a tree contains MSE\_min (= True) or not (= False)
\begin{description}
\item[{Arguments:}] \leavevmode
tree \{decisionnode\} -- tree that gets searched for result
result \{data\} -- result that the tree is searched for

\item[{Returns:}] \leavevmode
bool -- True if result is in the tree, false if not

\end{description}

\end{fulllineitems}



\begin{fulllineitems}
\pysiglinewithargsret{\sphinxcode{ForestFire.Main.}\sphinxbfcode{main\_loop}}{\emph{n\_start}, \emph{pruning}, \emph{min\_data}, \emph{n\_forests}, \emph{n\_trees}, \emph{n\_configs\_biased}, \emph{n\_configs\_unbiased}, \emph{multiplier\_stepup}, \emph{seen\_forests}, \emph{weight\_mean}, \emph{weight\_gradient}, \emph{scoref}, \emph{demo\_mode}, \emph{plot\_enable}}{}
Load raw data and Generate database for Random Forest. Iteratively build and burn down new Random Forests, predict the performance of new feature sets and compute two new feature sets per round.

Arguments:
\begin{itemize}
\item {} 
n\_start \{int\} -- number of runs before building first RF = number of data points in first RF; minimum = 4, default = 50

\item {} 
pruning \{float\} -- if greater than zero, branches of a Decision Tree will be pruned proportional to pruning value; default = 0

\item {} 
min\_data \{float\} -- minimum percentage of Datasets that is used in RF generation; default = 0.2

\item {} 
n\_forests \{int\} -- number of forests; minimum=1;  default = 25

\item {} 
n\_trees \{int\} -- \# number of trees that stand in a forest; min = 3; default = number of features x 3 x

\item {} 
n\_configs\_biased \{int\} -- \# number of deliberately chosen feature sets that get predicted in each forest; default = n\_trees x 5

\item {} 
n\_configs\_unbiased \{int\} -- \# number of randomly chosen feature sets that get predicted in each forest; default = n\_configs\_biased x0.2

\item {} 
multiplier\_stepup \{float\} -- \# sets how aggressively the feature importance changes; default = 0.25

\item {} 
seen\_forests \{int\} -- \# number of recent forests that are taken into acount for generating probability of the chosen feature sets default = 4

\item {} 
weight\_mean \{float\} -- \# weight of the mean in calculating the new probability for selecting future feature sets; default = 0.2

\item {} 
weight\_gradient \{bool\} -- \# weight of the gradient in calculating the new probability for selecting future feature sets; default = 0.8

\item {} 
scoref \{function\} -- \# which scoring metric should be used in the Decision Tree (available: entropy and giniimpurity); default = entropy

\item {} 
demo\_mode bool -- \# if true a comparison between the Random Forest driven Search and a random search is done

\item {} 
plot\_enable bool -- \# decide if at the end a plot should be generated , only possible in demo mode

\end{itemize}

\end{fulllineitems}

\end{sphinxadmonition}


\section{Random Forest}
\label{\detokenize{RF:random-forest}}\label{\detokenize{RF::doc}}\label{\detokenize{RF:id1}}
corresponding file: \href{https://github.com/weinertmos/ForestFire/blob/master/source/ForestFire/Main.py}{Main.py}


\subsection{Why a single Decision Tree is not enough}
\label{\detokenize{RF:why-a-single-decision-tree-is-not-enough}}
A single Decision Tree is already a fully fledged classifier that can be used to determine which features are of more importance than the rest.
The higher up in the hirarchy of the tree a feature stands the more decisive it is with regard to how well it splits the data in two separate lists.
The feature at the top node of a tree can be considered the most important one, a feature that appears on the lower levels is not as importnant.
Consequently a feature that is not at all appearing in the tree is even less important - it is even possible it distorts performance of the {\hyperref[\detokenize{Overview:mla}]{\sphinxcrossref{\DUrole{std,std-ref}{MLA}}}}.
As a consequence it might be reasonable to leave features with little importance out of the {\hyperref[\detokenize{Overview:mla}]{\sphinxcrossref{\DUrole{std,std-ref}{MLA}}}} and only present it with the important ones.

Applied to a convenient data set this approach can work.
But there are challenges that arise in most real world data sets.
The data can be clustered, i.e. a number of subsequent data sets might follow a certain pattern that is overlooked by a single Decision Tree because it is presented with all the data sets combined.
In addition a single tree that sees all features of the data set tends to be biased towards the most dominant features.

A logical implication to these two challenges is to introduce randomness:
\begin{itemize}
\item {} 
present a single tree with only a random subset of all data sets

\item {} 
present a single tree with only a random subsmet of all features

\end{itemize}

This will reduce the bias of the tree, but increase its variance.
By building multiple trees and averaging their results the variance can again be reduced.
The term for this construct is {\hyperref[\detokenize{Overview:term-random-forest}]{\sphinxtermref{\DUrole{xref,std,std-term}{Random Forest}}}}.


\subsection{Growing a Random Forest}
\label{\detokenize{RF:growing-a-random-forest}}
In {\hyperref[\detokenize{RF:buildforest}]{\sphinxcrossref{\DUrole{std,std-ref}{buildforest}}}} the Random Forest is built according to the following steps:
\begin{enumerate}
\item {} 
select random data and feature sets from the {\hyperref[\detokenize{Generate_Database:compute}]{\sphinxcrossref{\DUrole{std,std-ref}{generated Database}}}}

\item {} 
build a single tree with ``limited view''

\item {} 
{\hyperref[\detokenize{Overview:term-pruning}]{\sphinxtermref{\DUrole{xref,std,std-term}{prune}}}} the tree (if {\hyperref[\detokenize{execution:hyperparameters}]{\sphinxcrossref{\DUrole{std,std-ref}{enabled}}}})

\item {} 
reward features that lead to the best result

\item {} 
punish features that don't lead to the best result

\item {} 
Build next Tree

\end{enumerate}

After a new tree is built the feature importance for the whole Forest is {\hyperref[\detokenize{RF:update-rf}]{\sphinxcrossref{\DUrole{std,std-ref}{updated}}}} according to the number of appearances in the single trees.
The higher up a feature gets selected in a tree the higher it is rated. The punishment for features that don't lead to the best results is weaker than the reward for leading to the best results.
Features that are not included in the tree get neither a positive nor a negative rating.
Instead their probability of getting chosen as a biased feature set in {\hyperref[\detokenize{RF:pred-new}]{\sphinxcrossref{\DUrole{std,std-ref}{Predicting new feature sets}}}} is set to zero.


\subsection{Extracting the feature Importance from a Random Forest}
\label{\detokenize{RF:extracting-the-feature-importance-from-a-random-forest}}
From the so far completed Random Forests the resulting importance of each single feature can be extracted.
The terms ``importance'' and ``probability'' of a feature are used synonymously, since this value will be used for selecting new feature sets in {\hyperref[\detokenize{RF:pred-new}]{\sphinxcrossref{\DUrole{std,std-ref}{Predicting new feature sets}}}}.

In {\hyperref[\detokenize{RF:update-prob}]{\sphinxcrossref{\DUrole{std,std-ref}{update\_prob}}}} the current feature importance / probability is calculated.
Therefore several parameters are taken into account:
\begin{quote}
\begin{itemize}
\item {} 
seen\_forests: Only a fix number of recent Forest is taken into consideration

\item {} 
weight\_mean: From the last seen\_forest Forests the mean of each feature is calculated and weighed accordingly

\item {} 
weight\_gradient: From the last seen\_forest Forests the gradient of each feature is calculated and weighed accordingly

\end{itemize}
\phantomsection\label{\detokenize{RF:multi}}\begin{itemize}
\item {} 
multiplier: each feature probability is potentized by the current multiplier in order to achieve a more distinct distribution of the probabilites

\item {} 
prob\_current: the resulting probability for a feature is a combination of its recent trends for both gradient and mean (for details see {\hyperref[\detokenize{RF:update-prob}]{\sphinxcrossref{\DUrole{std,std-ref}{update\_prob}}}})

\end{itemize}
\end{quote}


\subsubsection{Multiplier Stepup}
\label{\detokenize{RF:multiplier-stepup}}
The multiplier that is {\hyperref[\detokenize{RF:multi}]{\sphinxcrossref{\DUrole{std,std-ref}{applied}}}} as an exponent to all single feature probabilities is a quantity that is scaled dynamically.
Depending on the {\hyperref[\detokenize{Overview:term-raw-data-set}]{\sphinxtermref{\DUrole{xref,std,std-term}{Raw data set}}}} set it is possible that the feature importances in a Random Forest are all very close to the average importance, hence resembling nothing more than a randomly chosen distribution.
In order to avoid this {\hyperref[\detokenize{Overview:term-forestfire}]{\sphinxtermref{\DUrole{xref,std,std-term}{ForestFire}}}} examines the importances of every single feature after a Random Forest is built.
If the highest feature importance does not lie above a certain threshold (default: 2 times the average importance) the multiplier is raised by the {\hyperref[\detokenize{execution:hyperparameters}]{\sphinxcrossref{\DUrole{std,std-ref}{hyperparameter multiplier\_stepup}}}}.


\subsection{Predicting new feature sets}
\label{\detokenize{RF:pred-new}}\label{\detokenize{RF:predicting-new-feature-sets}}
After the forest is built it can be used to make predictions (see {\hyperref[\detokenize{RF:forest-predict}]{\sphinxcrossref{\DUrole{std,std-ref}{forest\_predict}}}}) about the performance of arbitrary feature sets.
A new feature set candidate gets classified in every single forest.
The results are averaged.
From the vast amount of possible feature sets two different groups of feature sets are considered:
\begin{itemize}
\item {} 
feature sets biased according to the average importance of each feature (prob\_current from {\hyperref[\detokenize{RF:update-prob}]{\sphinxcrossref{\DUrole{std,std-ref}{update\_prob}}}})

\item {} 
entirely randomly chosen feature sets

\end{itemize}

The two {\hyperref[\detokenize{execution:hyperparameters}]{\sphinxcrossref{\DUrole{std,std-ref}{hyperparameters}}}} \sphinxstyleemphasis{n\_configs\_biased} and \sphinxstyleemphasis{n\_configs\_unbiased} determine the amount of feature sets that get tested.

For selecting the biased feature sets the probability of choosing a particular feature depends on its rating calculated in {\hyperref[\detokenize{RF:buildforest}]{\sphinxcrossref{\DUrole{std,std-ref}{buildforest}}}}.
The unbiased feature sets are chosen randomly.

Every candidate for future computation in the {\hyperref[\detokenize{Overview:mla}]{\sphinxcrossref{\DUrole{std,std-ref}{MLA}}}} gets predicted in every tree that stands in the {\hyperref[\detokenize{Overview:term-random-forest}]{\sphinxtermref{\DUrole{xref,std,std-term}{Random Forest}}}}. The results are incorporated by their average (mean) and variance.

Of all predicted feature sets two are chosen for the next computing run with the {\hyperref[\detokenize{Overview:mla}]{\sphinxcrossref{\DUrole{std,std-ref}{MLA}}}}. One with a high average (mean) and one with a high variance (respectively a combination of both, for details see {\hyperref[\detokenize{RF:forest-predict}]{\sphinxcrossref{\DUrole{std,std-ref}{forest\_predict}}}}).

If a feature set has already been computed before, it will not be computed again.
Instead its result is copied to the database.

The {\hyperref[\detokenize{Update_Database:update-database}]{\sphinxcrossref{\DUrole{std,std-ref}{Updating of the database}}}} depicts the last step in the ForestFire Loop.

\begin{sphinxadmonition}{important}{Important:}
\sphinxstylestrong{Functions used in this chapter}
\phantomsection\label{\detokenize{RF:buildforest}}\index{buildforest() (in module ForestFire.Main)}

\begin{fulllineitems}
\phantomsection\label{\detokenize{RF:ForestFire.Main.buildforest}}\pysiglinewithargsret{\sphinxcode{ForestFire.Main.}\sphinxbfcode{buildforest}}{\emph{data}, \emph{n\_trees}, \emph{scoref}, \emph{n\_feat}, \emph{min\_data}, \emph{pruning}}{}
Growing the Random Forest

The Random Forest consists of n\_trees. Each tree sees only a subset of the data and a subset of the features.
Important: a tree never sees the original data set, only the performance of the classifying algorithm
For significant conclusions enough trees must be generated in order to gain the statistical benefits that overcome bad outputs
\begin{description}
\item[{Arguments:}] \leavevmode\begin{itemize}
\item {} 
data \{numpy.array\} -- data set the Forest is built upon

\item {} 
n\_trees \{int\} -- number of trees in a Decision tree

\item {} 
scoref \{function\} -- scoring metric for finding new nodes

\item {} 
n\_feat \{int\} -- number of features in data

\item {} 
min\_data \{float\} -- minimum percentage of all data sets that a tree will see

\item {} 
pruning \{bool\} -- pruning enabled (\textgreater{}0) / disabled(=0)

\end{itemize}

\item[{Returns:}] \leavevmode\begin{itemize}
\item {} 
RF --  dictionary = importances of single features in the forest

\item {} 
prob\_current -- single value for importance, used for generating new biased feature sets

\item {} 
trees -- contains all single trees that stand in the Forest

\end{itemize}

\end{description}

\end{fulllineitems}

\phantomsection\label{\detokenize{RF:update-rf}}\index{update\_RF() (in module ForestFire.Main)}

\begin{fulllineitems}
\phantomsection\label{\detokenize{RF:ForestFire.Main.update_RF}}\pysiglinewithargsret{\sphinxcode{ForestFire.Main.}\sphinxbfcode{update\_RF}}{\emph{RF}, \emph{path}, \emph{tree}, \emph{rand\_feat}}{}
for each tree the features that lead to the leaf with the lowest Error will get rewarded.
Features that don't lead to the leaf with the lowest Error will get punished (only by 20\% of
the amount the ``good'' featurtes get rewarded).

RF is a dictionary that gets updated after a new tree is built and thus contains the cummulation of all
feature appearences in the whole forest.
\begin{description}
\item[{Arguments:}] \leavevmode\begin{itemize}
\item {} 
RF \{dict\} -- dictionary that counts occurrence / absence of different features

\item {} 
path \{numpy.array\} -- structure of the current tree

\item {} 
tree \{decisionnode\} -- tree that gets examined

\item {} 
rand\_feat \{list\} -- boolean mask of selected features (1 = selected, 0 = not selected)

\end{itemize}

\item[{Returns:}] \leavevmode\begin{itemize}
\item {} 
RF -- updated dictionary that counts occurrence / absence of different features

\end{itemize}

\end{description}

\end{fulllineitems}

\phantomsection\label{\detokenize{RF:update-prob}}\index{update\_prob() (in module ForestFire.Main)}

\begin{fulllineitems}
\phantomsection\label{\detokenize{RF:ForestFire.Main.update_prob}}\pysiglinewithargsret{\sphinxcode{ForestFire.Main.}\sphinxbfcode{update\_prob}}{\emph{Probability}, \emph{i}, \emph{weight\_mean}, \emph{weight\_gradient}, \emph{multiplier}, \emph{seen\_forests}}{}
Calculates the current Importance / Probability of the single features

Based on the probabilities of each feature in past Forests a new current\_prob is calculated that takes into
account the mean and the gradient of the prior feature importances.
\begin{description}
\item[{Arguments:}] \leavevmode\begin{itemize}
\item {} 
Probability \{numpy array\} -- contains Importances of single features for all past Random Forests

\item {} 
i \{integer\} -- number of current Forest

\item {} 
weight\_mean \{float\} -- weight of the mean in calculating resulting probability

\item {} 
weight\_gradient \{float\} -- weight of the var in calculating resulting probability

\item {} 
multiplier \{float\} -- exponent for amplifying probabilities

\item {} 
seen\_forests \{integer\} -- number of before built forest that are considered

\end{itemize}

\item[{Returns:}] \leavevmode
prob\_current -- list of floats representing the calculated aggregation of recent feature importances

\end{description}

\end{fulllineitems}

\phantomsection\label{\detokenize{RF:forest-predict}}\index{forest\_predict() (in module ForestFire.Main)}

\begin{fulllineitems}
\phantomsection\label{\detokenize{RF:ForestFire.Main.forest_predict}}\pysiglinewithargsret{\sphinxcode{ForestFire.Main.}\sphinxbfcode{forest\_predict}}{\emph{data}, \emph{trees}, \emph{prob}, \emph{n\_configs}, \emph{biased}}{}
Predict performance of new feature sets

Predicts biased and unbiased feature sets in the before constructed Random Forest.
Feature sets are predicted in every single Decision Tree in the Random Forest.
Results are represented as (mean+0.1*var) and (variance+0.1*mean) for each feature set.
The two best feature sets are selected to be sent into the {\hyperref[\detokenize{Overview:mla}]{\sphinxcrossref{\DUrole{std,std-ref}{MLA}}}}.
\begin{description}
\item[{Arguments:}] \leavevmode\begin{itemize}
\item {} 
data \{numpy.array\} -- contains all previous computing runs

\item {} 
trees \{decisionnodes\} -- the trees that make up the Random Forest

\item {} 
prob \{array of floats\} -- probability that a feature gets chosen into a feature set

\item {} 
n\_configs \{int\} -- number of feature sets to be generated

\item {} 
biased \{bool\} -- true for biased feature selection, false for unbiased feature selection

\end{itemize}

\item[{Returns:}] \leavevmode\begin{itemize}
\item {} 
best mean -- highest average of all predicted feature sets

\item {} 
best feature set mean -- corresponding boolean list of features (0=feature not chosen, 1=feature chosen)

\item {} 
best var -- highest variance of all predicted feature sets

\item {} 
best feature set var -- corresponding boolean list of features (0=feature not chosen, 1=feature chosen)

\end{itemize}

\end{description}

\end{fulllineitems}



\begin{fulllineitems}
\pysiglinewithargsret{\sphinxcode{ForestFire.Main.}\sphinxbfcode{main\_loop}}{\emph{n\_start}, \emph{pruning}, \emph{min\_data}, \emph{n\_forests}, \emph{n\_trees}, \emph{n\_configs\_biased}, \emph{n\_configs\_unbiased}, \emph{multiplier\_stepup}, \emph{seen\_forests}, \emph{weight\_mean}, \emph{weight\_gradient}, \emph{scoref}, \emph{demo\_mode}, \emph{plot\_enable}}{}
Load raw data and Generate database for Random Forest. Iteratively build and burn down new Random Forests, predict the performance of new feature sets and compute two new feature sets per round.

Arguments:
\begin{itemize}
\item {} 
n\_start \{int\} -- number of runs before building first RF = number of data points in first RF; minimum = 4, default = 50

\item {} 
pruning \{float\} -- if greater than zero, branches of a Decision Tree will be pruned proportional to pruning value; default = 0

\item {} 
min\_data \{float\} -- minimum percentage of Datasets that is used in RF generation; default = 0.2

\item {} 
n\_forests \{int\} -- number of forests; minimum=1;  default = 25

\item {} 
n\_trees \{int\} -- \# number of trees that stand in a forest; min = 3; default = number of features x 3 x

\item {} 
n\_configs\_biased \{int\} -- \# number of deliberately chosen feature sets that get predicted in each forest; default = n\_trees x 5

\item {} 
n\_configs\_unbiased \{int\} -- \# number of randomly chosen feature sets that get predicted in each forest; default = n\_configs\_biased x0.2

\item {} 
multiplier\_stepup \{float\} -- \# sets how aggressively the feature importance changes; default = 0.25

\item {} 
seen\_forests \{int\} -- \# number of recent forests that are taken into acount for generating probability of the chosen feature sets default = 4

\item {} 
weight\_mean \{float\} -- \# weight of the mean in calculating the new probability for selecting future feature sets; default = 0.2

\item {} 
weight\_gradient \{bool\} -- \# weight of the gradient in calculating the new probability for selecting future feature sets; default = 0.8

\item {} 
scoref \{function\} -- \# which scoring metric should be used in the Decision Tree (available: entropy and giniimpurity); default = entropy

\item {} 
demo\_mode bool -- \# if true a comparison between the Random Forest driven Search and a random search is done

\item {} 
plot\_enable bool -- \# decide if at the end a plot should be generated , only possible in demo mode

\end{itemize}

\end{fulllineitems}

\end{sphinxadmonition}


\section{Update Database}
\label{\detokenize{Update_Database:update-database}}\label{\detokenize{Update_Database::doc}}\label{\detokenize{Update_Database:id1}}
In chapter {\hyperref[\detokenize{RF:random-forest}]{\sphinxcrossref{\DUrole{std,std-ref}{Random Forest}}}} the process of choosing two new feature sets is described.
Those newly chosen feature sets and their performance are added to the initially {\hyperref[\detokenize{Generate_Database:compute}]{\sphinxcrossref{\DUrole{std,std-ref}{generated database}}}}.

See {\hyperref[\detokenize{Update_Database:update-database-fun}]{\sphinxcrossref{\DUrole{std,std-ref}{update\_database}}}} for details.

\begin{sphinxadmonition}{important}{Important:}
\sphinxstylestrong{Functions used in this chapter}
\phantomsection\label{\detokenize{Update_Database:update-database-fun}}\index{update\_database() (in module ForestFire.Main)}

\begin{fulllineitems}
\phantomsection\label{\detokenize{Update_Database:ForestFire.Main.update_database}}\pysiglinewithargsret{\sphinxcode{ForestFire.Main.}\sphinxbfcode{update\_database}}{\emph{X}, \emph{y}, \emph{data}, \emph{mask\_best\_featureset}, \emph{X\_test}, \emph{y\_test}}{}
Appends newly tested feature sets and their result to the already calculated feature sets
\begin{description}
\item[{Arguments:}] \leavevmode\begin{itemize}
\item {} 
X \{numpy array\} -- X rat data sets

\item {} 
y \{numpy array\} -- y raw data sets

\item {} 
data \{{[}type{]}\} -- data set the Forest is built upon

\item {} 
mask\_best\_featureset \{bool\} -- feature set (1: feature contained, 0: feature not contained)

\item {} 
X\_test \{numpy array\} -- test data set

\item {} 
y\_test \{numpy array\} -- test data set

\end{itemize}

\item[{Returns:}] \leavevmode
data -- updated data base

\end{description}

\end{fulllineitems}



\begin{fulllineitems}
\pysiglinewithargsret{\sphinxcode{ForestFire.Main.}\sphinxbfcode{main\_loop}}{\emph{n\_start}, \emph{pruning}, \emph{min\_data}, \emph{n\_forests}, \emph{n\_trees}, \emph{n\_configs\_biased}, \emph{n\_configs\_unbiased}, \emph{multiplier\_stepup}, \emph{seen\_forests}, \emph{weight\_mean}, \emph{weight\_gradient}, \emph{scoref}, \emph{demo\_mode}, \emph{plot\_enable}}{}
Load raw data and Generate database for Random Forest. Iteratively build and burn down new Random Forests, predict the performance of new feature sets and compute two new feature sets per round.

Arguments:
\begin{itemize}
\item {} 
n\_start \{int\} -- number of runs before building first RF = number of data points in first RF; minimum = 4, default = 50

\item {} 
pruning \{float\} -- if greater than zero, branches of a Decision Tree will be pruned proportional to pruning value; default = 0

\item {} 
min\_data \{float\} -- minimum percentage of Datasets that is used in RF generation; default = 0.2

\item {} 
n\_forests \{int\} -- number of forests; minimum=1;  default = 25

\item {} 
n\_trees \{int\} -- \# number of trees that stand in a forest; min = 3; default = number of features x 3 x

\item {} 
n\_configs\_biased \{int\} -- \# number of deliberately chosen feature sets that get predicted in each forest; default = n\_trees x 5

\item {} 
n\_configs\_unbiased \{int\} -- \# number of randomly chosen feature sets that get predicted in each forest; default = n\_configs\_biased x0.2

\item {} 
multiplier\_stepup \{float\} -- \# sets how aggressively the feature importance changes; default = 0.25

\item {} 
seen\_forests \{int\} -- \# number of recent forests that are taken into acount for generating probability of the chosen feature sets default = 4

\item {} 
weight\_mean \{float\} -- \# weight of the mean in calculating the new probability for selecting future feature sets; default = 0.2

\item {} 
weight\_gradient \{bool\} -- \# weight of the gradient in calculating the new probability for selecting future feature sets; default = 0.8

\item {} 
scoref \{function\} -- \# which scoring metric should be used in the Decision Tree (available: entropy and giniimpurity); default = entropy

\item {} 
demo\_mode bool -- \# if true a comparison between the Random Forest driven Search and a random search is done

\item {} 
plot\_enable bool -- \# decide if at the end a plot should be generated , only possible in demo mode

\end{itemize}

\end{fulllineitems}

\end{sphinxadmonition}


\chapter{Evaluation Mode}
\label{\detokenize{index:demo}}\label{\detokenize{index:evaluation-mode}}

\section{Evaluation Mode}
\label{\detokenize{Evaluation:evaluation-mode}}\label{\detokenize{Evaluation::doc}}
If the {\hyperref[\detokenize{execution:hyperparameters}]{\sphinxcrossref{\DUrole{std,std-ref}{hyperparameter}}}} \sphinxstyleemphasis{demo\_mode} is set to \sphinxstyleemphasis{True} the performance of ForestFire can be compared to a randomized search of new feature sets.
For every new feature set that gets calculated with ForestFire a randomly generated feature set is calculated.
Information about the performance is printed out after both ForestFire and the randomized search are done.


\subsection{Plots}
\label{\detokenize{Evaluation:plots}}
If the {\hyperref[\detokenize{execution:hyperparameters}]{\sphinxcrossref{\DUrole{std,std-ref}{hyperparameter}}}} \sphinxstyleemphasis{plot\_enable} is set to \sphinxstyleemphasis{True}, the performance of both are plotted over the number of generated feature sets.
Example Plots look like this:
\begin{figure}[htbp]
\centering

\noindent\sphinxincludegraphics[scale=0.35]{{results_current}.png}
\end{figure}
\begin{figure}[htbp]
\centering

\noindent\sphinxincludegraphics[scale=0.35]{{results_all_time}.png}
\end{figure}

\begin{sphinxadmonition}{important}{Important:}
\sphinxstylestrong{Functions used in this chapter}


\begin{fulllineitems}
\pysiglinewithargsret{\sphinxcode{ForestFire.Main.}\sphinxbfcode{main\_loop}}{\emph{n\_start}, \emph{pruning}, \emph{min\_data}, \emph{n\_forests}, \emph{n\_trees}, \emph{n\_configs\_biased}, \emph{n\_configs\_unbiased}, \emph{multiplier\_stepup}, \emph{seen\_forests}, \emph{weight\_mean}, \emph{weight\_gradient}, \emph{scoref}, \emph{demo\_mode}, \emph{plot\_enable}}{}
Load raw data and Generate database for Random Forest. Iteratively build and burn down new Random Forests, predict the performance of new feature sets and compute two new feature sets per round.

Arguments:
\begin{itemize}
\item {} 
n\_start \{int\} -- number of runs before building first RF = number of data points in first RF; minimum = 4, default = 50

\item {} 
pruning \{float\} -- if greater than zero, branches of a Decision Tree will be pruned proportional to pruning value; default = 0

\item {} 
min\_data \{float\} -- minimum percentage of Datasets that is used in RF generation; default = 0.2

\item {} 
n\_forests \{int\} -- number of forests; minimum=1;  default = 25

\item {} 
n\_trees \{int\} -- \# number of trees that stand in a forest; min = 3; default = number of features x 3 x

\item {} 
n\_configs\_biased \{int\} -- \# number of deliberately chosen feature sets that get predicted in each forest; default = n\_trees x 5

\item {} 
n\_configs\_unbiased \{int\} -- \# number of randomly chosen feature sets that get predicted in each forest; default = n\_configs\_biased x0.2

\item {} 
multiplier\_stepup \{float\} -- \# sets how aggressively the feature importance changes; default = 0.25

\item {} 
seen\_forests \{int\} -- \# number of recent forests that are taken into acount for generating probability of the chosen feature sets default = 4

\item {} 
weight\_mean \{float\} -- \# weight of the mean in calculating the new probability for selecting future feature sets; default = 0.2

\item {} 
weight\_gradient \{bool\} -- \# weight of the gradient in calculating the new probability for selecting future feature sets; default = 0.8

\item {} 
scoref \{function\} -- \# which scoring metric should be used in the Decision Tree (available: entropy and giniimpurity); default = entropy

\item {} 
demo\_mode bool -- \# if true a comparison between the Random Forest driven Search and a random search is done

\item {} 
plot\_enable bool -- \# decide if at the end a plot should be generated , only possible in demo mode

\end{itemize}

\end{fulllineitems}

\end{sphinxadmonition}


\chapter{Source Code}
\label{\detokenize{index:source-code}}

\section{Source Code}
\label{\detokenize{SourceCode:source-code}}\label{\detokenize{SourceCode::doc}}\label{\detokenize{SourceCode:sourcecode}}
In this section the complete source Code is presented in html friendly style.


\subsection{import\_data.py}
\label{\detokenize{SourceCode:import-data-py}}
\begin{sphinxVerbatim}[commandchars=\\\{\}]
\PYG{k+kn}{import} \PYG{n+nn}{numpy} \PYG{k}{as} \PYG{n+nn}{np}


\PYG{k}{def} \PYG{n+nf}{import\PYGZus{}data}\PYG{p}{(}\PYG{p}{)}\PYG{p}{:}
    \PYG{l+s+sd}{\PYGZdq{}\PYGZdq{}\PYGZdq{}import the raw data from two numpy arrays.}

\PYG{l+s+sd}{    Import raw data from two numpy arrays X.npy and y.npy. }
\PYG{l+s+sd}{    Set how train and test data are to be split for fix splits.}
\PYG{l+s+sd}{    Returns train/test splits as well as number of features.}

\PYG{l+s+sd}{    Returns:}

\PYG{l+s+sd}{        * X\PYGZus{}test \PYGZob{}np.array\PYGZcb{} \PYGZhy{}\PYGZhy{} result training data}
\PYG{l+s+sd}{        * X\PYGZus{}train \PYGZob{}np.array\PYGZcb{} \PYGZhy{}\PYGZhy{} feature training data}
\PYG{l+s+sd}{        * y\PYGZus{}test \PYGZob{}np.array\PYGZcb{} \PYGZhy{}\PYGZhy{} result test data}
\PYG{l+s+sd}{        * y\PYGZus{}train \PYGZob{}np.array\PYGZcb{} \PYGZhy{}\PYGZhy{} result training data}

\PYG{l+s+sd}{    \PYGZdq{}\PYGZdq{}\PYGZdq{}}
    \PYG{n}{X} \PYG{o}{=} \PYG{n}{np}\PYG{o}{.}\PYG{n}{load}\PYG{p}{(}\PYG{l+s+s1}{\PYGZsq{}}\PYG{l+s+s1}{X.npy}\PYG{l+s+s1}{\PYGZsq{}}\PYG{p}{)}
    \PYG{n}{X\PYGZus{}train} \PYG{o}{=} \PYG{n}{X}\PYG{p}{[}\PYG{p}{:}\PYG{l+m+mi}{500}\PYG{p}{]}
    \PYG{n}{X\PYGZus{}test} \PYG{o}{=} \PYG{n}{X}\PYG{p}{[}\PYG{l+m+mi}{500}\PYG{p}{:}\PYG{l+m+mi}{800}\PYG{p}{]}
    \PYG{n}{y} \PYG{o}{=} \PYG{n}{np}\PYG{o}{.}\PYG{n}{load}\PYG{p}{(}\PYG{l+s+s1}{\PYGZsq{}}\PYG{l+s+s1}{y.npy}\PYG{l+s+s1}{\PYGZsq{}}\PYG{p}{)}
    \PYG{n}{y\PYGZus{}train} \PYG{o}{=} \PYG{n}{y}\PYG{p}{[}\PYG{p}{:}\PYG{l+m+mi}{500}\PYG{p}{]}
    \PYG{n}{y\PYGZus{}test} \PYG{o}{=} \PYG{n}{y}\PYG{p}{[}\PYG{l+m+mi}{500}\PYG{p}{:}\PYG{l+m+mi}{800}\PYG{p}{]}
    \PYG{n}{n\PYGZus{}feat} \PYG{o}{=} \PYG{n+nb}{len}\PYG{p}{(}\PYG{n}{X}\PYG{p}{[}\PYG{l+m+mi}{0}\PYG{p}{]}\PYG{p}{)}

    \PYG{k}{return} \PYG{n}{X\PYGZus{}test}\PYG{p}{,} \PYG{n}{X\PYGZus{}train}\PYG{p}{,} \PYG{n}{y\PYGZus{}test}\PYG{p}{,} \PYG{n}{y\PYGZus{}train}\PYG{p}{,} \PYG{n}{n\PYGZus{}feat}
\end{sphinxVerbatim}


\subsection{compute.py}
\label{\detokenize{SourceCode:compute-py}}
\begin{sphinxVerbatim}[commandchars=\\\{\}]
\PYG{c+c1}{\PYGZsh{} Imports}
\PYG{k+kn}{import} \PYG{n+nn}{numpy} \PYG{k}{as} \PYG{n+nn}{np}
\PYG{k+kn}{from} \PYG{n+nn}{sklearn} \PYG{k}{import} \PYG{n}{svm}
\PYG{k+kn}{from} \PYG{n+nn}{sklearn}\PYG{n+nn}{.}\PYG{n+nn}{metrics} \PYG{k}{import} \PYG{n}{accuracy\PYGZus{}score}
\PYG{k+kn}{from} \PYG{n+nn}{sklearn}\PYG{n+nn}{.}\PYG{n+nn}{model\PYGZus{}selection} \PYG{k}{import} \PYG{n}{GridSearchCV}

\PYG{c+c1}{\PYGZsh{} make sure that a high score is better than a low score! If you use accuracy, a high accuracy is better than a low}
\PYG{c+c1}{\PYGZsh{} one. If you use Error (e.g. MSE) make sure it is negative (negative MSE)!}


\PYG{k}{def} \PYG{n+nf}{compute}\PYG{p}{(}\PYG{n}{X\PYGZus{}train}\PYG{p}{,} \PYG{n}{y\PYGZus{}train}\PYG{p}{,} \PYG{n}{mask\PYGZus{}sub\PYGZus{}features}\PYG{p}{,} \PYG{n}{X\PYGZus{}test}\PYG{p}{,} \PYG{n}{y\PYGZus{}test}\PYG{p}{)}\PYG{p}{:}
    \PYG{l+s+sd}{\PYGZdq{}\PYGZdq{}\PYGZdq{}Computes a new dataset for the Random Forest with the underlying machine learning algorithm.}

\PYG{l+s+sd}{    Configure your machine learning algorithm here.}
\PYG{l+s+sd}{    Add imports at the top of the file.}
\PYG{l+s+sd}{    If no train / test split is done during import, X\PYGZus{}train and X\PYGZus{}test are equal (y\PYGZus{}train and y\PYGZus{}test as well).}
\PYG{l+s+sd}{    In this case define your own splits with your machine learning algorithm.}

\PYG{l+s+sd}{    Arguments:}

\PYG{l+s+sd}{        * X\PYGZus{}train \PYGZob{}np.array\PYGZcb{} \PYGZhy{}\PYGZhy{} feature training data}
\PYG{l+s+sd}{        * y\PYGZus{}train \PYGZob{}np.array\PYGZcb{} \PYGZhy{}\PYGZhy{} result training data}
\PYG{l+s+sd}{        * mask\PYGZus{}sub\PYGZus{}features \PYGZob{}np.array\PYGZcb{} \PYGZhy{}\PYGZhy{} feature set = dedicated part of all features}
\PYG{l+s+sd}{        * X\PYGZus{}test \PYGZob{}np.array\PYGZcb{} \PYGZhy{}\PYGZhy{} result training data}
\PYG{l+s+sd}{        * y\PYGZus{}test \PYGZob{}np.array\PYGZcb{} \PYGZhy{}\PYGZhy{} result test data}

\PYG{l+s+sd}{    Returns:}
\PYG{l+s+sd}{        score \PYGZob{}np.float64\PYGZcb{} \PYGZhy{}\PYGZhy{} score of the selected feature set}
\PYG{l+s+sd}{    \PYGZdq{}\PYGZdq{}\PYGZdq{}}

    \PYG{c+c1}{\PYGZsh{} insert your own machine learning algorithm \PYGZsh{}}
    \PYG{n}{param\PYGZus{}grid} \PYG{o}{=} \PYG{p}{[}\PYG{p}{\PYGZob{}}\PYG{l+s+s1}{\PYGZsq{}}\PYG{l+s+s1}{C}\PYG{l+s+s1}{\PYGZsq{}}\PYG{p}{:} \PYG{n}{np}\PYG{o}{.}\PYG{n}{logspace}\PYG{p}{(}\PYG{o}{\PYGZhy{}}\PYG{l+m+mi}{1}\PYG{p}{,} \PYG{l+m+mi}{1}\PYG{p}{,} \PYG{l+m+mi}{16}\PYG{p}{)}\PYG{p}{,} \PYG{l+s+s1}{\PYGZsq{}}\PYG{l+s+s1}{gamma}\PYG{l+s+s1}{\PYGZsq{}}\PYG{p}{:} \PYG{n}{np}\PYG{o}{.}\PYG{n}{logspace}\PYG{p}{(}\PYG{o}{\PYGZhy{}}\PYG{l+m+mi}{1}\PYG{p}{,} \PYG{l+m+mi}{1}\PYG{p}{,} \PYG{l+m+mi}{16}\PYG{p}{)}\PYG{p}{\PYGZcb{}}\PYG{p}{]}
    \PYG{n}{clf} \PYG{o}{=} \PYG{n}{svm}\PYG{o}{.}\PYG{n}{SVC}\PYG{p}{(}\PYG{p}{)}  \PYG{c+c1}{\PYGZsh{} SVR for regression, SVC for classification}
    \PYG{n}{grid} \PYG{o}{=} \PYG{n}{GridSearchCV}\PYG{p}{(}\PYG{n}{clf}\PYG{p}{,} \PYG{n}{param\PYGZus{}grid}\PYG{p}{,} \PYG{n}{cv}\PYG{o}{=}\PYG{k+kc}{None}\PYG{p}{,} \PYG{n}{n\PYGZus{}jobs}\PYG{o}{=}\PYG{o}{\PYGZhy{}}\PYG{l+m+mi}{1}\PYG{p}{,} \PYG{n}{scoring}\PYG{o}{=}\PYG{l+s+s1}{\PYGZsq{}}\PYG{l+s+s1}{neg\PYGZus{}mean\PYGZus{}squared\PYGZus{}error}\PYG{l+s+s1}{\PYGZsq{}}\PYG{p}{,} \PYG{n}{pre\PYGZus{}dispatch}\PYG{o}{=}\PYG{l+m+mi}{8}\PYG{p}{)}
    \PYG{n}{grid}\PYG{o}{.}\PYG{n}{fit}\PYG{p}{(}\PYG{n}{X\PYGZus{}train}\PYG{p}{,} \PYG{n}{y\PYGZus{}train}\PYG{p}{)}
    \PYG{n}{y\PYGZus{}pred} \PYG{o}{=} \PYG{n}{grid}\PYG{o}{.}\PYG{n}{predict}\PYG{p}{(}\PYG{n}{X\PYGZus{}test}\PYG{p}{[}\PYG{p}{:}\PYG{p}{,} \PYG{n}{mask\PYGZus{}sub\PYGZus{}features}\PYG{p}{]}\PYG{p}{)}

    \PYG{c+c1}{\PYGZsh{} store the result in score \PYGZsh{}}
    \PYG{n}{score} \PYG{o}{=} \PYG{n}{accuracy\PYGZus{}score}\PYG{p}{(}\PYG{n}{y\PYGZus{}test}\PYG{p}{,} \PYG{n}{y\PYGZus{}pred}\PYG{p}{)}
    \PYG{c+c1}{\PYGZsh{} print score}
    \PYG{k}{return} \PYG{n}{score}
    \PYG{c+c1}{\PYGZsh{} print grid.cv\PYGZus{}results\PYGZus{}}
    \PYG{c+c1}{\PYGZsh{} print (grid.grid\PYGZus{}scores\PYGZus{})}
    \PYG{c+c1}{\PYGZsh{} print(grid.best\PYGZus{}score\PYGZus{})}
    \PYG{c+c1}{\PYGZsh{} print(grid.best\PYGZus{}params\PYGZus{})}
    \PYG{c+c1}{\PYGZsh{} return grid.best\PYGZus{}score\PYGZus{}}
\end{sphinxVerbatim}


\subsection{run\_ForestFire.py}
\label{\detokenize{SourceCode:run-forestfire-py}}
\begin{sphinxVerbatim}[commandchars=\\\{\}]
\PYG{k+kn}{import} \PYG{n+nn}{Main}
\PYG{k+kn}{import} \PYG{n+nn}{numpy} \PYG{k}{as} \PYG{n+nn}{np}

\PYG{n}{name} \PYG{o}{=} \PYG{l+s+s1}{\PYGZsq{}}\PYG{l+s+s1}{\PYGZus{}\PYGZus{}main\PYGZus{}\PYGZus{}}\PYG{l+s+s1}{\PYGZsq{}}
\PYG{c+c1}{\PYGZsh{} Hyperparameters \PYGZsh{}}

\PYG{c+c1}{\PYGZsh{} number of runs before building first Random Forest = number of data points in first RF; minimum = 4, default = 50}
\PYG{c+c1}{\PYGZsh{} adjust according to computational capabilities and demands of the underlying machine learning algorithm}
\PYG{n}{n\PYGZus{}start} \PYG{o}{=} \PYG{l+m+mi}{30}  \PYG{c+c1}{\PYGZsh{} default = 30e}
\PYG{c+c1}{\PYGZsh{} if pruning is greater than zero, branches of a Decision Tree will be pruned proportional to pruning value; default = 0}
\PYG{c+c1}{\PYGZsh{} advanced parameter. If set too high, all trees will be cut down to stumps. Increase carefully. Start with values between 0 and 1.}
\PYG{n}{pruning} \PYG{o}{=} \PYG{l+m+mf}{0.3}
\PYG{c+c1}{\PYGZsh{} minimum percentage of Datasets that is used in RF generation; default = 0.2}
\PYG{n}{min\PYGZus{}data} \PYG{o}{=} \PYG{l+m+mf}{0.1}
\PYG{c+c1}{\PYGZsh{} number of forests; minimum=1;  default = 25}
\PYG{c+c1}{\PYGZsh{} adjust according to computational capabilities. For each forest two new computational runs are done. default = 20}
\PYG{n}{n\PYGZus{}forests} \PYG{o}{=} \PYG{l+m+mi}{85}

\PYG{c+c1}{\PYGZsh{} number of trees that stand in a forest; min = 3; default = number of features * 3}
\PYG{n}{n\PYGZus{}trees} \PYG{o}{=} \PYG{l+s+s1}{\PYGZsq{}}\PYG{l+s+s1}{default}\PYG{l+s+s1}{\PYGZsq{}}
\PYG{c+c1}{\PYGZsh{} number of deliberately chosen feature sets that get predicted in each forest; default = n\PYGZus{}trees * 5}
\PYG{n}{n\PYGZus{}configs\PYGZus{}biased} \PYG{o}{=} \PYG{l+s+s1}{\PYGZsq{}}\PYG{l+s+s1}{default}\PYG{l+s+s1}{\PYGZsq{}}
\PYG{c+c1}{\PYGZsh{} number of randomly chosen feature sets that get predicted in each forest; default = n\PYGZus{}configs\PYGZus{}biased * 0.2}
\PYG{n}{n\PYGZus{}configs\PYGZus{}unbiased} \PYG{o}{=} \PYG{l+s+s1}{\PYGZsq{}}\PYG{l+s+s1}{default}\PYG{l+s+s1}{\PYGZsq{}}
\PYG{c+c1}{\PYGZsh{} sets how aggressively the feature importance changes; default = 0.25}
\PYG{c+c1}{\PYGZsh{} higher values will increase pressure on how often promising features will be selected.}
\PYG{c+c1}{\PYGZsh{} advanced parameter, adjust carefully. If set too high the risk of runnning into local extrema rises.}
\PYG{n}{multiplier\PYGZus{}stepup} \PYG{o}{=} \PYG{l+s+s1}{\PYGZsq{}}\PYG{l+s+s1}{default}\PYG{l+s+s1}{\PYGZsq{}}
\PYG{c+c1}{\PYGZsh{} number of recent forests that are taken into acount for generating probability of the chosen feature sets default = 0.01}
\PYG{n}{seen\PYGZus{}forests} \PYG{o}{=} \PYG{l+s+s1}{\PYGZsq{}}\PYG{l+s+s1}{default}\PYG{l+s+s1}{\PYGZsq{}}
\PYG{c+c1}{\PYGZsh{} the chosen feature sets default = 4}

\PYG{c+c1}{\PYGZsh{} weight of the mean in calculating the new probability for selecting future feature sets; default = 0.2}
\PYG{n}{weight\PYGZus{}mean} \PYG{o}{=} \PYG{l+s+s1}{\PYGZsq{}}\PYG{l+s+s1}{default}\PYG{l+s+s1}{\PYGZsq{}}
\PYG{c+c1}{\PYGZsh{} weight of the gradient in calculating the new probability for selecting future feature sets; default = 0.8}
\PYG{n}{weight\PYGZus{}gradient} \PYG{o}{=} \PYG{l+s+s1}{\PYGZsq{}}\PYG{l+s+s1}{default}\PYG{l+s+s1}{\PYGZsq{}}

\PYG{c+c1}{\PYGZsh{} which scoring metric should be used in the Decision Tree (available: entropy, giniimpurity and variance); default = entropy}
\PYG{c+c1}{\PYGZsh{} select variance for numerical values in y only}
\PYG{n}{scoref} \PYG{o}{=} \PYG{l+s+s1}{\PYGZsq{}}\PYG{l+s+s1}{entropy}\PYG{l+s+s1}{\PYGZsq{}}
\PYG{c+c1}{\PYGZsh{} set random seed for repeatabilit; comment out if no repeatability is required; default = 1}
\PYG{n}{np}\PYG{o}{.}\PYG{n}{random}\PYG{o}{.}\PYG{n}{seed}\PYG{p}{(}\PYG{l+m+mi}{10}\PYG{p}{)}

\PYG{c+c1}{\PYGZsh{} if true a comparison between the Random Forest driven Search and a random search is done}
\PYG{n}{demo\PYGZus{}mode} \PYG{o}{=} \PYG{k+kc}{True}
\PYG{c+c1}{\PYGZsh{} decide if at the end a plot should be generated , only valid in demo mode}
\PYG{n}{plot\PYGZus{}enable} \PYG{o}{=} \PYG{k+kc}{True}

\PYG{k}{if} \PYG{n}{name} \PYG{o}{==} \PYG{l+s+s1}{\PYGZsq{}}\PYG{l+s+s1}{\PYGZus{}\PYGZus{}main\PYGZus{}\PYGZus{}}\PYG{l+s+s1}{\PYGZsq{}}\PYG{p}{:}
    \PYG{n}{Main}\PYG{o}{.}\PYG{n}{main\PYGZus{}loop}\PYG{p}{(}\PYG{n}{n\PYGZus{}start}\PYG{p}{,} \PYG{n}{pruning}\PYG{p}{,} \PYG{n}{min\PYGZus{}data}\PYG{p}{,} \PYG{n}{n\PYGZus{}forests}\PYG{p}{,} \PYG{n}{n\PYGZus{}trees}\PYG{p}{,} \PYG{n}{n\PYGZus{}configs\PYGZus{}biased}\PYG{p}{,} \PYG{n}{n\PYGZus{}configs\PYGZus{}unbiased}\PYG{p}{,} \PYG{n}{multiplier\PYGZus{}stepup}\PYG{p}{,} \PYG{n}{seen\PYGZus{}forests}\PYG{p}{,}
                   \PYG{n}{weight\PYGZus{}mean}\PYG{p}{,} \PYG{n}{weight\PYGZus{}gradient}\PYG{p}{,} \PYG{n}{scoref}\PYG{p}{,} \PYG{n}{demo\PYGZus{}mode}\PYG{p}{,} \PYG{n}{plot\PYGZus{}enable}\PYG{p}{)}
\end{sphinxVerbatim}


\subsection{Main.py}
\label{\detokenize{SourceCode:main-py}}
\begin{sphinxVerbatim}[commandchars=\\\{\}]
\PYG{c+c1}{\PYGZsh{}  Imports}
\PYG{k+kn}{import} \PYG{n+nn}{numpy} \PYG{k}{as} \PYG{n+nn}{np}
\PYG{k+kn}{import} \PYG{n+nn}{matplotlib}\PYG{n+nn}{.}\PYG{n+nn}{pyplot} \PYG{k}{as} \PYG{n+nn}{plt}
\PYG{k+kn}{from} \PYG{n+nn}{sklearn} \PYG{k}{import} \PYG{n}{preprocessing}

\PYG{c+c1}{\PYGZsh{} For Debugging only}
\PYG{k+kn}{from} \PYG{n+nn}{PIL} \PYG{k}{import} \PYG{n}{Image}\PYG{p}{,} \PYG{n}{ImageDraw}

\PYG{c+c1}{\PYGZsh{} Import files}
\PYG{k+kn}{from} \PYG{n+nn}{compute} \PYG{k}{import} \PYG{n}{compute}
\PYG{k+kn}{from} \PYG{n+nn}{import\PYGZus{}data} \PYG{k}{import} \PYG{n}{import\PYGZus{}data}

\PYG{c+c1}{\PYGZsh{} matplotlib.use(\PYGZsq{}TkAgg\PYGZsq{})  \PYGZsh{} set Backend}


\PYG{c+c1}{\PYGZsh{} change settings}
\PYG{n}{np}\PYG{o}{.}\PYG{n}{set\PYGZus{}printoptions}\PYG{p}{(}\PYG{n}{threshold}\PYG{o}{=}\PYG{n}{np}\PYG{o}{.}\PYG{n}{inf}\PYG{p}{)}   \PYG{c+c1}{\PYGZsh{} print whole numpy array in console}
\PYG{n}{np}\PYG{o}{.}\PYG{n}{seterr}\PYG{p}{(}\PYG{n}{divide}\PYG{o}{=}\PYG{l+s+s1}{\PYGZsq{}}\PYG{l+s+s1}{ignore}\PYG{l+s+s1}{\PYGZsq{}}\PYG{p}{,} \PYG{n}{invalid}\PYG{o}{=}\PYG{l+s+s1}{\PYGZsq{}}\PYG{l+s+s1}{ignore}\PYG{l+s+s1}{\PYGZsq{}}\PYG{p}{)}  \PYG{c+c1}{\PYGZsh{} ignore warnings if dividing by zero or NaN}
\PYG{n}{plt}\PYG{o}{.}\PYG{n}{style}\PYG{o}{.}\PYG{n}{use}\PYG{p}{(}\PYG{l+s+s1}{\PYGZsq{}}\PYG{l+s+s1}{bmh}\PYG{l+s+s1}{\PYGZsq{}}\PYG{p}{)}
\PYG{n}{plt}\PYG{o}{.}\PYG{n}{rcParams}\PYG{o}{.}\PYG{n}{update}\PYG{p}{(}\PYG{p}{\PYGZob{}}\PYG{l+s+s1}{\PYGZsq{}}\PYG{l+s+s1}{font.size}\PYG{l+s+s1}{\PYGZsq{}}\PYG{p}{:} \PYG{l+m+mi}{25}\PYG{p}{\PYGZcb{}}\PYG{p}{)}

\PYG{c+c1}{\PYGZsh{} Definitions \PYGZsh{}}


\PYG{k}{def} \PYG{n+nf}{gen\PYGZus{}database}\PYG{p}{(}\PYG{n}{n\PYGZus{}start}\PYG{p}{,} \PYG{n}{X}\PYG{p}{,} \PYG{n}{y}\PYG{p}{,} \PYG{n}{X\PYGZus{}test}\PYG{p}{,} \PYG{n}{y\PYGZus{}test}\PYG{p}{)}\PYG{p}{:}
    \PYG{l+s+sd}{\PYGZdq{}\PYGZdq{}\PYGZdq{}Runs the underlying :ref:{}`MLA \PYGZlt{}MLA\PYGZgt{}{}` *n\PYGZus{}start* times to generate a database from which Random Forests can be built.}

\PYG{l+s+sd}{    Arguments:}
\PYG{l+s+sd}{        * n\PYGZus{}start \PYGZob{}int\PYGZcb{} \PYGZhy{}\PYGZhy{} number of times the underlying :ref:{}`MLA \PYGZlt{}MLA\PYGZgt{}{}` is executed}
\PYG{l+s+sd}{        * X \PYGZob{}numpy.array\PYGZcb{} \PYGZhy{}\PYGZhy{} raw data}
\PYG{l+s+sd}{        * y \PYGZob{}numpy.array\PYGZcb{} \PYGZhy{}\PYGZhy{} raw data}
\PYG{l+s+sd}{        * X\PYGZus{}test \PYGZob{}numpy.array\PYGZcb{} \PYGZhy{}\PYGZhy{} test data}
\PYG{l+s+sd}{        * y\PYGZus{}test \PYGZob{}numpy.array\PYGZcb{} \PYGZhy{}\PYGZhy{} test data}

\PYG{l+s+sd}{    Returns:}
\PYG{l+s+sd}{        [numpy.array] \PYGZhy{}\PYGZhy{} data set containing feature sets and corresponding results}
\PYG{l+s+sd}{    \PYGZdq{}\PYGZdq{}\PYGZdq{}}
    \PYG{n}{X\PYGZus{}DT} \PYG{o}{=} \PYG{n}{np}\PYG{o}{.}\PYG{n}{zeros}\PYG{p}{(}\PYG{p}{(}\PYG{n}{n\PYGZus{}start}\PYG{p}{,} \PYG{n+nb}{len}\PYG{p}{(}\PYG{n}{X}\PYG{p}{[}\PYG{l+m+mi}{0}\PYG{p}{]}\PYG{p}{)}\PYG{p}{)}\PYG{p}{,} \PYG{n}{dtype}\PYG{o}{=}\PYG{n+nb}{bool}\PYG{p}{)}  \PYG{c+c1}{\PYGZsh{} Prelocate Memory}
    \PYG{c+c1}{\PYGZsh{} print X\PYGZus{}DT}
    \PYG{n}{y\PYGZus{}DT} \PYG{o}{=} \PYG{n}{np}\PYG{o}{.}\PYG{n}{zeros}\PYG{p}{(}\PYG{p}{(}\PYG{n}{n\PYGZus{}start}\PYG{p}{,} \PYG{l+m+mi}{1}\PYG{p}{)}\PYG{p}{)}  \PYG{c+c1}{\PYGZsh{} Prelocate Memory}

    \PYG{c+c1}{\PYGZsh{} create SVMs that can only see subset of features}
    \PYG{k}{for} \PYG{n}{i} \PYG{o+ow}{in} \PYG{n+nb}{range}\PYG{p}{(}\PYG{n}{n\PYGZus{}start}\PYG{p}{)}\PYG{p}{:}
        \PYG{c+c1}{\PYGZsh{} create random mask to select subgroup of features}
        \PYG{n}{mask\PYGZus{}sub\PYGZus{}features} \PYG{o}{=} \PYG{n}{np}\PYG{o}{.}\PYG{n}{zeros}\PYG{p}{(}\PYG{n+nb}{len}\PYG{p}{(}\PYG{n}{X}\PYG{p}{[}\PYG{l+m+mi}{0}\PYG{p}{]}\PYG{p}{)}\PYG{p}{,} \PYG{n}{dtype}\PYG{o}{=}\PYG{n+nb}{bool}\PYG{p}{)}  \PYG{c+c1}{\PYGZsh{} Prelocate Memory}
        \PYG{c+c1}{\PYGZsh{} mask\PYGZus{}sub\PYGZus{}data = np.zeros(len(X), dtype=bool)  \PYGZsh{} Prelocate Memory}
        \PYG{c+c1}{\PYGZsh{} selecting features: any number between 1 and all features are selected}
        \PYG{n}{size} \PYG{o}{=} \PYG{n}{np}\PYG{o}{.}\PYG{n}{random}\PYG{o}{.}\PYG{n}{choice}\PYG{p}{(}\PYG{n+nb}{range}\PYG{p}{(}\PYG{n+nb}{len}\PYG{p}{(}\PYG{n}{X}\PYG{p}{[}\PYG{l+m+mi}{0}\PYG{p}{]}\PYG{p}{)} \PYG{o}{\PYGZhy{}} \PYG{l+m+mi}{1}\PYG{p}{)}\PYG{p}{)} \PYG{o}{+} \PYG{l+m+mi}{1}
        \PYG{n}{rand\PYGZus{}feat} \PYG{o}{=} \PYG{n}{np}\PYG{o}{.}\PYG{n}{random}\PYG{o}{.}\PYG{n}{choice}\PYG{p}{(}\PYG{n+nb}{range}\PYG{p}{(}\PYG{n+nb}{len}\PYG{p}{(}\PYG{n}{X}\PYG{p}{[}\PYG{l+m+mi}{0}\PYG{p}{]}\PYG{p}{)}\PYG{p}{)}\PYG{p}{,} \PYG{n}{size}\PYG{o}{=}\PYG{n}{size}\PYG{p}{,} \PYG{n}{replace}\PYG{o}{=}\PYG{k+kc}{True}\PYG{p}{,} \PYG{n}{p}\PYG{o}{=}\PYG{k+kc}{None}\PYG{p}{)}  \PYG{c+c1}{\PYGZsh{} in first run prob is None \PYGZhy{}\PYGZhy{}\PYGZgt{} all features are equally selected, in later runs prob is result of previous RF results}
        \PYG{n}{mask\PYGZus{}sub\PYGZus{}features}\PYG{p}{[}\PYG{n}{rand\PYGZus{}feat}\PYG{p}{]} \PYG{o}{=} \PYG{k+kc}{True}  \PYG{c+c1}{\PYGZsh{} set chosen features to True}

        \PYG{c+c1}{\PYGZsh{} Select Train and Test Data for subgroup}
        \PYG{c+c1}{\PYGZsh{} print X}
        \PYG{n}{X\PYGZus{}sub} \PYG{o}{=} \PYG{n}{X}\PYG{p}{[}\PYG{p}{:}\PYG{p}{,} \PYG{n}{mask\PYGZus{}sub\PYGZus{}features}\PYG{p}{]}  \PYG{c+c1}{\PYGZsh{} select only chosen features (still all datasets)}
        \PYG{c+c1}{\PYGZsh{} print len(X\PYGZus{}sub[0])}
        \PYG{c+c1}{\PYGZsh{} print X\PYGZus{}sub[0]}

        \PYG{c+c1}{\PYGZsh{} compute subgroup}
        \PYG{c+c1}{\PYGZsh{} print X\PYGZus{}sub}
        \PYG{n}{y\PYGZus{}DT}\PYG{p}{[}\PYG{n}{i}\PYG{p}{]} \PYG{o}{=} \PYG{n}{compute}\PYG{p}{(}\PYG{n}{X\PYGZus{}sub}\PYG{p}{,} \PYG{n}{y}\PYG{p}{,} \PYG{n}{mask\PYGZus{}sub\PYGZus{}features}\PYG{p}{,} \PYG{n}{X\PYGZus{}test}\PYG{p}{,} \PYG{n}{y\PYGZus{}test}\PYG{p}{)}

        \PYG{c+c1}{\PYGZsh{} Save Data}
        \PYG{n}{X\PYGZus{}DT}\PYG{p}{[}\PYG{n}{i}\PYG{p}{]} \PYG{o}{=} \PYG{n}{mask\PYGZus{}sub\PYGZus{}features}  \PYG{c+c1}{\PYGZsh{} for the Decision Tree / Random Forest the X values are the information about whether an SVM has seen a certain feature or not}
    \PYG{c+c1}{\PYGZsh{} print X\PYGZus{}DT}
    \PYG{c+c1}{\PYGZsh{} print y\PYGZus{}DT}

    \PYG{c+c1}{\PYGZsh{} merge X and y values}
    \PYG{n}{Data} \PYG{o}{=} \PYG{n}{np}\PYG{o}{.}\PYG{n}{concatenate}\PYG{p}{(}\PYG{p}{(}\PYG{n}{X\PYGZus{}DT}\PYG{p}{,} \PYG{n}{y\PYGZus{}DT}\PYG{p}{)}\PYG{p}{,} \PYG{n}{axis}\PYG{o}{=}\PYG{l+m+mi}{1}\PYG{p}{)}  \PYG{c+c1}{\PYGZsh{} this Dataset goes into the Decision Tree / Random Forest}
    \PYG{k}{return} \PYG{n}{Data}


\PYG{c+c1}{\PYGZsh{} Functions for Generating Database for RF}


\PYG{c+c1}{\PYGZsh{} Decision Tree}


\PYG{c+c1}{\PYGZsh{} class definition}
\PYG{k}{class} \PYG{n+nc}{decisionnode}\PYG{p}{:}
    \PYG{l+s+sd}{\PYGZdq{}\PYGZdq{}\PYGZdq{}Base class that a decision tree is built of.}


\PYG{l+s+sd}{    Keyword Arguments:}
\PYG{l+s+sd}{        * col \PYGZob{}integer\PYGZcb{} \PYGZhy{}\PYGZhy{} column number = decision criterium for splitting data (default: \PYGZob{}\PYGZhy{}1\PYGZcb{})}
\PYG{l+s+sd}{        * value \PYGZob{}integer/float/string\PYGZcb{} \PYGZhy{}\PYGZhy{} value by which data gets split (default: \PYGZob{}None\PYGZcb{})}
\PYG{l+s+sd}{        * results \PYGZob{}integer/float/string\PYGZcb{} \PYGZhy{}\PYGZhy{} if node is an end node (=leaf) it contains the results (default: \PYGZob{}None\PYGZcb{})}
\PYG{l+s+sd}{        * tb \PYGZob{}decisionnode\PYGZcb{} \PYGZhy{}\PYGZhy{} next smaller node containing the true branch (default: \PYGZob{}None\PYGZcb{})}
\PYG{l+s+sd}{        * fb \PYGZob{}decisionnode\PYGZcb{} \PYGZhy{}\PYGZhy{} next smaller node containing the false branch (default: \PYGZob{}None\PYGZcb{})}
\PYG{l+s+sd}{    \PYGZdq{}\PYGZdq{}\PYGZdq{}}

    \PYG{k}{def} \PYG{n+nf}{\PYGZus{}\PYGZus{}init\PYGZus{}\PYGZus{}}\PYG{p}{(}\PYG{n+nb+bp}{self}\PYG{p}{,} \PYG{n}{col}\PYG{o}{=}\PYG{o}{\PYGZhy{}}\PYG{l+m+mi}{1}\PYG{p}{,} \PYG{n}{value}\PYG{o}{=}\PYG{k+kc}{None}\PYG{p}{,} \PYG{n}{results}\PYG{o}{=}\PYG{k+kc}{None}\PYG{p}{,} \PYG{n}{tb}\PYG{o}{=}\PYG{k+kc}{None}\PYG{p}{,} \PYG{n}{fb}\PYG{o}{=}\PYG{k+kc}{None}\PYG{p}{)}\PYG{p}{:}
        \PYG{n+nb+bp}{self}\PYG{o}{.}\PYG{n}{col} \PYG{o}{=} \PYG{n}{col}
        \PYG{n+nb+bp}{self}\PYG{o}{.}\PYG{n}{value} \PYG{o}{=} \PYG{n}{value}
        \PYG{n+nb+bp}{self}\PYG{o}{.} \PYG{n}{results} \PYG{o}{=} \PYG{n}{results}
        \PYG{n+nb+bp}{self}\PYG{o}{.}\PYG{n}{tb} \PYG{o}{=} \PYG{n}{tb}
        \PYG{n+nb+bp}{self}\PYG{o}{.}\PYG{n}{fb} \PYG{o}{=} \PYG{n}{fb}

\PYG{c+c1}{\PYGZsh{} Functions for DT}


\PYG{c+c1}{\PYGZsh{} Divides a set on a specific column. Can handle numeric}
\PYG{c+c1}{\PYGZsh{} or nominal vlaues}
\PYG{k}{def} \PYG{n+nf}{divideset}\PYG{p}{(}\PYG{n}{rows}\PYG{p}{,} \PYG{n}{column}\PYG{p}{,} \PYG{n}{value}\PYG{p}{)}\PYG{p}{:}
    \PYG{l+s+sd}{\PYGZdq{}\PYGZdq{}\PYGZdq{} splits a data set into two separate sets according to the column and the value that is passed into.}

\PYG{l+s+sd}{    If value is a number the comparison is done with \PYGZlt{}= and \PYGZgt{}=.}
\PYG{l+s+sd}{    If value is not a number the exact value is compared}

\PYG{l+s+sd}{    Arguments:}
\PYG{l+s+sd}{        * rows \PYGZob{}list\PYGZcb{} \PYGZhy{}\PYGZhy{} data set that is split}
\PYG{l+s+sd}{        * column\PYGZob{}integer\PYGZcb{} \PYGZhy{}\PYGZhy{} column by which data gets split}
\PYG{l+s+sd}{        * value \PYGZob{}number/string\PYGZcb{} \PYGZhy{}\PYGZhy{} value by which data gets split}

\PYG{l+s+sd}{    Returns:}
\PYG{l+s+sd}{        [list] \PYGZhy{}\PYGZhy{} two listso}
\PYG{l+s+sd}{    \PYGZdq{}\PYGZdq{}\PYGZdq{}}
    \PYG{n}{split\PYGZus{}function} \PYG{o}{=} \PYG{k+kc}{None}  \PYG{c+c1}{\PYGZsh{} Prelocate}
    \PYG{k}{if} \PYG{n+nb}{isinstance}\PYG{p}{(}\PYG{n}{value}\PYG{p}{,} \PYG{n+nb}{int}\PYG{p}{)} \PYG{o+ow}{or} \PYG{n+nb}{isinstance}\PYG{p}{(}\PYG{n}{value}\PYG{p}{,} \PYG{n+nb}{float}\PYG{p}{)}\PYG{p}{:}
        \PYG{k}{def} \PYG{n+nf}{split\PYGZus{}function}\PYG{p}{(}\PYG{n}{row}\PYG{p}{)}\PYG{p}{:}
            \PYG{k}{return} \PYG{n}{row}\PYG{p}{[}\PYG{n}{column}\PYG{p}{]} \PYG{o}{\PYGZgt{}}\PYG{o}{=} \PYG{n}{value}  \PYG{c+c1}{\PYGZsh{} quick function definition}
    \PYG{k}{else}\PYG{p}{:}
        \PYG{k}{def} \PYG{n+nf}{split\PYGZus{}function}\PYG{p}{(}\PYG{n}{row}\PYG{p}{)}\PYG{p}{:}
            \PYG{k}{return} \PYG{n}{row}\PYG{p}{[}\PYG{n}{column}\PYG{p}{]} \PYG{o}{==} \PYG{n}{value}
    \PYG{c+c1}{\PYGZsh{} divide the rows into two sets and return them}
    \PYG{n}{set1} \PYG{o}{=} \PYG{p}{[}\PYG{n}{row} \PYG{k}{for} \PYG{n}{row} \PYG{o+ow}{in} \PYG{n}{rows} \PYG{k}{if} \PYG{n}{split\PYGZus{}function}\PYG{p}{(}\PYG{n}{row}\PYG{p}{)}\PYG{p}{]}  \PYG{c+c1}{\PYGZsh{} positive side \PYGZgt{}= or ==}
    \PYG{n}{set2} \PYG{o}{=} \PYG{p}{[}\PYG{n}{row} \PYG{k}{for} \PYG{n}{row} \PYG{o+ow}{in} \PYG{n}{rows} \PYG{k}{if} \PYG{o+ow}{not} \PYG{n}{split\PYGZus{}function}\PYG{p}{(}\PYG{n}{row}\PYG{p}{)}\PYG{p}{]}  \PYG{c+c1}{\PYGZsh{} negative side True or False}
    \PYG{k}{return} \PYG{p}{(}\PYG{n}{set1}\PYG{p}{,} \PYG{n}{set2}\PYG{p}{)}


\PYG{c+c1}{\PYGZsh{} Create counts of possible results (the last column of each row is the result) = how many different results are in a list}
\PYG{k}{def} \PYG{n+nf}{uniquecounts}\PYG{p}{(}\PYG{n}{rows}\PYG{p}{)}\PYG{p}{:}
    \PYG{l+s+sd}{\PYGZdq{}\PYGZdq{}\PYGZdq{}evaluate how many unique elements are in a given list}

\PYG{l+s+sd}{    Arguments:}
\PYG{l+s+sd}{        rows \PYGZob{}list\PYGZcb{} \PYGZhy{}\PYGZhy{} evaluated list}

\PYG{l+s+sd}{    Returns:}
\PYG{l+s+sd}{        integer \PYGZhy{}\PYGZhy{} number of unique elements}
\PYG{l+s+sd}{    \PYGZdq{}\PYGZdq{}\PYGZdq{}}
    \PYG{n}{results} \PYG{o}{=} \PYG{p}{\PYGZob{}}\PYG{p}{\PYGZcb{}}
    \PYG{k}{for} \PYG{n}{row} \PYG{o+ow}{in} \PYG{n}{rows}\PYG{p}{:}
        \PYG{c+c1}{\PYGZsh{} The result is the last column}
        \PYG{n}{r} \PYG{o}{=} \PYG{n}{row}\PYG{p}{[}\PYG{n+nb}{len}\PYG{p}{(}\PYG{n}{row}\PYG{p}{)} \PYG{o}{\PYGZhy{}} \PYG{l+m+mi}{1}\PYG{p}{]}
        \PYG{c+c1}{\PYGZsh{} if r not already in results, entry will be generated}
        \PYG{k}{if} \PYG{n}{r} \PYG{o+ow}{not} \PYG{o+ow}{in} \PYG{n}{results}\PYG{p}{:}
            \PYG{n}{results}\PYG{p}{[}\PYG{n}{r}\PYG{p}{]} \PYG{o}{=} \PYG{l+m+mi}{0}
        \PYG{c+c1}{\PYGZsh{} increase count of r by one}
        \PYG{n}{results}\PYG{p}{[}\PYG{n}{r}\PYG{p}{]} \PYG{o}{+}\PYG{o}{=} \PYG{l+m+mi}{1}
    \PYG{k}{return} \PYG{n}{results}


\PYG{k}{def} \PYG{n+nf}{giniimpurity}\PYG{p}{(}\PYG{n}{rows}\PYG{p}{)}\PYG{p}{:}
    \PYG{l+s+sd}{\PYGZdq{}\PYGZdq{}\PYGZdq{} Probability that a randomly placed item will be in the wrong category}

\PYG{l+s+sd}{    Calculates the probability of each possible outcome by dividing the number of times that outcome occurs}
\PYG{l+s+sd}{    by the total number of rows in the set.}
\PYG{l+s+sd}{    It then adds up the products of all these probabilities.}
\PYG{l+s+sd}{    This gives the overall chance that a row would be randomly assigned to the wrong outcome.}
\PYG{l+s+sd}{    The higher this probability, the worse the split.}

\PYG{l+s+sd}{    Returns:}
\PYG{l+s+sd}{        float \PYGZhy{}\PYGZhy{} probability of being in the wrong category}
\PYG{l+s+sd}{    \PYGZdq{}\PYGZdq{}\PYGZdq{}}
    \PYG{n}{total} \PYG{o}{=} \PYG{n+nb}{len}\PYG{p}{(}\PYG{n}{rows}\PYG{p}{)}
    \PYG{n}{counts} \PYG{o}{=} \PYG{n}{uniquecounts}\PYG{p}{(}\PYG{n}{rows}\PYG{p}{)}
    \PYG{n}{imp} \PYG{o}{=} \PYG{l+m+mi}{0}
    \PYG{k}{for} \PYG{n}{k1} \PYG{o+ow}{in} \PYG{n}{counts}\PYG{p}{:}
        \PYG{n}{p1} \PYG{o}{=} \PYG{n+nb}{float}\PYG{p}{(}\PYG{n}{counts}\PYG{p}{[}\PYG{n}{k1}\PYG{p}{]}\PYG{p}{)} \PYG{o}{/} \PYG{n}{total}
        \PYG{k}{for} \PYG{n}{k2} \PYG{o+ow}{in} \PYG{n}{counts}\PYG{p}{:}
            \PYG{k}{if} \PYG{n}{k1} \PYG{o}{==} \PYG{n}{k2}\PYG{p}{:}
                \PYG{k}{continue}
            \PYG{n}{p2} \PYG{o}{=} \PYG{n+nb}{float}\PYG{p}{(}\PYG{n}{counts}\PYG{p}{[}\PYG{n}{k2}\PYG{p}{]}\PYG{p}{)} \PYG{o}{/} \PYG{n}{total}
            \PYG{n}{imp} \PYG{o}{+}\PYG{o}{=} \PYG{n}{p1} \PYG{o}{*} \PYG{n}{p2}
    \PYG{k}{return} \PYG{n}{imp}


\PYG{k}{def} \PYG{n+nf}{entropy}\PYG{p}{(}\PYG{n}{rows}\PYG{p}{)}\PYG{p}{:}
    \PYG{l+s+sd}{\PYGZdq{}\PYGZdq{}\PYGZdq{}Entropy is the sum of p(x)log(p(x)) across all the different possible results \PYGZhy{}\PYGZhy{}\PYGZgt{} how mixed is a list}

\PYG{l+s+sd}{    Funciton calculates the frequency of each item (the number of times it appears divided by the total number of rows)}
\PYG{l+s+sd}{    and applies these formulas:}

\PYG{l+s+sd}{    .. math::}
\PYG{l+s+sd}{        p(i) = frequency(outcome) = \PYGZbs{}dfrac\PYGZob{}count(outcome)\PYGZcb{}\PYGZob{}count(total rows)\PYGZcb{}}

\PYG{l+s+sd}{        Entropy = \PYGZbs{}sum(p(i)) \PYGZbs{}cdot  \PYGZbs{}log(p(i)) \PYGZbs{} for \PYGZbs{} all \PYGZbs{} outcomes}


\PYG{l+s+sd}{    The higher the entropy, the worse the split.}

\PYG{l+s+sd}{    Arguments:}
\PYG{l+s+sd}{        rows \PYGZob{}list\PYGZcb{} \PYGZhy{}\PYGZhy{} list to evaluate}

\PYG{l+s+sd}{    Returns:}
\PYG{l+s+sd}{        [float] \PYGZhy{}\PYGZhy{} entropy of the list}
\PYG{l+s+sd}{    \PYGZdq{}\PYGZdq{}\PYGZdq{}}
    \PYG{k+kn}{from} \PYG{n+nn}{math} \PYG{k}{import} \PYG{n}{log}

    \PYG{k}{def} \PYG{n+nf}{log2}\PYG{p}{(}\PYG{n}{x}\PYG{p}{)}\PYG{p}{:}
        \PYG{k}{return} \PYG{n}{log}\PYG{p}{(}\PYG{n}{x}\PYG{p}{)} \PYG{o}{/} \PYG{n}{log}\PYG{p}{(}\PYG{l+m+mi}{2}\PYG{p}{)}
    \PYG{n}{results} \PYG{o}{=} \PYG{n}{uniquecounts}\PYG{p}{(}\PYG{n}{rows}\PYG{p}{)}
    \PYG{c+c1}{\PYGZsh{} calculate Entropy}
    \PYG{n}{ent} \PYG{o}{=} \PYG{l+m+mf}{0.0}
    \PYG{k}{for} \PYG{n}{r} \PYG{o+ow}{in} \PYG{n}{results}\PYG{o}{.}\PYG{n}{keys}\PYG{p}{(}\PYG{p}{)}\PYG{p}{:}
        \PYG{n}{p} \PYG{o}{=} \PYG{n+nb}{float}\PYG{p}{(}\PYG{n}{results}\PYG{p}{[}\PYG{n}{r}\PYG{p}{]}\PYG{p}{)} \PYG{o}{/} \PYG{n+nb}{len}\PYG{p}{(}\PYG{n}{rows}\PYG{p}{)}
        \PYG{n}{ent} \PYG{o}{\PYGZhy{}}\PYG{o}{=} \PYG{n}{p} \PYG{o}{*} \PYG{n}{log2}\PYG{p}{(}\PYG{n}{p}\PYG{p}{)}
    \PYG{k}{return} \PYG{n}{ent}


\PYG{k}{def} \PYG{n+nf}{variance}\PYG{p}{(}\PYG{n}{rows}\PYG{p}{)}\PYG{p}{:}
    \PYG{l+s+sd}{\PYGZdq{}\PYGZdq{}\PYGZdq{}Evaluates how close together numerical values lie}

\PYG{l+s+sd}{    Calculates mean and variance for given list}

\PYG{l+s+sd}{    .. math::}
\PYG{l+s+sd}{        mean = \PYGZbs{}dfrac\PYGZob{}\PYGZbs{}sum(entries)\PYGZcb{}\PYGZob{}number \PYGZbs{} of \PYGZbs{} entries\PYGZcb{}}

\PYG{l+s+sd}{        variance = \PYGZbs{}sum(entry \PYGZhy{} mean) \PYGZca{} 2}

\PYG{l+s+sd}{    Arguments:}
\PYG{l+s+sd}{        rows \PYGZob{}list\PYGZcb{} \PYGZhy{}\PYGZhy{} list to evaluate}

\PYG{l+s+sd}{    Returns:}
\PYG{l+s+sd}{        number \PYGZhy{}\PYGZhy{} variance of the list}
\PYG{l+s+sd}{    \PYGZdq{}\PYGZdq{}\PYGZdq{}}
    \PYG{k}{if} \PYG{n+nb}{len}\PYG{p}{(}\PYG{n}{rows}\PYG{p}{)} \PYG{o}{==} \PYG{l+m+mi}{0}\PYG{p}{:}
        \PYG{k}{return} \PYG{l+m+mi}{0}
    \PYG{n}{data} \PYG{o}{=} \PYG{p}{[}\PYG{n+nb}{float}\PYG{p}{(}\PYG{n}{row}\PYG{p}{[}\PYG{n+nb}{len}\PYG{p}{(}\PYG{n}{row}\PYG{p}{)} \PYG{o}{\PYGZhy{}} \PYG{l+m+mi}{1}\PYG{p}{]}\PYG{p}{)} \PYG{k}{for} \PYG{n}{row} \PYG{o+ow}{in} \PYG{n}{rows}\PYG{p}{]}
    \PYG{n}{mean} \PYG{o}{=} \PYG{n+nb}{sum}\PYG{p}{(}\PYG{n}{data}\PYG{p}{)} \PYG{o}{/} \PYG{n+nb}{len}\PYG{p}{(}\PYG{n}{data}\PYG{p}{)}
    \PYG{n}{variance} \PYG{o}{=} \PYG{n+nb}{sum}\PYG{p}{(}\PYG{p}{[}\PYG{p}{(}\PYG{n}{d} \PYG{o}{\PYGZhy{}} \PYG{n}{mean}\PYG{p}{)} \PYG{o}{*}\PYG{o}{*} \PYG{l+m+mi}{2} \PYG{k}{for} \PYG{n}{d} \PYG{o+ow}{in} \PYG{n}{data}\PYG{p}{]}\PYG{p}{)} \PYG{o}{/} \PYG{n+nb}{len}\PYG{p}{(}\PYG{n}{data}\PYG{p}{)}
    \PYG{k}{return} \PYG{n}{variance}


\PYG{c+c1}{\PYGZsh{} building the tree}
\PYG{k}{def} \PYG{n+nf}{buildtree}\PYG{p}{(}\PYG{n}{rows}\PYG{p}{,} \PYG{n}{scoref}\PYG{p}{)}\PYG{p}{:}
    \PYG{l+s+sd}{\PYGZdq{}\PYGZdq{}\PYGZdq{}recursively builds decisionnode objects that form a decision tree}

\PYG{l+s+sd}{    At each node the best possible split is calculated (depending on the evaluation metric).}
\PYG{l+s+sd}{    If no further split is neccessary the remaining items and their number of occurence}
\PYG{l+s+sd}{    are written in the results property.}

\PYG{l+s+sd}{    Arguments:}
\PYG{l+s+sd}{        rows \PYGZob{}list\PYGZcb{} \PYGZhy{}\PYGZhy{} dataset from which to build the tree}
\PYG{l+s+sd}{        scoref \PYGZob{}function\PYGZcb{} \PYGZhy{}\PYGZhy{} evaluation metric (entropy / gini coefficient)}

\PYG{l+s+sd}{    Returns:}
\PYG{l+s+sd}{        decisionnode \PYGZhy{}\PYGZhy{} either two decisionnodes for true and false branch or one decisionnode with results (leaf node)}
\PYG{l+s+sd}{    \PYGZdq{}\PYGZdq{}\PYGZdq{}}
    \PYG{k}{if} \PYG{n+nb}{len}\PYG{p}{(}\PYG{n}{rows}\PYG{p}{)} \PYG{o}{==} \PYG{l+m+mi}{0}\PYG{p}{:}
        \PYG{k}{return} \PYG{n}{decisionnode}\PYG{p}{(}\PYG{p}{)}
    \PYG{n}{current\PYGZus{}score} \PYG{o}{=} \PYG{n}{scoref}\PYG{p}{(}\PYG{n}{rows}\PYG{p}{)}

    \PYG{c+c1}{\PYGZsh{} Set up variables to track the best criteria}
    \PYG{n}{best\PYGZus{}gain} \PYG{o}{=} \PYG{l+m+mf}{0.0}
    \PYG{n}{best\PYGZus{}criteria} \PYG{o}{=} \PYG{k+kc}{None}
    \PYG{n}{best\PYGZus{}sets} \PYG{o}{=} \PYG{k+kc}{None}

    \PYG{n}{column\PYGZus{}count} \PYG{o}{=} \PYG{n+nb}{len}\PYG{p}{(}\PYG{n}{rows}\PYG{p}{[}\PYG{l+m+mi}{0}\PYG{p}{]}\PYG{p}{)} \PYG{o}{\PYGZhy{}} \PYG{l+m+mi}{1}  \PYG{c+c1}{\PYGZsh{} number of columns minus last one (result)}
    \PYG{k}{for} \PYG{n}{col} \PYG{o+ow}{in} \PYG{n+nb}{range}\PYG{p}{(}\PYG{l+m+mi}{0}\PYG{p}{,} \PYG{n}{column\PYGZus{}count}\PYG{p}{)}\PYG{p}{:}
        \PYG{c+c1}{\PYGZsh{} Generate the list of different values in this column}
        \PYG{n}{column\PYGZus{}values} \PYG{o}{=} \PYG{p}{\PYGZob{}}\PYG{p}{\PYGZcb{}}
        \PYG{k}{for} \PYG{n}{row} \PYG{o+ow}{in} \PYG{n}{rows}\PYG{p}{:}
            \PYG{n}{column\PYGZus{}values}\PYG{p}{[}\PYG{n}{row}\PYG{p}{[}\PYG{n}{col}\PYG{p}{]}\PYG{p}{]} \PYG{o}{=} \PYG{l+m+mi}{1}
        \PYG{c+c1}{\PYGZsh{} Try dividing the rows up for each value in this column}
        \PYG{k}{for} \PYG{n}{value} \PYG{o+ow}{in} \PYG{n}{column\PYGZus{}values}\PYG{o}{.}\PYG{n}{keys}\PYG{p}{(}\PYG{p}{)}\PYG{p}{:}
            \PYG{p}{(}\PYG{n}{set1}\PYG{p}{,} \PYG{n}{set2}\PYG{p}{)} \PYG{o}{=} \PYG{n}{divideset}\PYG{p}{(}\PYG{n}{rows}\PYG{p}{,} \PYG{n}{col}\PYG{p}{,} \PYG{n}{value}\PYG{p}{)}

            \PYG{c+c1}{\PYGZsh{} Information Gain}
            \PYG{n}{p} \PYG{o}{=} \PYG{n+nb}{float}\PYG{p}{(}\PYG{n+nb}{len}\PYG{p}{(}\PYG{n}{set1}\PYG{p}{)}\PYG{p}{)} \PYG{o}{/} \PYG{n+nb}{len}\PYG{p}{(}\PYG{n}{rows}\PYG{p}{)}  \PYG{c+c1}{\PYGZsh{} = ration(Anteil) of list 1 against whole list (list1+list2)}
            \PYG{n}{gain} \PYG{o}{=} \PYG{n}{current\PYGZus{}score} \PYG{o}{\PYGZhy{}} \PYG{n}{p} \PYG{o}{*} \PYG{n}{scoref}\PYG{p}{(}\PYG{n}{set1}\PYG{p}{)} \PYG{o}{\PYGZhy{}} \PYG{p}{(}\PYG{l+m+mi}{1} \PYG{o}{\PYGZhy{}} \PYG{n}{p}\PYG{p}{)} \PYG{o}{*} \PYG{n}{scoref}\PYG{p}{(}\PYG{n}{set2}\PYG{p}{)}  \PYG{c+c1}{\PYGZsh{} set1 and set2 can be exchanged}
            \PYG{k}{if} \PYG{n}{gain} \PYG{o}{\PYGZgt{}} \PYG{n}{best\PYGZus{}gain} \PYG{o+ow}{and} \PYG{n+nb}{len}\PYG{p}{(}\PYG{n}{set1}\PYG{p}{)} \PYG{o}{\PYGZgt{}} \PYG{l+m+mi}{0} \PYG{o+ow}{and} \PYG{n+nb}{len}\PYG{p}{(}\PYG{n}{set2}\PYG{p}{)} \PYG{o}{\PYGZgt{}} \PYG{l+m+mi}{0}\PYG{p}{:}
                \PYG{n}{best\PYGZus{}gain} \PYG{o}{=} \PYG{n}{gain}
                \PYG{n}{best\PYGZus{}criteria} \PYG{o}{=} \PYG{p}{(}\PYG{n}{col}\PYG{p}{,} \PYG{n}{value}\PYG{p}{)}
                \PYG{n}{best\PYGZus{}sets} \PYG{o}{=} \PYG{p}{(}\PYG{n}{set1}\PYG{p}{,} \PYG{n}{set2}\PYG{p}{)}
    \PYG{c+c1}{\PYGZsh{} print \PYGZdq{}Best Gain = \PYGZdq{} + str(best\PYGZus{}gain)}
    \PYG{c+c1}{\PYGZsh{} print \PYGZdq{}Best criteria = \PYGZdq{} + str(best\PYGZus{}criteria)}

    \PYG{c+c1}{\PYGZsh{} Create subbranches}
    \PYG{k}{if} \PYG{n}{best\PYGZus{}gain} \PYG{o}{\PYGZgt{}} \PYG{l+m+mi}{0}\PYG{p}{:}
        \PYG{n}{trueBranch} \PYG{o}{=} \PYG{n}{buildtree}\PYG{p}{(}\PYG{n}{best\PYGZus{}sets}\PYG{p}{[}\PYG{l+m+mi}{0}\PYG{p}{]}\PYG{p}{,} \PYG{n}{scoref}\PYG{p}{)}
        \PYG{n}{falseBranch} \PYG{o}{=} \PYG{n}{buildtree}\PYG{p}{(}\PYG{n}{best\PYGZus{}sets}\PYG{p}{[}\PYG{l+m+mi}{1}\PYG{p}{]}\PYG{p}{,} \PYG{n}{scoref}\PYG{p}{)}
        \PYG{k}{return} \PYG{n}{decisionnode}\PYG{p}{(}\PYG{n}{col}\PYG{o}{=}\PYG{n}{best\PYGZus{}criteria}\PYG{p}{[}\PYG{l+m+mi}{0}\PYG{p}{]}\PYG{p}{,} \PYG{n}{value}\PYG{o}{=}\PYG{n}{best\PYGZus{}criteria}\PYG{p}{[}\PYG{l+m+mi}{1}\PYG{p}{]}\PYG{p}{,} \PYG{n}{tb}\PYG{o}{=}\PYG{n}{trueBranch}\PYG{p}{,} \PYG{n}{fb}\PYG{o}{=}\PYG{n}{falseBranch}\PYG{p}{)}
    \PYG{k}{else}\PYG{p}{:}
        \PYG{k}{return} \PYG{n}{decisionnode}\PYG{p}{(}\PYG{n}{results}\PYG{o}{=}\PYG{n}{uniquecounts}\PYG{p}{(}\PYG{n}{rows}\PYG{p}{)}\PYG{p}{)}


\PYG{k}{def} \PYG{n+nf}{printtree}\PYG{p}{(}\PYG{n}{tree}\PYG{p}{,} \PYG{n}{indent}\PYG{o}{=}\PYG{l+s+s1}{\PYGZsq{}}\PYG{l+s+s1}{ }\PYG{l+s+s1}{\PYGZsq{}}\PYG{p}{)}\PYG{p}{:}
    \PYG{l+s+sd}{\PYGZdq{}\PYGZdq{}\PYGZdq{}prints out the tree on the command line}

\PYG{l+s+sd}{    Arguments:}
\PYG{l+s+sd}{        tree \PYGZob{}decisionnode\PYGZcb{} \PYGZhy{}\PYGZhy{} tree that gets printed}

\PYG{l+s+sd}{    \PYGZdq{}\PYGZdq{}\PYGZdq{}}
    \PYG{k}{if} \PYG{n}{tree}\PYG{o}{.}\PYG{n}{results} \PYG{o+ow}{is} \PYG{o+ow}{not} \PYG{k+kc}{None}\PYG{p}{:}
        \PYG{n+nb}{print} \PYG{n+nb}{str}\PYG{p}{(}\PYG{n}{tree}\PYG{o}{.}\PYG{n}{results}\PYG{p}{)}
    \PYG{k}{else}\PYG{p}{:}
        \PYG{n+nb}{print} \PYG{n+nb}{str}\PYG{p}{(}\PYG{n}{tree}\PYG{o}{.}\PYG{n}{col}\PYG{p}{)} \PYG{o}{+} \PYG{l+s+s1}{\PYGZsq{}}\PYG{l+s+s1}{: }\PYG{l+s+s1}{\PYGZsq{}} \PYG{o}{+} \PYG{n+nb}{str}\PYG{p}{(}\PYG{n}{tree}\PYG{o}{.}\PYG{n}{value}\PYG{p}{)} \PYG{o}{+} \PYG{l+s+s1}{\PYGZsq{}}\PYG{l+s+s1}{?}\PYG{l+s+s1}{\PYGZsq{}}
        \PYG{n+nb}{print} \PYG{n}{indent} \PYG{o}{+} \PYG{l+s+s1}{\PYGZsq{}}\PYG{l+s+s1}{T\PYGZhy{}\PYGZhy{}\PYGZgt{}}\PYG{l+s+s1}{\PYGZsq{}}\PYG{p}{,}
        \PYG{n}{printtree}\PYG{p}{(}\PYG{n}{tree}\PYG{o}{.}\PYG{n}{tb}\PYG{p}{,} \PYG{n}{indent} \PYG{o}{+} \PYG{l+s+s1}{\PYGZsq{}}\PYG{l+s+s1}{   }\PYG{l+s+s1}{\PYGZsq{}}\PYG{p}{)}
        \PYG{n+nb}{print} \PYG{n}{indent} \PYG{o}{+} \PYG{l+s+s1}{\PYGZsq{}}\PYG{l+s+s1}{F\PYGZhy{}\PYGZhy{}\PYGZgt{}}\PYG{l+s+s1}{\PYGZsq{}}\PYG{p}{,}
        \PYG{n}{printtree}\PYG{p}{(}\PYG{n}{tree}\PYG{o}{.}\PYG{n}{fb}\PYG{p}{,} \PYG{n}{indent} \PYG{o}{+} \PYG{l+s+s1}{\PYGZsq{}}\PYG{l+s+s1}{   }\PYG{l+s+s1}{\PYGZsq{}}\PYG{p}{)}


\PYG{k}{def} \PYG{n+nf}{getwidth}\PYG{p}{(}\PYG{n}{tree}\PYG{p}{)}\PYG{p}{:}
    \PYG{l+s+sd}{\PYGZdq{}\PYGZdq{}\PYGZdq{}returns the number of leaves = endnodes in the tree}

\PYG{l+s+sd}{    Arguments:}
\PYG{l+s+sd}{        tree \PYGZob{}decisionnode\PYGZcb{} \PYGZhy{}\PYGZhy{} tree to examine}

\PYG{l+s+sd}{    Returns:}
\PYG{l+s+sd}{        number \PYGZhy{}\PYGZhy{} number of endnodes}
\PYG{l+s+sd}{    \PYGZdq{}\PYGZdq{}\PYGZdq{}}
    \PYG{k}{if} \PYG{n}{tree}\PYG{o}{.}\PYG{n}{tb} \PYG{o+ow}{is} \PYG{k+kc}{None} \PYG{o+ow}{and} \PYG{n}{tree}\PYG{o}{.}\PYG{n}{fb} \PYG{o+ow}{is} \PYG{k+kc}{None}\PYG{p}{:}
        \PYG{k}{return} \PYG{l+m+mi}{1}
    \PYG{k}{return} \PYG{n}{getwidth}\PYG{p}{(}\PYG{n}{tree}\PYG{o}{.}\PYG{n}{tb}\PYG{p}{)} \PYG{o}{+} \PYG{n}{getwidth}\PYG{p}{(}\PYG{n}{tree}\PYG{o}{.}\PYG{n}{fb}\PYG{p}{)}


\PYG{k}{def} \PYG{n+nf}{getdepth}\PYG{p}{(}\PYG{n}{tree}\PYG{p}{)}\PYG{p}{:}
    \PYG{l+s+sd}{\PYGZdq{}\PYGZdq{}\PYGZdq{}returns the maximum number of consecutive nodes}

\PYG{l+s+sd}{    Arguments:}
\PYG{l+s+sd}{        tree \PYGZob{}decisionnode\PYGZcb{} \PYGZhy{}\PYGZhy{} tree to examine}

\PYG{l+s+sd}{    Returns:}
\PYG{l+s+sd}{        number \PYGZhy{}\PYGZhy{} maximum number of consecutive nodes}
\PYG{l+s+sd}{    \PYGZdq{}\PYGZdq{}\PYGZdq{}}
    \PYG{k}{if} \PYG{n}{tree}\PYG{o}{.}\PYG{n}{tb} \PYG{o+ow}{is} \PYG{k+kc}{None} \PYG{o+ow}{and} \PYG{n}{tree}\PYG{o}{.}\PYG{n}{fb} \PYG{o+ow}{is} \PYG{k+kc}{None}\PYG{p}{:}
        \PYG{k}{return} \PYG{l+m+mi}{0}
    \PYG{k}{return} \PYG{n+nb}{max}\PYG{p}{(}\PYG{n}{getdepth}\PYG{p}{(}\PYG{n}{tree}\PYG{o}{.}\PYG{n}{tb}\PYG{p}{)}\PYG{p}{,} \PYG{n}{getdepth}\PYG{p}{(}\PYG{n}{tree}\PYG{o}{.}\PYG{n}{fb}\PYG{p}{)}\PYG{p}{)} \PYG{o}{+} \PYG{l+m+mi}{1}


\PYG{k}{def} \PYG{n+nf}{drawtree}\PYG{p}{(}\PYG{n}{tree}\PYG{p}{,} \PYG{n}{jpeg}\PYG{o}{=}\PYG{l+s+s1}{\PYGZsq{}}\PYG{l+s+s1}{tree.jpg}\PYG{l+s+s1}{\PYGZsq{}}\PYG{p}{)}\PYG{p}{:}
    \PYG{l+s+sd}{\PYGZdq{}\PYGZdq{}\PYGZdq{}visualization of the tree in a jpeg}

\PYG{l+s+sd}{    Arguments:}
\PYG{l+s+sd}{        tree \PYGZob{}decisionnode\PYGZcb{} \PYGZhy{}\PYGZhy{} tree to draw}

\PYG{l+s+sd}{    Keyword Arguments:}
\PYG{l+s+sd}{        jpeg \PYGZob{}str\PYGZcb{} \PYGZhy{}\PYGZhy{} Name of the .jpg (default: \PYGZob{}\PYGZsq{}tree.jpg\PYGZsq{}\PYGZcb{})}
\PYG{l+s+sd}{    \PYGZdq{}\PYGZdq{}\PYGZdq{}}
    \PYG{n}{w} \PYG{o}{=} \PYG{n}{getwidth}\PYG{p}{(}\PYG{n}{tree}\PYG{p}{)} \PYG{o}{*} \PYG{l+m+mi}{100}
    \PYG{n}{h} \PYG{o}{=} \PYG{n}{getdepth}\PYG{p}{(}\PYG{n}{tree}\PYG{p}{)} \PYG{o}{*} \PYG{l+m+mi}{100} \PYG{o}{+} \PYG{l+m+mi}{120}

    \PYG{n}{img} \PYG{o}{=} \PYG{n}{Image}\PYG{o}{.}\PYG{n}{new}\PYG{p}{(}\PYG{l+s+s1}{\PYGZsq{}}\PYG{l+s+s1}{RGB}\PYG{l+s+s1}{\PYGZsq{}}\PYG{p}{,} \PYG{p}{(}\PYG{n}{w}\PYG{p}{,} \PYG{n}{h}\PYG{p}{)}\PYG{p}{,} \PYG{p}{(}\PYG{l+m+mi}{255}\PYG{p}{,} \PYG{l+m+mi}{255}\PYG{p}{,} \PYG{l+m+mi}{255}\PYG{p}{)}\PYG{p}{)}
    \PYG{n}{draw} \PYG{o}{=} \PYG{n}{ImageDraw}\PYG{o}{.}\PYG{n}{Draw}\PYG{p}{(}\PYG{n}{img}\PYG{p}{)}

    \PYG{n}{drawnode}\PYG{p}{(}\PYG{n}{draw}\PYG{p}{,} \PYG{n}{tree}\PYG{p}{,} \PYG{n}{w} \PYG{o}{/} \PYG{l+m+mi}{2}\PYG{p}{,} \PYG{l+m+mi}{20}\PYG{p}{)}
    \PYG{n}{img}\PYG{o}{.}\PYG{n}{save}\PYG{p}{(}\PYG{n}{jpeg}\PYG{p}{,} \PYG{l+s+s1}{\PYGZsq{}}\PYG{l+s+s1}{JPEG}\PYG{l+s+s1}{\PYGZsq{}}\PYG{p}{)}


\PYG{k}{def} \PYG{n+nf}{drawnode}\PYG{p}{(}\PYG{n}{draw}\PYG{p}{,} \PYG{n}{tree}\PYG{p}{,} \PYG{n}{x}\PYG{p}{,} \PYG{n}{y}\PYG{p}{)}\PYG{p}{:}
    \PYG{l+s+sd}{\PYGZdq{}\PYGZdq{}\PYGZdq{}Helper Function for drawtree, draws a single node}

\PYG{l+s+sd}{    Arguments:}
\PYG{l+s+sd}{        draw \PYGZob{}img\PYGZcb{} \PYGZhy{}\PYGZhy{} node to be drawn}
\PYG{l+s+sd}{        tree \PYGZob{}decisionnode\PYGZcb{} \PYGZhy{}\PYGZhy{} tree that the node belongs to}
\PYG{l+s+sd}{        x \PYGZob{}number\PYGZcb{} \PYGZhy{}\PYGZhy{} x location}
\PYG{l+s+sd}{        y \PYGZob{}number\PYGZcb{} \PYGZhy{}\PYGZhy{} y location}
\PYG{l+s+sd}{    \PYGZdq{}\PYGZdq{}\PYGZdq{}}
    \PYG{k}{if} \PYG{n}{tree}\PYG{o}{.}\PYG{n}{results} \PYG{o+ow}{is} \PYG{k+kc}{None}\PYG{p}{:}
        \PYG{c+c1}{\PYGZsh{} Get the width of each branch}
        \PYG{n}{w1} \PYG{o}{=} \PYG{n}{getwidth}\PYG{p}{(}\PYG{n}{tree}\PYG{o}{.}\PYG{n}{fb}\PYG{p}{)} \PYG{o}{*} \PYG{l+m+mi}{100}
        \PYG{n}{w2} \PYG{o}{=} \PYG{n}{getwidth}\PYG{p}{(}\PYG{n}{tree}\PYG{o}{.}\PYG{n}{tb}\PYG{p}{)} \PYG{o}{*} \PYG{l+m+mi}{100}

        \PYG{c+c1}{\PYGZsh{} Determine the total space required by this node}
        \PYG{n}{left} \PYG{o}{=} \PYG{n}{x} \PYG{o}{\PYGZhy{}} \PYG{p}{(}\PYG{n}{w1} \PYG{o}{+} \PYG{n}{w2}\PYG{p}{)} \PYG{o}{/} \PYG{l+m+mi}{2}
        \PYG{n}{right} \PYG{o}{=} \PYG{n}{x} \PYG{o}{+} \PYG{p}{(}\PYG{n}{w1} \PYG{o}{+} \PYG{n}{w2}\PYG{p}{)} \PYG{o}{/} \PYG{l+m+mi}{2}

        \PYG{c+c1}{\PYGZsh{} Draw the condition string}
        \PYG{n}{draw}\PYG{o}{.}\PYG{n}{text}\PYG{p}{(}\PYG{p}{(}\PYG{n}{x} \PYG{o}{\PYGZhy{}} \PYG{l+m+mi}{20}\PYG{p}{,} \PYG{n}{y} \PYG{o}{\PYGZhy{}} \PYG{l+m+mi}{10}\PYG{p}{)}\PYG{p}{,} \PYG{n+nb}{str}\PYG{p}{(}\PYG{n}{tree}\PYG{o}{.}\PYG{n}{col}\PYG{p}{)} \PYG{o}{+} \PYG{l+s+s1}{\PYGZsq{}}\PYG{l+s+s1}{:}\PYG{l+s+s1}{\PYGZsq{}} \PYG{o}{+} \PYG{n+nb}{str}\PYG{p}{(}\PYG{n}{tree}\PYG{o}{.}\PYG{n}{value}\PYG{p}{)}\PYG{p}{,} \PYG{p}{(}\PYG{l+m+mi}{0}\PYG{p}{,} \PYG{l+m+mi}{0}\PYG{p}{,} \PYG{l+m+mi}{0}\PYG{p}{)}\PYG{p}{)}

        \PYG{c+c1}{\PYGZsh{} Draw links to the branches}
        \PYG{n}{draw}\PYG{o}{.}\PYG{n}{line}\PYG{p}{(}\PYG{p}{(}\PYG{n}{x}\PYG{p}{,} \PYG{n}{y}\PYG{p}{,} \PYG{n}{left} \PYG{o}{+} \PYG{n}{w1} \PYG{o}{/} \PYG{l+m+mi}{2}\PYG{p}{,} \PYG{n}{y} \PYG{o}{+} \PYG{l+m+mi}{100}\PYG{p}{)}\PYG{p}{,} \PYG{n}{fill}\PYG{o}{=}\PYG{p}{(}\PYG{l+m+mi}{255}\PYG{p}{,} \PYG{l+m+mi}{0}\PYG{p}{,} \PYG{l+m+mi}{0}\PYG{p}{)}\PYG{p}{)}
        \PYG{n}{draw}\PYG{o}{.} \PYG{n}{line}\PYG{p}{(}\PYG{p}{(}\PYG{n}{x}\PYG{p}{,} \PYG{n}{y}\PYG{p}{,} \PYG{n}{right} \PYG{o}{\PYGZhy{}} \PYG{n}{w2} \PYG{o}{/} \PYG{l+m+mi}{2}\PYG{p}{,} \PYG{n}{y} \PYG{o}{+} \PYG{l+m+mi}{100}\PYG{p}{)}\PYG{p}{,} \PYG{n}{fill}\PYG{o}{=}\PYG{p}{(}\PYG{l+m+mi}{255}\PYG{p}{,} \PYG{l+m+mi}{0}\PYG{p}{,} \PYG{l+m+mi}{0}\PYG{p}{)}\PYG{p}{)}

        \PYG{c+c1}{\PYGZsh{} Draw the branch nodes}
        \PYG{n}{drawnode}\PYG{p}{(}\PYG{n}{draw}\PYG{p}{,} \PYG{n}{tree}\PYG{o}{.}\PYG{n}{fb}\PYG{p}{,} \PYG{n}{left} \PYG{o}{+} \PYG{n}{w1} \PYG{o}{/} \PYG{l+m+mi}{2}\PYG{p}{,} \PYG{n}{y} \PYG{o}{+} \PYG{l+m+mi}{100}\PYG{p}{)}
        \PYG{n}{drawnode}\PYG{p}{(}\PYG{n}{draw}\PYG{p}{,} \PYG{n}{tree}\PYG{o}{.}\PYG{n}{tb}\PYG{p}{,} \PYG{n}{right} \PYG{o}{\PYGZhy{}} \PYG{n}{w2} \PYG{o}{/} \PYG{l+m+mi}{2}\PYG{p}{,} \PYG{n}{y} \PYG{o}{+} \PYG{l+m+mi}{100}\PYG{p}{)}
    \PYG{k}{else}\PYG{p}{:}
        \PYG{n}{txt} \PYG{o}{=} \PYG{l+s+s1}{\PYGZsq{}}\PYG{l+s+s1}{ }\PYG{l+s+se}{\PYGZbs{}n}\PYG{l+s+s1}{\PYGZsq{}}\PYG{o}{.}\PYG{n}{join}\PYG{p}{(}\PYG{p}{[}\PYG{l+s+s1}{\PYGZsq{}}\PYG{l+s+si}{\PYGZpc{}s}\PYG{l+s+s1}{:}\PYG{l+s+si}{\PYGZpc{}d}\PYG{l+s+s1}{\PYGZsq{}} \PYG{o}{\PYGZpc{}} \PYG{n}{v} \PYG{k}{for} \PYG{n}{v} \PYG{o+ow}{in} \PYG{n}{tree}\PYG{o}{.}\PYG{n}{results}\PYG{o}{.}\PYG{n}{items}\PYG{p}{(}\PYG{p}{)}\PYG{p}{]}\PYG{p}{)}
        \PYG{n}{draw}\PYG{o}{.}\PYG{n}{text}\PYG{p}{(}\PYG{p}{(}\PYG{n}{x} \PYG{o}{\PYGZhy{}} \PYG{l+m+mi}{20}\PYG{p}{,} \PYG{n}{y}\PYG{p}{)}\PYG{p}{,} \PYG{n}{txt}\PYG{p}{,} \PYG{p}{(}\PYG{l+m+mi}{0}\PYG{p}{,} \PYG{l+m+mi}{0}\PYG{p}{,} \PYG{l+m+mi}{0}\PYG{p}{)}\PYG{p}{)}


\PYG{k}{def} \PYG{n+nf}{prune}\PYG{p}{(}\PYG{n}{tree}\PYG{p}{,} \PYG{n}{mingain}\PYG{p}{)}\PYG{p}{:}
    \PYG{l+s+sd}{\PYGZdq{}\PYGZdq{}\PYGZdq{}prunes the leaves of a tree in order to reduce complexity}

\PYG{l+s+sd}{    By looking at the information gain that is achieved by splitting data further and further and checking if}
\PYG{l+s+sd}{    it is above the mingain threshold, neighbouring leaves can be collapsed to a single leaf.}

\PYG{l+s+sd}{    Arguments:}
\PYG{l+s+sd}{        tree \PYGZob{}decisionnode\PYGZcb{} \PYGZhy{}\PYGZhy{} tree that gets pruned}
\PYG{l+s+sd}{        mingain \PYGZob{}number\PYGZcb{} \PYGZhy{}\PYGZhy{} threshold for pruning}
\PYG{l+s+sd}{    \PYGZdq{}\PYGZdq{}\PYGZdq{}}
    \PYG{k}{if} \PYG{n}{getdepth}\PYG{p}{(}\PYG{n}{tree}\PYG{p}{)} \PYG{o}{==} \PYG{l+m+mi}{0}\PYG{p}{:}
        \PYG{k}{return}
    \PYG{c+c1}{\PYGZsh{} If the branches aren\PYGZsq{}t leaves, then prune them}
    \PYG{k}{if} \PYG{n}{tree}\PYG{o}{.}\PYG{n}{tb}\PYG{o}{.}\PYG{n}{results} \PYG{o+ow}{is} \PYG{k+kc}{None}\PYG{p}{:}
        \PYG{n}{prune}\PYG{p}{(}\PYG{n}{tree}\PYG{o}{.}\PYG{n}{tb}\PYG{p}{,} \PYG{n}{mingain}\PYG{p}{)}
    \PYG{k}{if} \PYG{n}{tree}\PYG{o}{.}\PYG{n}{fb}\PYG{o}{.}\PYG{n}{results} \PYG{o+ow}{is} \PYG{k+kc}{None}\PYG{p}{:}
        \PYG{n}{prune}\PYG{p}{(}\PYG{n}{tree}\PYG{o}{.}\PYG{n}{fb}\PYG{p}{,} \PYG{n}{mingain}\PYG{p}{)}

    \PYG{c+c1}{\PYGZsh{} If both the subbranches are now leaves, see if they should be merged}
    \PYG{k}{if} \PYG{n}{tree}\PYG{o}{.}\PYG{n}{tb}\PYG{o}{.}\PYG{n}{results} \PYG{o+ow}{is} \PYG{o+ow}{not} \PYG{k+kc}{None} \PYG{o+ow}{and} \PYG{n}{tree}\PYG{o}{.}\PYG{n}{fb}\PYG{o}{.}\PYG{n}{results} \PYG{o+ow}{is} \PYG{o+ow}{not} \PYG{k+kc}{None}\PYG{p}{:}
        \PYG{c+c1}{\PYGZsh{} Build a combined dataset}
        \PYG{n}{tb}\PYG{p}{,} \PYG{n}{fb} \PYG{o}{=} \PYG{p}{[}\PYG{p}{]}\PYG{p}{,} \PYG{p}{[}\PYG{p}{]}
        \PYG{c+c1}{\PYGZsh{} v equals key, c equals value, results in a list of the different values each added up}
        \PYG{k}{for} \PYG{n}{v}\PYG{p}{,} \PYG{n}{c} \PYG{o+ow}{in} \PYG{n}{tree}\PYG{o}{.}\PYG{n}{tb}\PYG{o}{.}\PYG{n}{results}\PYG{o}{.}\PYG{n}{items}\PYG{p}{(}\PYG{p}{)}\PYG{p}{:}
            \PYG{n}{tb} \PYG{o}{+}\PYG{o}{=} \PYG{p}{[}\PYG{p}{[}\PYG{n}{v}\PYG{p}{]}\PYG{p}{]} \PYG{o}{*} \PYG{n}{c}
        \PYG{k}{for} \PYG{n}{v}\PYG{p}{,} \PYG{n}{c} \PYG{o+ow}{in} \PYG{n}{tree}\PYG{o}{.}\PYG{n}{fb}\PYG{o}{.}\PYG{n}{results}\PYG{o}{.}\PYG{n}{items}\PYG{p}{(}\PYG{p}{)}\PYG{p}{:}
            \PYG{n}{fb} \PYG{o}{+}\PYG{o}{=} \PYG{p}{[}\PYG{p}{[}\PYG{n}{v}\PYG{p}{]}\PYG{p}{]} \PYG{o}{*} \PYG{n}{c}

        \PYG{c+c1}{\PYGZsh{} Test the reduction in entropy}
        \PYG{n}{delta} \PYG{o}{=} \PYG{n}{entropy}\PYG{p}{(}\PYG{n}{tb} \PYG{o}{+} \PYG{n}{fb}\PYG{p}{)} \PYG{o}{\PYGZhy{}} \PYG{p}{(}\PYG{n}{entropy}\PYG{p}{(}\PYG{n}{tb}\PYG{p}{)} \PYG{o}{+} \PYG{n}{entropy}\PYG{p}{(}\PYG{n}{fb}\PYG{p}{)}\PYG{p}{)} \PYG{o}{/} \PYG{l+m+mi}{2}
        \PYG{c+c1}{\PYGZsh{} print delta}
        \PYG{k}{if} \PYG{n}{delta} \PYG{o}{\PYGZlt{}} \PYG{n}{mingain}\PYG{p}{:}
            \PYG{c+c1}{\PYGZsh{} Merge the branches}
            \PYG{n}{tree}\PYG{o}{.}\PYG{n}{tb}\PYG{p}{,} \PYG{n}{tree}\PYG{o}{.}\PYG{n}{fb} \PYG{o}{=} \PYG{k+kc}{None}\PYG{p}{,} \PYG{k+kc}{None}
            \PYG{n}{tree}\PYG{o}{.}\PYG{n}{results} \PYG{o}{=} \PYG{n}{uniquecounts}\PYG{p}{(}\PYG{n}{tb} \PYG{o}{+} \PYG{n}{fb}\PYG{p}{)}
            \PYG{c+c1}{\PYGZsh{} print \PYGZdq{}tree pruned\PYGZdq{}}


\PYG{k}{def} \PYG{n+nf}{classify}\PYG{p}{(}\PYG{n}{observation}\PYG{p}{,} \PYG{n}{tree}\PYG{p}{)}\PYG{p}{:}
    \PYG{l+s+sd}{\PYGZdq{}\PYGZdq{}\PYGZdq{}takes a new data set that gets classified and the tree that determines the classification and returns the estimated result.}

\PYG{l+s+sd}{    Arguments:}
\PYG{l+s+sd}{        observation \PYGZob{}numpy.array\PYGZcb{} \PYGZhy{}\PYGZhy{} the new data set that gets classified, e.g. test data set}
\PYG{l+s+sd}{        tree \PYGZob{}decisionnode\PYGZcb{} \PYGZhy{}\PYGZhy{} tree that observation gets classified in}

\PYG{l+s+sd}{    Returns:}
\PYG{l+s+sd}{        data \PYGZhy{}\PYGZhy{} expected result}
\PYG{l+s+sd}{    \PYGZdq{}\PYGZdq{}\PYGZdq{}}
    \PYG{k}{if} \PYG{n}{tree}\PYG{o}{.}\PYG{n}{results} \PYG{o+ow}{is} \PYG{o+ow}{not} \PYG{k+kc}{None}\PYG{p}{:}
        \PYG{k}{return} \PYG{n}{tree}\PYG{o}{.}\PYG{n}{results}
    \PYG{k}{else}\PYG{p}{:}
        \PYG{n}{v} \PYG{o}{=} \PYG{n}{observation}\PYG{p}{[}\PYG{n}{tree}\PYG{o}{.}\PYG{n}{col}\PYG{p}{]}
        \PYG{k}{if} \PYG{n}{v} \PYG{o+ow}{is} \PYG{k+kc}{None}\PYG{p}{:}
            \PYG{n}{tr}\PYG{p}{,} \PYG{n}{fr} \PYG{o}{=} \PYG{n}{classify}\PYG{p}{(}\PYG{n}{observation}\PYG{p}{,} \PYG{n}{tree}\PYG{o}{.}\PYG{n}{tb}\PYG{p}{)}\PYG{p}{,} \PYG{n}{classify}\PYG{p}{(}\PYG{n}{observation}\PYG{p}{,} \PYG{n}{tree}\PYG{o}{.}\PYG{n}{fb}\PYG{p}{)}
            \PYG{n}{tcount} \PYG{o}{=} \PYG{n+nb}{sum}\PYG{p}{(}\PYG{n}{tr}\PYG{o}{.}\PYG{n}{values}\PYG{p}{(}\PYG{p}{)}\PYG{p}{)}
            \PYG{n}{fcount} \PYG{o}{=} \PYG{n+nb}{sum}\PYG{p}{(}\PYG{n}{fr}\PYG{o}{.}\PYG{n}{values}\PYG{p}{(}\PYG{p}{)}\PYG{p}{)}
            \PYG{n}{tw} \PYG{o}{=} \PYG{n+nb}{float}\PYG{p}{(}\PYG{n}{tcount}\PYG{p}{)} \PYG{o}{/} \PYG{p}{(}\PYG{n}{tcount} \PYG{o}{+} \PYG{n}{fcount}\PYG{p}{)}
            \PYG{n}{fw} \PYG{o}{=} \PYG{l+m+mi}{1} \PYG{o}{\PYGZhy{}} \PYG{n}{tw}
            \PYG{n}{result} \PYG{o}{=} \PYG{p}{\PYGZob{}}\PYG{p}{\PYGZcb{}}
            \PYG{k}{for} \PYG{n}{k}\PYG{p}{,} \PYG{n}{v} \PYG{o+ow}{in} \PYG{n}{tr}\PYG{o}{.}\PYG{n}{items}\PYG{p}{(}\PYG{p}{)}\PYG{p}{:}  \PYG{c+c1}{\PYGZsh{} k is name, v is value}
                \PYG{n}{result}\PYG{p}{[}\PYG{n}{k}\PYG{p}{]} \PYG{o}{=} \PYG{n}{v} \PYG{o}{*} \PYG{n}{tw}
            \PYG{k}{for} \PYG{n}{k}\PYG{p}{,} \PYG{n}{v} \PYG{o+ow}{in} \PYG{n}{fr}\PYG{o}{.}\PYG{n}{items}\PYG{p}{(}\PYG{p}{)}\PYG{p}{:}
                \PYG{n}{result}\PYG{p}{[}\PYG{n}{k}\PYG{p}{]} \PYG{o}{=} \PYG{n}{result}\PYG{o}{.}\PYG{n}{setdefault}\PYG{p}{(}\PYG{n}{k}\PYG{p}{,} \PYG{l+m+mi}{0}\PYG{p}{)} \PYG{o}{+} \PYG{p}{(}\PYG{n}{v} \PYG{o}{*} \PYG{n}{fw}\PYG{p}{)}
            \PYG{k}{return} \PYG{n}{result}
        \PYG{k}{else}\PYG{p}{:}
            \PYG{k}{if} \PYG{n+nb}{isinstance}\PYG{p}{(}\PYG{n}{v}\PYG{p}{,} \PYG{n+nb}{int}\PYG{p}{)} \PYG{o+ow}{or} \PYG{n+nb}{isinstance}\PYG{p}{(}\PYG{n}{v}\PYG{p}{,} \PYG{n+nb}{float}\PYG{p}{)}\PYG{p}{:}
                \PYG{k}{if} \PYG{n}{v} \PYG{o}{\PYGZgt{}}\PYG{o}{=} \PYG{n}{tree}\PYG{o}{.}\PYG{n}{value}\PYG{p}{:}
                    \PYG{n}{branch} \PYG{o}{=} \PYG{n}{tree}\PYG{o}{.}\PYG{n}{tb}
                \PYG{k}{else}\PYG{p}{:}
                    \PYG{n}{branch} \PYG{o}{=} \PYG{n}{tree}\PYG{o}{.}\PYG{n}{fb}
            \PYG{k}{else}\PYG{p}{:}
                \PYG{k}{if} \PYG{n}{v} \PYG{o}{==} \PYG{n}{tree}\PYG{o}{.}\PYG{n}{value}\PYG{p}{:}
                    \PYG{n}{branch} \PYG{o}{=} \PYG{n}{tree}\PYG{o}{.}\PYG{n}{tb}
                \PYG{k}{else}\PYG{p}{:}
                    \PYG{n}{branch} \PYG{o}{=} \PYG{n}{tree}\PYG{o}{.}\PYG{n}{fb}
        \PYG{k}{return} \PYG{n}{classify}\PYG{p}{(}\PYG{n}{observation}\PYG{p}{,} \PYG{n}{branch}\PYG{p}{)}


\PYG{k}{def} \PYG{n+nf}{path\PYGZus{}gen}\PYG{p}{(}\PYG{n}{tree}\PYG{p}{)}\PYG{p}{:}
    \PYG{l+s+sd}{\PYGZdq{}\PYGZdq{}\PYGZdq{}Create a path Matrix which contains the structure of the tree. Calls path\PYGZus{}gen2 to do so.}

\PYG{l+s+sd}{    Arguments:}
\PYG{l+s+sd}{        tree \PYGZob{}decisionnode\PYGZcb{} \PYGZhy{}\PYGZhy{} tree of which the data structure is stored}

\PYG{l+s+sd}{    Returns:}
\PYG{l+s+sd}{        numpy.array \PYGZhy{}\PYGZhy{} data structure of the tree, NaN means there is no more branch}
\PYG{l+s+sd}{    \PYGZdq{}\PYGZdq{}\PYGZdq{}}
    \PYG{n}{z1} \PYG{o}{=} \PYG{l+m+mi}{0}  \PYG{c+c1}{\PYGZsh{} equals number of leafs, increases during creation of path}
    \PYG{n}{z2} \PYG{o}{=} \PYG{l+m+mi}{0}  \PYG{c+c1}{\PYGZsh{} equals depth, fluctuates during creation of path}
    \PYG{n}{width} \PYG{o}{=} \PYG{n}{getwidth}\PYG{p}{(}\PYG{n}{tree}\PYG{p}{)}
    \PYG{n}{depth} \PYG{o}{=} \PYG{n}{getdepth}\PYG{p}{(}\PYG{n}{tree}\PYG{p}{)} \PYG{o}{+} \PYG{l+m+mi}{1}  \PYG{c+c1}{\PYGZsh{} +1 for target values}
    \PYG{n}{path} \PYG{o}{=} \PYG{n}{np}\PYG{o}{.}\PYG{n}{zeros}\PYG{p}{(}\PYG{p}{(}\PYG{n}{width}\PYG{p}{,} \PYG{n}{depth}\PYG{p}{)}\PYG{p}{)}  \PYG{c+c1}{\PYGZsh{} Prelocate Memory}
    \PYG{n}{path}\PYG{p}{[}\PYG{p}{:}\PYG{p}{:}\PYG{p}{]} \PYG{o}{=} \PYG{k+kc}{None}  \PYG{c+c1}{\PYGZsh{} NaN in final result means branch is shorter than total depth}
    \PYG{n}{path}\PYG{p}{,} \PYG{n}{z1} \PYG{o}{=} \PYG{n}{path\PYGZus{}gen2}\PYG{p}{(}\PYG{n}{tree}\PYG{p}{,} \PYG{n}{width}\PYG{p}{,} \PYG{n}{depth}\PYG{p}{,} \PYG{n}{path}\PYG{p}{,} \PYG{n}{z2}\PYG{p}{,} \PYG{n}{z1}\PYG{p}{)}
    \PYG{k}{return} \PYG{n}{path}


\PYG{k}{def} \PYG{n+nf}{path\PYGZus{}gen2}\PYG{p}{(}\PYG{n}{tree}\PYG{p}{,} \PYG{n}{width}\PYG{p}{,} \PYG{n}{depth}\PYG{p}{,} \PYG{n}{path}\PYG{p}{,} \PYG{n}{z2}\PYG{p}{,} \PYG{n}{z1}\PYG{p}{)}\PYG{p}{:}
    \PYG{l+s+sd}{\PYGZdq{}\PYGZdq{}\PYGZdq{}Create a path Matrix which contains the structure of the tree.}

\PYG{l+s+sd}{    creates a matrix \PYGZsq{}path\PYGZsq{} that represents the structure of the tree and the decisions made at each node, last column contains the average MSE at that leaf}
\PYG{l+s+sd}{    the sooner a feature gets chosen as a split feature the more important it is (the farther on the left it appears in path matrix)}
\PYG{l+s+sd}{    order that leaves are written in (top to bottom): function will crawl to the rightmost leaf first (positive side), then jump back up one level and move one step to the left (loop)}

\PYG{l+s+sd}{    Arguments:}
\PYG{l+s+sd}{        tree \PYGZob{}decisionnode\PYGZcb{} \PYGZhy{}\PYGZhy{} tree of which the data structure is stored}
\PYG{l+s+sd}{        width \PYGZob{}int\PYGZcb{} \PYGZhy{}\PYGZhy{} width of the tree}
\PYG{l+s+sd}{        depth \PYGZob{}int\PYGZcb{} \PYGZhy{}\PYGZhy{} depth of the tree}
\PYG{l+s+sd}{        path \PYGZob{}[type]\PYGZcb{} \PYGZhy{}\PYGZhy{} current path matrix, gets updated during function calls}
\PYG{l+s+sd}{        z2 \PYGZob{}int\PYGZcb{} \PYGZhy{}\PYGZhy{} control variable for current depth}
\PYG{l+s+sd}{        z1 \PYGZob{}int\PYGZcb{} \PYGZhy{}\PYGZhy{} control variable for current width}

\PYG{l+s+sd}{    Returns:}
\PYG{l+s+sd}{        numpy.array \PYGZhy{}\PYGZhy{} the structure of the tree}
\PYG{l+s+sd}{    \PYGZdq{}\PYGZdq{}\PYGZdq{}}
    \PYG{k}{while} \PYG{n}{z1} \PYG{o}{\PYGZlt{}} \PYG{n}{width}\PYG{p}{:}  \PYG{c+c1}{\PYGZsh{} continue until total number of leaves is reached}
        \PYG{k}{if} \PYG{n}{tree}\PYG{o}{.}\PYG{n}{results} \PYG{o+ow}{is} \PYG{k+kc}{None}\PYG{p}{:}  \PYG{c+c1}{\PYGZsh{} = if current node is not a leaf}
            \PYG{n}{path}\PYG{p}{[}\PYG{n}{z1}\PYG{p}{,} \PYG{n}{z2}\PYG{p}{]} \PYG{o}{=} \PYG{n}{tree}\PYG{o}{.}\PYG{n}{col}  \PYG{c+c1}{\PYGZsh{} write split feature of that node into path matrix}
            \PYG{n}{z2} \PYG{o}{+}\PYG{o}{=} \PYG{l+m+mi}{1}  \PYG{c+c1}{\PYGZsh{} increase depth counter}
            \PYG{n}{path}\PYG{p}{,} \PYG{n}{z1} \PYG{o}{=} \PYG{n}{path\PYGZus{}gen2}\PYG{p}{(}\PYG{n}{tree}\PYG{o}{.}\PYG{n}{tb}\PYG{p}{,} \PYG{n}{width}\PYG{p}{,} \PYG{n}{depth}\PYG{p}{,} \PYG{n}{path}\PYG{p}{,} \PYG{n}{z2}\PYG{p}{,} \PYG{n}{z1}\PYG{p}{)}  \PYG{c+c1}{\PYGZsh{} recursively call path\PYGZus{}gen function in order to proceed to next deeper node in direction of tb}
            \PYG{k}{for} \PYG{n}{x} \PYG{o+ow}{in} \PYG{n+nb}{range}\PYG{p}{(}\PYG{n}{z2}\PYG{p}{)}\PYG{p}{:}
                \PYG{n}{path}\PYG{p}{[}\PYG{n}{z1}\PYG{p}{,} \PYG{n}{x}\PYG{p}{]} \PYG{o}{=} \PYG{n}{path}\PYG{p}{[}\PYG{n}{z1} \PYG{o}{\PYGZhy{}} \PYG{l+m+mi}{1}\PYG{p}{,} \PYG{n}{x}\PYG{p}{]}  \PYG{c+c1}{\PYGZsh{} assign the former columns the same value as the leaf above}
            \PYG{n}{path}\PYG{p}{,} \PYG{n}{z1} \PYG{o}{=} \PYG{n}{path\PYGZus{}gen2}\PYG{p}{(}\PYG{n}{tree}\PYG{o}{.}\PYG{n}{fb}\PYG{p}{,} \PYG{n}{width}\PYG{p}{,} \PYG{n}{depth}\PYG{p}{,} \PYG{n}{path}\PYG{p}{,} \PYG{n}{z2}\PYG{p}{,} \PYG{n}{z1}\PYG{p}{)}  \PYG{c+c1}{\PYGZsh{} recursively call path\PYGZus{}gen function in order to proceed to next deeper node in direction of fb}
            \PYG{n}{z2} \PYG{o}{\PYGZhy{}}\PYG{o}{=} \PYG{l+m+mi}{1}  \PYG{c+c1}{\PYGZsh{} after reaching the deepest fb leaf move up one level in depth}
            \PYG{k}{break}
        \PYG{k}{else}\PYG{p}{:}  \PYG{c+c1}{\PYGZsh{} = if current node is a leaf}
            \PYG{n}{path}\PYG{p}{[}\PYG{n}{z1}\PYG{p}{,} \PYG{o}{\PYGZhy{}}\PYG{l+m+mi}{1}\PYG{p}{]} \PYG{o}{=} \PYG{n}{np}\PYG{o}{.}\PYG{n}{mean}\PYG{p}{(}\PYG{n}{tree}\PYG{o}{.}\PYG{n}{results}\PYG{o}{.}\PYG{n}{keys}\PYG{p}{(}\PYG{p}{)}\PYG{p}{)}  \PYG{c+c1}{\PYGZsh{} put the average MSE in the last column of path}
            \PYG{n}{z1} \PYG{o}{+}\PYG{o}{=} \PYG{l+m+mi}{1}  \PYG{c+c1}{\PYGZsh{} current leaf is completely written into path, proceeding to next leaf}
            \PYG{k}{break}
    \PYG{k}{return} \PYG{n}{path}\PYG{p}{,} \PYG{n}{z1}  \PYG{c+c1}{\PYGZsh{} return the path matrix and current leaf number}


\PYG{k}{def} \PYG{n+nf}{check\PYGZus{}path}\PYG{p}{(}\PYG{n}{tree}\PYG{p}{,} \PYG{n}{result}\PYG{p}{)}\PYG{p}{:}
    \PYG{l+s+sd}{\PYGZdq{}\PYGZdq{}\PYGZdq{}Check if a tree contains MSE\PYGZus{}min (= True) or not (= False)}

\PYG{l+s+sd}{    Arguments:}
\PYG{l+s+sd}{        tree \PYGZob{}decisionnode\PYGZcb{} \PYGZhy{}\PYGZhy{} tree that gets searched for result}
\PYG{l+s+sd}{        result \PYGZob{}data\PYGZcb{} \PYGZhy{}\PYGZhy{} result that the tree is searched for}

\PYG{l+s+sd}{    Returns:}
\PYG{l+s+sd}{        bool \PYGZhy{}\PYGZhy{} True if result is in the tree, false if not}
\PYG{l+s+sd}{    \PYGZdq{}\PYGZdq{}\PYGZdq{}}
    \PYG{n}{path} \PYG{o}{=} \PYG{n}{path\PYGZus{}gen}\PYG{p}{(}\PYG{n}{tree}\PYG{p}{)}
    \PYG{k}{if} \PYG{n}{result} \PYG{o+ow}{in} \PYG{n}{path}\PYG{p}{[}\PYG{p}{:}\PYG{p}{,} \PYG{o}{\PYGZhy{}}\PYG{l+m+mi}{1}\PYG{p}{]}\PYG{p}{:}
        \PYG{k}{return} \PYG{k+kc}{True}
    \PYG{k}{else}\PYG{p}{:}
        \PYG{k}{return} \PYG{k+kc}{False}


\PYG{k}{def} \PYG{n+nf}{buildforest}\PYG{p}{(}\PYG{n}{data}\PYG{p}{,} \PYG{n}{n\PYGZus{}trees}\PYG{p}{,} \PYG{n}{scoref}\PYG{p}{,} \PYG{n}{n\PYGZus{}feat}\PYG{p}{,} \PYG{n}{min\PYGZus{}data}\PYG{p}{,} \PYG{n}{pruning}\PYG{p}{)}\PYG{p}{:}
    \PYG{l+s+sd}{\PYGZdq{}\PYGZdq{}\PYGZdq{}Growing the Random Forest}

\PYG{l+s+sd}{    The Random Forest consists of n\PYGZus{}trees. Each tree sees only a subset of the data and a subset of the features.}
\PYG{l+s+sd}{    Important: a tree never sees the original data set, only the performance of the classifying algorithm}
\PYG{l+s+sd}{    For significant conclusions enough trees must be generated in order to gain the statistical benefits that overcome bad outputs}

\PYG{l+s+sd}{    Arguments:}
\PYG{l+s+sd}{        * data \PYGZob{}numpy.array\PYGZcb{} \PYGZhy{}\PYGZhy{} data set the Forest is built upon}
\PYG{l+s+sd}{        * n\PYGZus{}trees \PYGZob{}int\PYGZcb{} \PYGZhy{}\PYGZhy{} number of trees in a Decision tree}
\PYG{l+s+sd}{        * scoref \PYGZob{}function\PYGZcb{} \PYGZhy{}\PYGZhy{} scoring metric for finding new nodes}
\PYG{l+s+sd}{        * n\PYGZus{}feat \PYGZob{}int\PYGZcb{} \PYGZhy{}\PYGZhy{} number of features in data}
\PYG{l+s+sd}{        * min\PYGZus{}data \PYGZob{}float\PYGZcb{} \PYGZhy{}\PYGZhy{} minimum percentage of all data sets that a tree will see}
\PYG{l+s+sd}{        * pruning \PYGZob{}bool\PYGZcb{} \PYGZhy{}\PYGZhy{} pruning enabled (\PYGZgt{}0) / disabled(=0)}

\PYG{l+s+sd}{    Returns:}
\PYG{l+s+sd}{        * RF \PYGZhy{}\PYGZhy{}  dictionary = importances of single features in the forest}
\PYG{l+s+sd}{        * prob\PYGZus{}current \PYGZhy{}\PYGZhy{} single value for importance, used for generating new biased feature sets}
\PYG{l+s+sd}{        * trees \PYGZhy{}\PYGZhy{} contains all single trees that stand in the Forest}
\PYG{l+s+sd}{    \PYGZdq{}\PYGZdq{}\PYGZdq{}}

    \PYG{c+c1}{\PYGZsh{} Initializations}
    \PYG{n}{prob\PYGZus{}current} \PYG{o}{=} \PYG{k+kc}{None}  \PYG{c+c1}{\PYGZsh{} Prelocate}
    \PYG{n}{RF} \PYG{o}{=} \PYG{p}{\PYGZob{}}\PYG{p}{\PYGZcb{}}  \PYG{c+c1}{\PYGZsh{} Prelocate dictionary for prioritizing important features}
    \PYG{n}{trees} \PYG{o}{=} \PYG{p}{[}\PYG{p}{]}  \PYG{c+c1}{\PYGZsh{} Prelocate list that will contain the trees that stand in the currently built forest}
    \PYG{n}{total\PYGZus{}best\PYGZus{}result} \PYG{o}{=} \PYG{k+kc}{None}  \PYG{c+c1}{\PYGZsh{} Prelocate}
    \PYG{n}{current\PYGZus{}best\PYGZus{}result} \PYG{o}{=} \PYG{k+kc}{None}  \PYG{c+c1}{\PYGZsh{} Prelocate}
    \PYG{n}{path\PYGZus{}min\PYGZus{}current} \PYG{o}{=} \PYG{p}{[}\PYG{p}{]}  \PYG{c+c1}{\PYGZsh{} Prelocate}
    \PYG{n}{wrongs} \PYG{o}{=} \PYG{l+m+mi}{0}  \PYG{c+c1}{\PYGZsh{} initialize number of (useless) trees that have only one node}

    \PYG{c+c1}{\PYGZsh{} build single trees}
    \PYG{k}{for} \PYG{n}{x} \PYG{o+ow}{in} \PYG{n+nb}{range}\PYG{p}{(}\PYG{n}{n\PYGZus{}trees}\PYG{p}{)}\PYG{p}{:}  \PYG{c+c1}{\PYGZsh{} n\PYGZus{}trees is the number of trees in the forest}

        \PYG{c+c1}{\PYGZsh{} select only subset of available datasets}
        \PYG{c+c1}{\PYGZsh{} create mask for randomly choosing subset of available datasets}
        \PYG{n}{mask\PYGZus{}sub\PYGZus{}data} \PYG{o}{=} \PYG{n}{np}\PYG{o}{.}\PYG{n}{zeros}\PYG{p}{(}\PYG{n}{data}\PYG{o}{.}\PYG{n}{shape}\PYG{p}{[}\PYG{l+m+mi}{0}\PYG{p}{]}\PYG{p}{,} \PYG{n}{dtype}\PYG{o}{=}\PYG{n+nb}{bool}\PYG{p}{)}  \PYG{c+c1}{\PYGZsh{} Prelocate}
        \PYG{c+c1}{\PYGZsh{} randomly choose the random datasets}
        \PYG{n}{rand\PYGZus{}data} \PYG{o}{=} \PYG{n}{np}\PYG{o}{.}\PYG{n}{random}\PYG{o}{.}\PYG{n}{choice}\PYG{p}{(}\PYG{n+nb}{range}\PYG{p}{(}\PYG{n}{data}\PYG{o}{.}\PYG{n}{shape}\PYG{p}{[}\PYG{l+m+mi}{0}\PYG{p}{]}\PYG{p}{)}\PYG{p}{,} \PYG{n}{size}\PYG{o}{=}\PYG{n+nb}{int}\PYG{p}{(}\PYG{n}{np}\PYG{o}{.}\PYG{n}{amax}\PYG{p}{(}\PYG{p}{(}\PYG{n}{np}\PYG{o}{.}\PYG{n}{around}\PYG{p}{(}\PYG{n+nb}{len}\PYG{p}{(}\PYG{n}{data}\PYG{p}{)} \PYG{o}{*} \PYG{n}{min\PYGZus{}data}\PYG{p}{,} \PYG{n}{decimals}\PYG{o}{=}\PYG{l+m+mi}{0}\PYG{p}{)}\PYG{p}{,}
                                                                             \PYG{n}{np}\PYG{o}{.}\PYG{n}{random}\PYG{o}{.}\PYG{n}{choice}\PYG{p}{(}\PYG{n+nb}{range}\PYG{p}{(}\PYG{n+nb}{len}\PYG{p}{(}\PYG{n}{data}\PYG{p}{)} \PYG{o}{\PYGZhy{}} \PYG{l+m+mi}{1}\PYG{p}{)}\PYG{p}{)} \PYG{o}{+} \PYG{l+m+mi}{1}\PYG{p}{)}\PYG{p}{,} \PYG{n}{axis}\PYG{o}{=}\PYG{k+kc}{None}\PYG{p}{)}\PYG{p}{)}\PYG{p}{,} \PYG{n}{replace}\PYG{o}{=}\PYG{k+kc}{False}\PYG{p}{,} \PYG{n}{p}\PYG{o}{=}\PYG{k+kc}{None}\PYG{p}{)}
        \PYG{c+c1}{\PYGZsh{} translate to selected data sets}
        \PYG{n}{mask\PYGZus{}sub\PYGZus{}data}\PYG{p}{[}\PYG{n}{rand\PYGZus{}data}\PYG{p}{]} \PYG{o}{=} \PYG{k+kc}{True}
        \PYG{n}{sub\PYGZus{}data} \PYG{o}{=} \PYG{n}{data}\PYG{p}{[}\PYG{n}{mask\PYGZus{}sub\PYGZus{}data}\PYG{p}{,} \PYG{p}{:}\PYG{p}{]}  \PYG{c+c1}{\PYGZsh{} random subset of datasets still including all features}

        \PYG{c+c1}{\PYGZsh{} select only subset of features}
        \PYG{c+c1}{\PYGZsh{} create mask for randomly choosing subset of available features}
        \PYG{n}{mask\PYGZus{}sub\PYGZus{}features} \PYG{o}{=} \PYG{n}{np}\PYG{o}{.}\PYG{n}{zeros}\PYG{p}{(}\PYG{n}{data}\PYG{o}{.}\PYG{n}{shape}\PYG{p}{[}\PYG{l+m+mi}{1}\PYG{p}{]}\PYG{p}{,} \PYG{n}{dtype}\PYG{o}{=}\PYG{n+nb}{bool}\PYG{p}{)}  \PYG{c+c1}{\PYGZsh{} Prelocate}
        \PYG{c+c1}{\PYGZsh{} randomly choose the random features}
        \PYG{n}{rand\PYGZus{}feat} \PYG{o}{=} \PYG{n}{np}\PYG{o}{.}\PYG{n}{random}\PYG{o}{.}\PYG{n}{choice}\PYG{p}{(}\PYG{n+nb}{range}\PYG{p}{(}\PYG{n}{data}\PYG{o}{.}\PYG{n}{shape}\PYG{p}{[}\PYG{l+m+mi}{1}\PYG{p}{]} \PYG{o}{\PYGZhy{}} \PYG{l+m+mi}{1}\PYG{p}{)}\PYG{p}{,} \PYG{n}{size}\PYG{o}{=}\PYG{n}{np}\PYG{o}{.}\PYG{n}{random}\PYG{o}{.}\PYG{n}{choice}\PYG{p}{(}\PYG{n+nb}{range}\PYG{p}{(}\PYG{n+nb}{len}\PYG{p}{(}\PYG{n}{data}\PYG{p}{[}\PYG{l+m+mi}{0}\PYG{p}{]}\PYG{p}{)} \PYG{o}{\PYGZhy{}} \PYG{l+m+mi}{1}\PYG{p}{)}\PYG{p}{)} \PYG{o}{+} \PYG{l+m+mi}{1}\PYG{p}{,} \PYG{n}{replace}\PYG{o}{=}\PYG{k+kc}{False}\PYG{p}{,} \PYG{n}{p}\PYG{o}{=}\PYG{k+kc}{None}\PYG{p}{)}
        \PYG{c+c1}{\PYGZsh{} sort ascending}
        \PYG{n}{rand\PYGZus{}feat} \PYG{o}{=} \PYG{n}{np}\PYG{o}{.}\PYG{n}{sort}\PYG{p}{(}\PYG{n}{rand\PYGZus{}feat}\PYG{p}{)}
        \PYG{c+c1}{\PYGZsh{} append last column with result}
        \PYG{n}{rand\PYGZus{}feat} \PYG{o}{=} \PYG{n}{np}\PYG{o}{.}\PYG{n}{append}\PYG{p}{(}\PYG{n}{rand\PYGZus{}feat}\PYG{p}{,} \PYG{n}{data}\PYG{o}{.}\PYG{n}{shape}\PYG{p}{[}\PYG{l+m+mi}{1}\PYG{p}{]} \PYG{o}{\PYGZhy{}} \PYG{l+m+mi}{1}\PYG{p}{)}
        \PYG{c+c1}{\PYGZsh{} translate to selected features}
        \PYG{n}{mask\PYGZus{}sub\PYGZus{}features}\PYG{p}{[}\PYG{n}{rand\PYGZus{}feat}\PYG{p}{]} \PYG{o}{=} \PYG{k+kc}{True}

        \PYG{n}{sub\PYGZus{}data} \PYG{o}{=} \PYG{n}{sub\PYGZus{}data}\PYG{p}{[}\PYG{p}{:}\PYG{p}{,} \PYG{n}{mask\PYGZus{}sub\PYGZus{}features}\PYG{p}{]}  \PYG{c+c1}{\PYGZsh{} random subset of datasets and random subset of features}
        \PYG{c+c1}{\PYGZsh{} print \PYGZdq{}sub\PYGZus{}data = \PYGZdq{} + str(sub\PYGZus{}data) \PYGZsh{} Debugging line}

        \PYG{c+c1}{\PYGZsh{} build the tree from the subset data, last column contains result}
        \PYG{c+c1}{\PYGZsh{} print \PYGZdq{}building tree\PYGZdq{} \PYGZsh{} Debugging Line}
        \PYG{n}{tree} \PYG{o}{=} \PYG{n}{buildtree}\PYG{p}{(}\PYG{n}{sub\PYGZus{}data}\PYG{p}{,} \PYG{n}{scoref}\PYG{p}{)}

        \PYG{c+c1}{\PYGZsh{} pruning the tree (if hyperparameter is enabled)}
        \PYG{k}{if} \PYG{n}{pruning} \PYG{o}{\PYGZgt{}} \PYG{l+m+mi}{0}\PYG{p}{:}
            \PYG{n}{prune}\PYG{p}{(}\PYG{n}{tree}\PYG{p}{,} \PYG{n}{pruning}\PYG{p}{)}

        \PYG{c+c1}{\PYGZsh{} draw the tree and create path matrix}
        \PYG{c+c1}{\PYGZsh{} drawtree(tree, jpeg=\PYGZsq{}treeview\PYGZus{}RF.jpg\PYGZsq{}) \PYGZsh{} Debbuging Line}

        \PYG{k}{if} \PYG{n}{getdepth}\PYG{p}{(}\PYG{n}{tree}\PYG{p}{)} \PYG{o+ow}{is} \PYG{l+m+mi}{0}\PYG{p}{:}  \PYG{c+c1}{\PYGZsh{} if the tree contains only a single node \PYGZhy{}\PYGZhy{}\PYGZgt{} tree is useless}
            \PYG{n}{wrongs} \PYG{o}{+}\PYG{o}{=} \PYG{l+m+mi}{1}
            \PYG{c+c1}{\PYGZsh{} print \PYGZdq{}wrongs: \PYGZdq{} + str(wrongs) \PYGZsh{} Debugging Line}
        \PYG{k}{else}\PYG{p}{:}  \PYG{c+c1}{\PYGZsh{} only increment feature counter if tree has more than one leaf}
            \PYG{n}{path} \PYG{o}{=} \PYG{n}{path\PYGZus{}gen}\PYG{p}{(}\PYG{n}{tree}\PYG{p}{)}  \PYG{c+c1}{\PYGZsh{} create path to current tree}
            \PYG{n}{current\PYGZus{}best\PYGZus{}result} \PYG{o}{=} \PYG{n}{np}\PYG{o}{.}\PYG{n}{max}\PYG{p}{(}\PYG{n}{path}\PYG{p}{[}\PYG{p}{:}\PYG{p}{,} \PYG{o}{\PYGZhy{}}\PYG{l+m+mi}{1}\PYG{p}{]}\PYG{p}{)}
            \PYG{n}{path\PYGZus{}min\PYGZus{}current} \PYG{o}{=} \PYG{n}{path}\PYG{p}{[}\PYG{n}{np}\PYG{o}{.}\PYG{n}{argmax}\PYG{p}{(}\PYG{n}{path}\PYG{p}{[}\PYG{p}{:}\PYG{p}{,} \PYG{o}{\PYGZhy{}}\PYG{l+m+mi}{1}\PYG{p}{]}\PYG{p}{)}\PYG{p}{]}

            \PYG{c+c1}{\PYGZsh{} update best result and corresponding path}
            \PYG{k}{if} \PYG{n}{total\PYGZus{}best\PYGZus{}result} \PYG{o+ow}{is} \PYG{k+kc}{None} \PYG{o+ow}{or} \PYG{n}{current\PYGZus{}best\PYGZus{}result} \PYG{o}{\PYGZgt{}} \PYG{n}{total\PYGZus{}best\PYGZus{}result}\PYG{p}{:}
                \PYG{n}{total\PYGZus{}best\PYGZus{}result} \PYG{o}{=} \PYG{n}{current\PYGZus{}best\PYGZus{}result}

            \PYG{c+c1}{\PYGZsh{} update the RF dictionary that rewards / punishes features}
            \PYG{n}{update\PYGZus{}RF}\PYG{p}{(}\PYG{n}{RF}\PYG{p}{,} \PYG{n}{path\PYGZus{}min\PYGZus{}current}\PYG{p}{,} \PYG{n}{tree}\PYG{p}{,} \PYG{n}{rand\PYGZus{}feat}\PYG{p}{)}
            \PYG{n}{trees}\PYG{o}{.}\PYG{n}{append}\PYG{p}{(}\PYG{n}{tree}\PYG{p}{)}
    \PYG{c+c1}{\PYGZsh{} print \PYGZdq{}RF: \PYGZdq{} + str(RF) \PYGZsh{} Debugging Line}
    \PYG{c+c1}{\PYGZsh{} print \PYGZdq{}Returning RF\PYGZdq{} \PYGZsh{} Debugging Line}

    \PYG{c+c1}{\PYGZsh{} a \PYGZdq{}wrong\PYGZdq{} tree is a tree with only one node that has no power to gain additional insight and therefore is useless...}
    \PYG{n+nb}{print} \PYG{l+s+s2}{\PYGZdq{}}\PYG{l+s+s2}{wrongs: }\PYG{l+s+s2}{\PYGZdq{}} \PYG{o}{+} \PYG{n+nb}{str}\PYG{p}{(}\PYG{n}{wrongs}\PYG{p}{)} \PYG{o}{+} \PYG{l+s+s2}{\PYGZdq{}}\PYG{l+s+s2}{/}\PYG{l+s+s2}{\PYGZdq{}} \PYG{o}{+} \PYG{n+nb}{str}\PYG{p}{(}\PYG{n}{n\PYGZus{}trees}\PYG{p}{)}

    \PYG{c+c1}{\PYGZsh{} Transform the counter for rewarded / punished features from RF dictionary into a proportionate number}
    \PYG{c+c1}{\PYGZsh{} set up scaler that projects accumulated values of RF onto a scale between 0 and 1}
    \PYG{n}{min\PYGZus{}max\PYGZus{}scaler} \PYG{o}{=} \PYG{n}{preprocessing}\PYG{o}{.}\PYG{n}{MinMaxScaler}\PYG{p}{(}\PYG{n}{feature\PYGZus{}range}\PYG{o}{=}\PYG{p}{(}\PYG{l+m+mi}{0}\PYG{p}{,} \PYG{l+m+mi}{1}\PYG{p}{)}\PYG{p}{)}
    \PYG{c+c1}{\PYGZsh{} take only values of RF, reshape them (otherwise deprecation warning), make them numpy array, and scale them (again) between 0 and 1}
    \PYG{n}{temp} \PYG{o}{=} \PYG{n}{min\PYGZus{}max\PYGZus{}scaler}\PYG{o}{.}\PYG{n}{fit\PYGZus{}transform}\PYG{p}{(}\PYG{n}{np}\PYG{o}{.}\PYG{n}{nan\PYGZus{}to\PYGZus{}num}\PYG{p}{(}\PYG{n}{np}\PYG{o}{.}\PYG{n}{array}\PYG{p}{(}\PYG{n}{RF}\PYG{o}{.}\PYG{n}{values}\PYG{p}{(}\PYG{p}{)}\PYG{p}{)}\PYG{p}{)}\PYG{o}{.}\PYG{n}{reshape}\PYG{p}{(}\PYG{o}{\PYGZhy{}}\PYG{l+m+mi}{1}\PYG{p}{,} \PYG{l+m+mi}{1}\PYG{p}{)}\PYG{p}{)}
    \PYG{c+c1}{\PYGZsh{} sum up values of RF, divide each value of RF by sum to get percentage \PYGZhy{}\PYGZhy{}\PYGZgt{} must sum up to 1}
    \PYG{n}{temp\PYGZus{}sum} \PYG{o}{=} \PYG{n}{np}\PYG{o}{.}\PYG{n}{sum}\PYG{p}{(}\PYG{n}{temp}\PYG{p}{)}
    \PYG{n}{temp\PYGZus{}percent} \PYG{o}{=} \PYG{n}{temp} \PYG{o}{*} \PYG{p}{(}\PYG{l+m+mf}{1.0} \PYG{o}{/} \PYG{n}{temp\PYGZus{}sum}\PYG{p}{)}
    \PYG{c+c1}{\PYGZsh{} print temp\PYGZus{}percent}
    \PYG{c+c1}{\PYGZsh{} update values in RF with scaled percentage values}
    \PYG{n}{i} \PYG{o}{=} \PYG{l+m+mi}{0}
    \PYG{k}{for} \PYG{n}{key} \PYG{o+ow}{in} \PYG{n}{RF}\PYG{p}{:}
        \PYG{n}{RF}\PYG{p}{[}\PYG{n}{key}\PYG{p}{]} \PYG{o}{=} \PYG{n}{temp\PYGZus{}percent}\PYG{p}{[}\PYG{n}{i}\PYG{p}{]}\PYG{p}{[}\PYG{l+m+mi}{0}\PYG{p}{]}  \PYG{c+c1}{\PYGZsh{} [0] because otherwise there would be an array inside the dictionary RF}
        \PYG{n}{i} \PYG{o}{+}\PYG{o}{=} \PYG{l+m+mi}{1}
    \PYG{c+c1}{\PYGZsh{} print \PYGZdq{}RF: \PYGZdq{} + str(RF) \PYGZsh{} Debugging Line}

    \PYG{c+c1}{\PYGZsh{} build a dictionary of most important features in a tree and how often they were chosen}
    \PYG{c+c1}{\PYGZsh{} create weights of features}
    \PYG{n}{weights} \PYG{o}{=} \PYG{p}{\PYGZob{}}\PYG{p}{\PYGZcb{}}  \PYG{c+c1}{\PYGZsh{} Prelocate}
    \PYG{n}{weights\PYGZus{}sorted} \PYG{o}{=} \PYG{p}{\PYGZob{}}\PYG{p}{\PYGZcb{}}  \PYG{c+c1}{\PYGZsh{} Prelocate}
    \PYG{c+c1}{\PYGZsh{} transfer values from dictionary into list}
    \PYG{k}{for} \PYG{n}{key}\PYG{p}{,} \PYG{n}{value} \PYG{o+ow}{in} \PYG{n}{RF}\PYG{o}{.}\PYG{n}{items}\PYG{p}{(}\PYG{p}{)}\PYG{p}{:}
        \PYG{n}{weights}\PYG{p}{[}\PYG{n}{key}\PYG{p}{]} \PYG{o}{=} \PYG{n+nb}{float}\PYG{p}{(}\PYG{n}{value}\PYG{p}{)}  \PYG{c+c1}{\PYGZsh{} create relative weight}
    \PYG{c+c1}{\PYGZsh{} some features might not get picked once, so their probability is set to zero}
    \PYG{k}{if} \PYG{n+nb}{len}\PYG{p}{(}\PYG{n}{weights}\PYG{p}{)} \PYG{o}{\PYGZlt{}} \PYG{n}{n\PYGZus{}feat}\PYG{p}{:}
        \PYG{k}{for} \PYG{n}{key} \PYG{o+ow}{in} \PYG{n+nb}{range}\PYG{p}{(}\PYG{n}{n\PYGZus{}feat}\PYG{p}{)}\PYG{p}{:}
            \PYG{k}{if} \PYG{n}{key} \PYG{o+ow}{not} \PYG{o+ow}{in} \PYG{n}{weights}\PYG{p}{:}
                \PYG{n}{weights}\PYG{p}{[}\PYG{n}{key}\PYG{p}{]} \PYG{o}{=} \PYG{l+m+mi}{0}
    \PYG{c+c1}{\PYGZsh{} print \PYGZdq{}weights = \PYGZdq{} + str(weights) \PYGZsh{} Debugging Line}
    \PYG{n}{weights\PYGZus{}sorted} \PYG{o}{=} \PYG{n+nb}{dict}\PYG{p}{(}\PYG{n+nb}{sorted}\PYG{p}{(}\PYG{n}{weights}\PYG{o}{.}\PYG{n}{items}\PYG{p}{(}\PYG{p}{)}\PYG{p}{,} \PYG{n}{key}\PYG{o}{=}\PYG{k}{lambda} \PYG{n}{value}\PYG{p}{:} \PYG{n}{value}\PYG{p}{[}\PYG{l+m+mi}{0}\PYG{p}{]}\PYG{p}{,} \PYG{n}{reverse}\PYG{o}{=}\PYG{k+kc}{False}\PYG{p}{)}\PYG{p}{)}  \PYG{c+c1}{\PYGZsh{} sort by frequency = importance}

    \PYG{c+c1}{\PYGZsh{} print \PYGZdq{}importance of features in random forest: \PYGZdq{} + str(weights\PYGZus{}sorted) \PYGZsh{} Debugging Line}

    \PYG{n}{prob\PYGZus{}current} \PYG{o}{=} \PYG{n}{np}\PYG{o}{.}\PYG{n}{array}\PYG{p}{(}\PYG{n}{weights\PYGZus{}sorted}\PYG{o}{.}\PYG{n}{values}\PYG{p}{(}\PYG{p}{)}\PYG{p}{)}  \PYG{c+c1}{\PYGZsh{} extract only the values of feature importance}

    \PYG{k}{return} \PYG{n}{RF}\PYG{p}{,} \PYG{n}{prob\PYGZus{}current}\PYG{p}{,} \PYG{n}{trees}


\PYG{k}{def} \PYG{n+nf}{update\PYGZus{}RF}\PYG{p}{(}\PYG{n}{RF}\PYG{p}{,} \PYG{n}{path}\PYG{p}{,} \PYG{n}{tree}\PYG{p}{,} \PYG{n}{rand\PYGZus{}feat}\PYG{p}{)}\PYG{p}{:}
    \PYG{l+s+sd}{\PYGZdq{}\PYGZdq{}\PYGZdq{}for each tree the features that lead to the leaf with the lowest Error will get rewarded.}
\PYG{l+s+sd}{    Features that don\PYGZsq{}t lead to the leaf with the lowest Error will get punished (only by 20\PYGZpc{} of}
\PYG{l+s+sd}{    the amount the \PYGZdq{}good\PYGZdq{} featurtes get rewarded).}


\PYG{l+s+sd}{    RF is a dictionary that gets updated after a new tree is built and thus contains the cummulation of all}
\PYG{l+s+sd}{    feature appearences in the whole forest.}

\PYG{l+s+sd}{    Arguments:}
\PYG{l+s+sd}{        * RF \PYGZob{}dict\PYGZcb{} \PYGZhy{}\PYGZhy{} dictionary that counts occurrence / absence of different features}
\PYG{l+s+sd}{        * path \PYGZob{}numpy.array\PYGZcb{} \PYGZhy{}\PYGZhy{} structure of the current tree}
\PYG{l+s+sd}{        * tree \PYGZob{}decisionnode\PYGZcb{} \PYGZhy{}\PYGZhy{} tree that gets examined}
\PYG{l+s+sd}{        * rand\PYGZus{}feat \PYGZob{}list\PYGZcb{} \PYGZhy{}\PYGZhy{} boolean mask of selected features (1 = selected, 0 = not selected)}

\PYG{l+s+sd}{    Returns:}
\PYG{l+s+sd}{        * RF \PYGZhy{}\PYGZhy{} updated dictionary that counts occurrence / absence of different features}
\PYG{l+s+sd}{    \PYGZdq{}\PYGZdq{}\PYGZdq{}}
    \PYG{n}{current\PYGZus{}depth} \PYG{o}{=} \PYG{n}{getdepth}\PYG{p}{(}\PYG{n}{tree}\PYG{p}{)}
    \PYG{c+c1}{\PYGZsh{} print \PYGZdq{}current path: \PYGZdq{} + str(path) \PYGZsh{} Debugging Line}
    \PYG{c+c1}{\PYGZsh{} print  \PYGZdq{}current depth = \PYGZdq{} + str(getdepth(tree)) \PYGZsh{} Debugging Line}
    \PYG{c+c1}{\PYGZsh{} print \PYGZdq{}current col: \PYGZdq{} + str(tree.col) \PYGZsh{} Debugging Line}
    \PYG{k}{if} \PYG{n}{current\PYGZus{}depth} \PYG{o}{==} \PYG{l+m+mi}{0}\PYG{p}{:}
        \PYG{k}{return} \PYG{n}{RF}
    \PYG{n}{MSE\PYGZus{}min} \PYG{o}{=} \PYG{n}{path}\PYG{p}{[}\PYG{o}{\PYGZhy{}}\PYG{l+m+mi}{1}\PYG{p}{]}
    \PYG{c+c1}{\PYGZsh{} print \PYGZdq{}MSE\PYGZus{}min: \PYGZdq{} + str(MSE\PYGZus{}min) \PYGZsh{} Debugging Line}
    \PYG{c+c1}{\PYGZsh{} print \PYGZdq{}Checking if MSE\PYGZus{}min is in True branch\PYGZdq{} \PYGZsh{} Debugging Line}
    \PYG{k}{if} \PYG{n}{check\PYGZus{}path}\PYG{p}{(}\PYG{n}{tree}\PYG{o}{.}\PYG{n}{tb}\PYG{p}{,} \PYG{n}{MSE\PYGZus{}min}\PYG{p}{)} \PYG{o+ow}{is} \PYG{k+kc}{True}\PYG{p}{:}
        \PYG{c+c1}{\PYGZsh{} print \PYGZdq{}MSE\PYGZus{}min is in True Branch\PYGZdq{} \PYGZsh{} Debugging Line}

        \PYG{c+c1}{\PYGZsh{} initialize the feature in dictionary RF if it appears for the first time}
        \PYG{k}{if} \PYG{n}{rand\PYGZus{}feat}\PYG{p}{[}\PYG{n+nb}{int}\PYG{p}{(}\PYG{n}{tree}\PYG{o}{.}\PYG{n}{col}\PYG{p}{)}\PYG{p}{]} \PYG{o+ow}{not} \PYG{o+ow}{in} \PYG{n}{RF}\PYG{p}{:}
            \PYG{n}{RF}\PYG{p}{[}\PYG{n}{rand\PYGZus{}feat}\PYG{p}{[}\PYG{n+nb}{int}\PYG{p}{(}\PYG{n}{tree}\PYG{o}{.}\PYG{n}{col}\PYG{p}{)}\PYG{p}{]}\PYG{p}{]} \PYG{o}{=} \PYG{n+nb}{float}\PYG{p}{(}\PYG{n}{current\PYGZus{}depth}\PYG{p}{)}
        \PYG{k}{else}\PYG{p}{:}  \PYG{c+c1}{\PYGZsh{} if the feature is already present in dictionary RF, increase counter}
            \PYG{n}{RF}\PYG{p}{[}\PYG{n}{rand\PYGZus{}feat}\PYG{p}{[}\PYG{n+nb}{int}\PYG{p}{(}\PYG{n}{tree}\PYG{o}{.}\PYG{n}{col}\PYG{p}{)}\PYG{p}{]}\PYG{p}{]} \PYG{o}{+}\PYG{o}{=} \PYG{n+nb}{float}\PYG{p}{(}\PYG{n}{current\PYGZus{}depth}\PYG{p}{)}
        \PYG{c+c1}{\PYGZsh{} print \PYGZdq{}added \PYGZdq{} + str(current\PYGZus{}depth) + \PYGZdq{} to feature  \PYGZdq{} + str(tree.col) \PYGZsh{} Debugging Line}
        \PYG{c+c1}{\PYGZsh{} print \PYGZdq{}current RF: \PYGZdq{} + str(RF) \PYGZsh{} Debugging Line}
        \PYG{n}{update\PYGZus{}RF}\PYG{p}{(}\PYG{n}{RF}\PYG{p}{,} \PYG{n}{path}\PYG{p}{[}\PYG{l+m+mi}{1}\PYG{p}{:}\PYG{p}{]}\PYG{p}{,} \PYG{n}{tree}\PYG{o}{.}\PYG{n}{tb}\PYG{p}{,} \PYG{n}{rand\PYGZus{}feat}\PYG{p}{)}  \PYG{c+c1}{\PYGZsh{} recursively jump into update\PYGZus{}RF again with shortened path at next level in true branch}
    \PYG{k}{else}\PYG{p}{:}
        \PYG{c+c1}{\PYGZsh{} print \PYGZdq{}MSE\PYGZus{}min is not in True Branch\PYGZdq{} \PYGZsh{} Debugging Line}
        \PYG{c+c1}{\PYGZsh{} print \PYGZdq{}Checking if MSE\PYGZus{}min is in False Branch\PYGZdq{} \PYGZsh{} Debugging Line}
        \PYG{k}{if} \PYG{n}{check\PYGZus{}path}\PYG{p}{(}\PYG{n}{tree}\PYG{o}{.}\PYG{n}{fb}\PYG{p}{,} \PYG{n}{MSE\PYGZus{}min}\PYG{p}{)} \PYG{o+ow}{is} \PYG{k+kc}{True}\PYG{p}{:}
            \PYG{c+c1}{\PYGZsh{} print \PYGZdq{}MSE\PYGZus{}min is in False Branch\PYGZdq{} \PYGZsh{} Debugging Line}
            \PYG{k}{if} \PYG{n}{rand\PYGZus{}feat}\PYG{p}{[}\PYG{n+nb}{int}\PYG{p}{(}\PYG{n}{tree}\PYG{o}{.}\PYG{n}{col}\PYG{p}{)}\PYG{p}{]} \PYG{o+ow}{not} \PYG{o+ow}{in} \PYG{n}{RF}\PYG{p}{:}  \PYG{c+c1}{\PYGZsh{} initialize the feature in dictionary RF if it appears for the first time}
                \PYG{n}{RF}\PYG{p}{[}\PYG{n}{rand\PYGZus{}feat}\PYG{p}{[}\PYG{n+nb}{int}\PYG{p}{(}\PYG{n}{tree}\PYG{o}{.}\PYG{n}{col}\PYG{p}{)}\PYG{p}{]}\PYG{p}{]} \PYG{o}{=} \PYG{o}{\PYGZhy{}}\PYG{l+m+mf}{0.2} \PYG{o}{*} \PYG{n+nb}{float}\PYG{p}{(}\PYG{n}{current\PYGZus{}depth}\PYG{p}{)}
            \PYG{k}{else}\PYG{p}{:}  \PYG{c+c1}{\PYGZsh{} if the feature is already present in dictionary RF, decrease counter}
                \PYG{n}{RF}\PYG{p}{[}\PYG{n}{rand\PYGZus{}feat}\PYG{p}{[}\PYG{n+nb}{int}\PYG{p}{(}\PYG{n}{tree}\PYG{o}{.}\PYG{n}{col}\PYG{p}{)}\PYG{p}{]}\PYG{p}{]} \PYG{o}{\PYGZhy{}}\PYG{o}{=} \PYG{n+nb}{float}\PYG{p}{(}\PYG{n}{current\PYGZus{}depth}\PYG{p}{)} \PYG{o}{*} \PYG{l+m+mf}{0.2}
            \PYG{c+c1}{\PYGZsh{} print \PYGZdq{}subtracted \PYGZdq{} + str(current\PYGZus{}depth*0.2) + \PYGZdq{} from feature \PYGZdq{} + str(tree.col) \PYGZsh{} Debugging Line}
            \PYG{c+c1}{\PYGZsh{} print \PYGZdq{}current RF: \PYGZdq{} + str(RF) \PYGZsh{} Debugging Line}
            \PYG{n}{update\PYGZus{}RF}\PYG{p}{(}\PYG{n}{RF}\PYG{p}{,} \PYG{n}{path}\PYG{p}{[}\PYG{l+m+mi}{1}\PYG{p}{:}\PYG{p}{]}\PYG{p}{,} \PYG{n}{tree}\PYG{o}{.}\PYG{n}{fb}\PYG{p}{,} \PYG{n}{rand\PYGZus{}feat}\PYG{p}{)}  \PYG{c+c1}{\PYGZsh{} recursively jump into update\PYGZus{}RF with shortened path at next level in false branch}


\PYG{k}{def} \PYG{n+nf}{forest\PYGZus{}predict}\PYG{p}{(}\PYG{n}{data}\PYG{p}{,} \PYG{n}{trees}\PYG{p}{,} \PYG{n}{prob}\PYG{p}{,} \PYG{n}{n\PYGZus{}configs}\PYG{p}{,} \PYG{n}{biased}\PYG{p}{)}\PYG{p}{:}
    \PYG{l+s+sd}{\PYGZdq{}\PYGZdq{}\PYGZdq{}Predict performance of new feature sets}

\PYG{l+s+sd}{    Predicts biased and unbiased feature sets in the before constructed Random Forest.}
\PYG{l+s+sd}{    Feature sets are predicted in every single Decision Tree in the Random Forest.}
\PYG{l+s+sd}{    Results are represented as (mean+0.1*var) and (variance+0.1*mean) for each feature set.}
\PYG{l+s+sd}{    The two best feature sets are selected to be sent into the :ref:{}`MLA \PYGZlt{}MLA\PYGZgt{}{}`.}


\PYG{l+s+sd}{    Arguments:}
\PYG{l+s+sd}{        * data \PYGZob{}numpy.array\PYGZcb{} \PYGZhy{}\PYGZhy{} contains all previous computing runs}
\PYG{l+s+sd}{        * trees \PYGZob{}decisionnodes\PYGZcb{} \PYGZhy{}\PYGZhy{} the trees that make up the Random Forest}
\PYG{l+s+sd}{        * prob \PYGZob{}array of floats\PYGZcb{} \PYGZhy{}\PYGZhy{} probability that a feature gets chosen into a feature set}
\PYG{l+s+sd}{        * n\PYGZus{}configs \PYGZob{}int\PYGZcb{} \PYGZhy{}\PYGZhy{} number of feature sets to be generated}
\PYG{l+s+sd}{        * biased \PYGZob{}bool\PYGZcb{} \PYGZhy{}\PYGZhy{} true for biased feature selection, false for unbiased feature selection}

\PYG{l+s+sd}{    Returns:}
\PYG{l+s+sd}{        * best mean \PYGZhy{}\PYGZhy{} highest average of all predicted feature sets}
\PYG{l+s+sd}{        * best feature set mean \PYGZhy{}\PYGZhy{} corresponding boolean list of features (0=feature not chosen, 1=feature chosen)}
\PYG{l+s+sd}{        * best var \PYGZhy{}\PYGZhy{} highest variance of all predicted feature sets}
\PYG{l+s+sd}{        * best feature set var \PYGZhy{}\PYGZhy{} corresponding boolean list of features (0=feature not chosen, 1=feature chosen)}
\PYG{l+s+sd}{    \PYGZdq{}\PYGZdq{}\PYGZdq{}}
    \PYG{k}{if} \PYG{n}{biased} \PYG{o+ow}{is} \PYG{o+ow}{not} \PYG{k+kc}{True}\PYG{p}{:}
        \PYG{n}{prob} \PYG{o}{=} \PYG{k+kc}{None}
    \PYG{c+c1}{\PYGZsh{} print \PYGZdq{}prob: \PYGZdq{} + str(prob) \PYGZsh{} Debugging Line}
    \PYG{c+c1}{\PYGZsh{} Prelocate variables}
    \PYG{n}{mean} \PYG{o}{=} \PYG{n}{np}\PYG{o}{.}\PYG{n}{zeros}\PYG{p}{(}\PYG{n}{n\PYGZus{}configs}\PYG{p}{)}
    \PYG{n}{var} \PYG{o}{=} \PYG{n}{np}\PYG{o}{.}\PYG{n}{zeros}\PYG{p}{(}\PYG{n}{n\PYGZus{}configs}\PYG{p}{)}
    \PYG{n}{best\PYGZus{}mean} \PYG{o}{=} \PYG{n}{np}\PYG{o}{.}\PYG{n}{array}\PYG{p}{(}\PYG{p}{[}\PYG{l+m+mi}{0}\PYG{p}{]}\PYG{p}{)}
    \PYG{n}{best\PYGZus{}var} \PYG{o}{=} \PYG{n}{np}\PYG{o}{.}\PYG{n}{array}\PYG{p}{(}\PYG{p}{[}\PYG{l+m+mi}{0}\PYG{p}{]}\PYG{p}{)}
    \PYG{n}{best\PYGZus{}featureset\PYGZus{}mean} \PYG{o}{=} \PYG{n}{np}\PYG{o}{.}\PYG{n}{array}\PYG{p}{(}\PYG{p}{[}\PYG{l+m+mi}{0}\PYG{p}{]}\PYG{p}{)}
    \PYG{n}{best\PYGZus{}featureset\PYGZus{}var} \PYG{o}{=} \PYG{n}{np}\PYG{o}{.}\PYG{n}{array}\PYG{p}{(}\PYG{p}{[}\PYG{l+m+mi}{0}\PYG{p}{]}\PYG{p}{)}

    \PYG{c+c1}{\PYGZsh{} new config (=feature set) is generated}
    \PYG{k}{for} \PYG{n}{x} \PYG{o+ow}{in} \PYG{n+nb}{range}\PYG{p}{(}\PYG{n}{n\PYGZus{}configs}\PYG{p}{)}\PYG{p}{:}  \PYG{c+c1}{\PYGZsh{} n\PYGZus{}configs\PYGZus{}biased is hyperparameter}
        \PYG{c+c1}{\PYGZsh{} create mask for choosing subfeatures}
        \PYG{n}{mask\PYGZus{}sub\PYGZus{}features} \PYG{o}{=} \PYG{n}{np}\PYG{o}{.}\PYG{n}{zeros}\PYG{p}{(}\PYG{n}{data}\PYG{o}{.}\PYG{n}{shape}\PYG{p}{[}\PYG{l+m+mi}{1}\PYG{p}{]} \PYG{o}{\PYGZhy{}} \PYG{l+m+mi}{1}\PYG{p}{,} \PYG{n}{dtype}\PYG{o}{=}\PYG{n+nb}{bool}\PYG{p}{)}  \PYG{c+c1}{\PYGZsh{} Prelocate Memory}
        \PYG{k}{if} \PYG{n}{prob} \PYG{o+ow}{is} \PYG{o+ow}{not} \PYG{k+kc}{None}\PYG{p}{:}
            \PYG{n}{rand\PYGZus{}feat} \PYG{o}{=} \PYG{n}{np}\PYG{o}{.}\PYG{n}{random}\PYG{o}{.}\PYG{n}{choice}\PYG{p}{(}\PYG{n+nb}{range}\PYG{p}{(}\PYG{n}{data}\PYG{o}{.}\PYG{n}{shape}\PYG{p}{[}\PYG{l+m+mi}{1}\PYG{p}{]} \PYG{o}{\PYGZhy{}} \PYG{l+m+mi}{1}\PYG{p}{)}\PYG{p}{,} \PYG{n}{size}\PYG{o}{=}\PYG{n+nb}{int}\PYG{p}{(}\PYG{n}{np}\PYG{o}{.}\PYG{n}{min}\PYG{p}{(}\PYG{p}{(}\PYG{n}{np}\PYG{o}{.}\PYG{n}{random}\PYG{o}{.}\PYG{n}{choice}\PYG{p}{(}\PYG{n+nb}{range}\PYG{p}{(}\PYG{n+nb}{len}\PYG{p}{(}\PYG{n}{data}\PYG{p}{[}\PYG{l+m+mi}{0}\PYG{p}{]}\PYG{p}{)} \PYG{o}{\PYGZhy{}} \PYG{l+m+mi}{1}\PYG{p}{)}\PYG{p}{)} \PYG{o}{+} \PYG{l+m+mi}{1}\PYG{p}{,} \PYG{n+nb}{len}\PYG{p}{(}\PYG{n}{np}\PYG{o}{.}\PYG{n}{nonzero}\PYG{p}{(}\PYG{n}{prob}\PYG{p}{)}\PYG{p}{[}\PYG{l+m+mi}{0}\PYG{p}{]}\PYG{p}{)}\PYG{p}{)}\PYG{p}{)}\PYG{p}{)}\PYG{p}{,}
                                         \PYG{n}{replace}\PYG{o}{=}\PYG{k+kc}{False}\PYG{p}{,} \PYG{n}{p}\PYG{o}{=}\PYG{n}{prob}\PYG{p}{)}  \PYG{c+c1}{\PYGZsh{} size must be \PYGZlt{}= nonzero values of p, otherwise one feature gets selected twice}
        \PYG{k}{if} \PYG{n}{prob} \PYG{o+ow}{is} \PYG{k+kc}{None}\PYG{p}{:}
            \PYG{n}{rand\PYGZus{}feat} \PYG{o}{=} \PYG{n}{np}\PYG{o}{.}\PYG{n}{random}\PYG{o}{.}\PYG{n}{choice}\PYG{p}{(}\PYG{n+nb}{range}\PYG{p}{(}\PYG{n}{data}\PYG{o}{.}\PYG{n}{shape}\PYG{p}{[}\PYG{l+m+mi}{1}\PYG{p}{]} \PYG{o}{\PYGZhy{}} \PYG{l+m+mi}{1}\PYG{p}{)}\PYG{p}{,} \PYG{n}{size}\PYG{o}{=}\PYG{n+nb}{int}\PYG{p}{(}\PYG{n}{np}\PYG{o}{.}\PYG{n}{random}\PYG{o}{.}\PYG{n}{choice}\PYG{p}{(}\PYG{n+nb}{range}\PYG{p}{(}\PYG{n+nb}{len}\PYG{p}{(}\PYG{n}{data}\PYG{p}{[}\PYG{l+m+mi}{0}\PYG{p}{]}\PYG{p}{)} \PYG{o}{\PYGZhy{}} \PYG{l+m+mi}{1}\PYG{p}{)}\PYG{p}{)} \PYG{o}{+} \PYG{l+m+mi}{1}\PYG{p}{)}\PYG{p}{,} \PYG{n}{replace}\PYG{o}{=}\PYG{k+kc}{False}\PYG{p}{,} \PYG{n}{p}\PYG{o}{=}\PYG{k+kc}{None}\PYG{p}{)}  \PYG{c+c1}{\PYGZsh{} size must be \PYGZlt{}= nonzero values of p, otherwise one feature gets selected twice}

        \PYG{n}{rand\PYGZus{}feat} \PYG{o}{=} \PYG{n}{np}\PYG{o}{.}\PYG{n}{sort}\PYG{p}{(}\PYG{n}{rand\PYGZus{}feat}\PYG{p}{)}  \PYG{c+c1}{\PYGZsh{} sort ascending}
        \PYG{n}{mask\PYGZus{}sub\PYGZus{}features}\PYG{p}{[}\PYG{n}{rand\PYGZus{}feat}\PYG{p}{]} \PYG{o}{=} \PYG{k+kc}{True}
        \PYG{c+c1}{\PYGZsh{} print \PYGZdq{}current feature set: \PYGZdq{} + str(mask\PYGZus{}sub\PYGZus{}features) \PYGZsh{} Debugging Line}

        \PYG{c+c1}{\PYGZsh{} Predict the new feature set}
        \PYG{n}{predictions} \PYG{o}{=} \PYG{n}{np}\PYG{o}{.}\PYG{n}{zeros}\PYG{p}{(}\PYG{n+nb}{len}\PYG{p}{(}\PYG{n}{trees}\PYG{p}{)}\PYG{p}{)}  \PYG{c+c1}{\PYGZsh{} Prelocate Memory}
        \PYG{n}{i} \PYG{o}{=} \PYG{l+m+mi}{0}  \PYG{c+c1}{\PYGZsh{} set counter for going through all trees}
        \PYG{c+c1}{\PYGZsh{} classify the randomly chosen feature sets in each tree}
        \PYG{k}{for} \PYG{n}{tree} \PYG{o+ow}{in} \PYG{n}{trees}\PYG{p}{:}
            \PYG{n}{predictions}\PYG{p}{[}\PYG{n}{i}\PYG{p}{]} \PYG{o}{=} \PYG{n}{classify}\PYG{p}{(}\PYG{n}{mask\PYGZus{}sub\PYGZus{}features}\PYG{p}{,} \PYG{n}{tree}\PYG{p}{)}\PYG{o}{.}\PYG{n}{keys}\PYG{p}{(}\PYG{p}{)}\PYG{p}{[}\PYG{l+m+mi}{0}\PYG{p}{]}
            \PYG{n}{i} \PYG{o}{+}\PYG{o}{=} \PYG{l+m+mi}{1}
        \PYG{c+c1}{\PYGZsh{} print \PYGZdq{}predictions: \PYGZdq{} + str(predictions) \PYGZsh{} Debugging Line}
        \PYG{c+c1}{\PYGZsh{} print \PYGZdq{}best\PYGZus{}mean = \PYGZdq{} + str(best\PYGZus{}mean) \PYGZsh{} Debugging Line}
        \PYG{c+c1}{\PYGZsh{} calculate mean an std for all predictions in a tree}
        \PYG{n}{mean}\PYG{p}{[}\PYG{n}{x}\PYG{p}{]} \PYG{o}{=} \PYG{n}{np}\PYG{o}{.}\PYG{n}{mean}\PYG{p}{(}\PYG{n}{predictions}\PYG{p}{)}
        \PYG{n}{var}\PYG{p}{[}\PYG{n}{x}\PYG{p}{]} \PYG{o}{=} \PYG{n}{np}\PYG{o}{.}\PYG{n}{var}\PYG{p}{(}\PYG{n}{predictions}\PYG{p}{)} \PYG{o}{/} \PYG{n+nb}{abs}\PYG{p}{(}\PYG{n}{mean}\PYG{p}{[}\PYG{n}{x}\PYG{p}{]}\PYG{p}{)}
        \PYG{c+c1}{\PYGZsh{} check if current mean and var are better than best mean and var}
        \PYG{c+c1}{\PYGZsh{} calculation: best\PYGZus{}mean = 1.0*mean + 0.1*var and vice versa}
        \PYG{k}{if} \PYG{n}{best\PYGZus{}mean} \PYG{o}{==} \PYG{p}{[}\PYG{l+m+mi}{0}\PYG{p}{]} \PYG{o+ow}{or} \PYG{n}{mean}\PYG{p}{[}\PYG{n}{x}\PYG{p}{]} \PYG{o}{+} \PYG{n}{var}\PYG{p}{[}\PYG{n}{x}\PYG{p}{]} \PYG{o}{*} \PYG{l+m+mf}{0.75} \PYG{o}{\PYGZgt{}} \PYG{n}{best\PYGZus{}mean}\PYG{p}{:}
            \PYG{n}{best\PYGZus{}mean} \PYG{o}{=} \PYG{n}{mean}\PYG{p}{[}\PYG{n}{x}\PYG{p}{]} \PYG{o}{+} \PYG{n}{var}\PYG{p}{[}\PYG{n}{x}\PYG{p}{]} \PYG{o}{*} \PYG{l+m+mf}{0.75}

            \PYG{c+c1}{\PYGZsh{} print \PYGZdq{}best\PYGZus{}mean updated: \PYGZdq{} + str(best\PYGZus{}mean) \PYGZsh{} Debugging Line}
            \PYG{n}{best\PYGZus{}featureset\PYGZus{}mean} \PYG{o}{=} \PYG{n}{mask\PYGZus{}sub\PYGZus{}features}
            \PYG{c+c1}{\PYGZsh{} print \PYGZdq{}best\PYGZus{}featureset\PYGZus{}mean = \PYGZdq{} + str(best\PYGZus{}featureset\PYGZus{}mean) \PYGZsh{} Debugging Line}
        \PYG{k}{if} \PYG{n}{best\PYGZus{}var} \PYG{o}{==} \PYG{p}{[}\PYG{l+m+mi}{0}\PYG{p}{]} \PYG{o+ow}{or} \PYG{n}{var}\PYG{p}{[}\PYG{n}{x}\PYG{p}{]} \PYG{o}{+} \PYG{n}{mean}\PYG{p}{[}\PYG{n}{x}\PYG{p}{]} \PYG{o}{*} \PYG{l+m+mf}{0.1} \PYG{o}{\PYGZgt{}} \PYG{n}{best\PYGZus{}var}\PYG{p}{:}
            \PYG{n}{best\PYGZus{}var} \PYG{o}{=} \PYG{n}{var}\PYG{p}{[}\PYG{n}{x}\PYG{p}{]} \PYG{o}{+} \PYG{n}{mean}\PYG{p}{[}\PYG{n}{x}\PYG{p}{]} \PYG{o}{*} \PYG{l+m+mf}{0.1}
            \PYG{c+c1}{\PYGZsh{} print \PYGZdq{}best\PYGZus{}var updated: \PYGZdq{} + str(best\PYGZus{}var) \PYGZsh{} Debugging Line}
            \PYG{n}{best\PYGZus{}featureset\PYGZus{}var} \PYG{o}{=} \PYG{n}{mask\PYGZus{}sub\PYGZus{}features}
            \PYG{c+c1}{\PYGZsh{} print \PYGZdq{}best\PYGZus{}featureset\PYGZus{}var = \PYGZdq{} + str(best\PYGZus{}featureset\PYGZus{}var) \PYGZsh{} Debugging Line}
    \PYG{c+c1}{\PYGZsh{} print \PYGZdq{}best mean for current forest: \PYGZdq{} + str(best\PYGZus{}mean) \PYGZsh{} Debugging Line}
    \PYG{c+c1}{\PYGZsh{} print \PYGZdq{}best feature set for best mean: \PYGZdq{} + str(best\PYGZus{}featureset\PYGZus{}mean) \PYGZsh{} Debugging Line}
    \PYG{c+c1}{\PYGZsh{} print \PYGZdq{}best var for current forest: \PYGZdq{} + str(best\PYGZus{}var) \PYGZsh{} Debugging Line}
    \PYG{c+c1}{\PYGZsh{} print \PYGZdq{}best feature set for best var\PYGZdq{} + str(best\PYGZus{}featureset\PYGZus{}var) \PYGZsh{} Debugging Line}
    \PYG{k}{return} \PYG{n}{best\PYGZus{}mean}\PYG{p}{,} \PYG{n}{best\PYGZus{}var}\PYG{p}{,} \PYG{n}{best\PYGZus{}featureset\PYGZus{}mean}\PYG{p}{,} \PYG{n}{best\PYGZus{}featureset\PYGZus{}var}


\PYG{c+c1}{\PYGZsh{} based on the probabilities of each feature in past Forests, a new current\PYGZus{}prob is calculated that takes into}
\PYG{c+c1}{\PYGZsh{} account the mean and the gradient of the prior feature importances}
\PYG{k}{def} \PYG{n+nf}{update\PYGZus{}prob}\PYG{p}{(}\PYG{n}{Probability}\PYG{p}{,} \PYG{n}{i}\PYG{p}{,} \PYG{n}{weight\PYGZus{}mean}\PYG{p}{,} \PYG{n}{weight\PYGZus{}gradient}\PYG{p}{,} \PYG{n}{multiplier}\PYG{p}{,} \PYG{n}{seen\PYGZus{}forests}\PYG{p}{)}\PYG{p}{:}
    \PYG{l+s+sd}{\PYGZdq{}\PYGZdq{}\PYGZdq{}Calculates the current Importance / Probability of the single features}

\PYG{l+s+sd}{    Based on the probabilities of each feature in past Forests a new current\PYGZus{}prob is calculated that takes into}
\PYG{l+s+sd}{    account the mean and the gradient of the prior feature importances.}


\PYG{l+s+sd}{    Arguments:}
\PYG{l+s+sd}{        * Probability \PYGZob{}numpy array\PYGZcb{} \PYGZhy{}\PYGZhy{} contains Importances of single features for all past Random Forests}
\PYG{l+s+sd}{        * i \PYGZob{}integer\PYGZcb{} \PYGZhy{}\PYGZhy{} number of current Forest}
\PYG{l+s+sd}{        * weight\PYGZus{}mean \PYGZob{}float\PYGZcb{} \PYGZhy{}\PYGZhy{} weight of the mean in calculating resulting probability}
\PYG{l+s+sd}{        * weight\PYGZus{}gradient \PYGZob{}float\PYGZcb{} \PYGZhy{}\PYGZhy{} weight of the var in calculating resulting probability}
\PYG{l+s+sd}{        * multiplier \PYGZob{}float\PYGZcb{} \PYGZhy{}\PYGZhy{} exponent for amplifying probabilities}
\PYG{l+s+sd}{        * seen\PYGZus{}forests \PYGZob{}integer\PYGZcb{} \PYGZhy{}\PYGZhy{} number of before built forest that are considered}

\PYG{l+s+sd}{    Returns:}
\PYG{l+s+sd}{        prob\PYGZus{}current \PYGZhy{}\PYGZhy{} list of floats representing the calculated aggregation of recent feature importances}
\PYG{l+s+sd}{    \PYGZdq{}\PYGZdq{}\PYGZdq{}}
    \PYG{c+c1}{\PYGZsh{} print \PYGZdq{}Probability: \PYGZdq{} + str(Probability[0:i + 1]) \PYGZsh{} Debugging Line}

    \PYG{c+c1}{\PYGZsh{} if only one or two calculations of prob has been done so far, leave prob empty}
    \PYG{c+c1}{\PYGZsh{} (np.gradient needs 3 points and 3 random Forests to provide better statistical insurance than only 1 Random Forest)}
    \PYG{k}{if} \PYG{n}{i} \PYG{o}{\PYGZlt{}}\PYG{o}{=} \PYG{l+m+mi}{1}\PYG{p}{:}
        \PYG{n}{prob\PYGZus{}current} \PYG{o}{=} \PYG{k+kc}{None}
    \PYG{k}{else}\PYG{p}{:}
        \PYG{c+c1}{\PYGZsh{} gradients contains the current gradient for each feature}
        \PYG{c+c1}{\PYGZsh{} map: function list is applied to all zip(transposed(a)) (without list: zip generatets tuple instead of list)}
        \PYG{k}{if} \PYG{n}{i} \PYG{o}{\PYGZlt{}} \PYG{n}{seen\PYGZus{}forests}\PYG{p}{:}
            \PYG{n}{gradients} \PYG{o}{=} \PYG{n}{np}\PYG{o}{.}\PYG{n}{gradient}\PYG{p}{(}\PYG{n+nb}{map}\PYG{p}{(}\PYG{n+nb}{list}\PYG{p}{,} \PYG{n+nb}{zip}\PYG{p}{(}\PYG{o}{*}\PYG{n}{Probability}\PYG{p}{[}\PYG{l+m+mi}{0}\PYG{p}{:}\PYG{n}{i} \PYG{o}{+} \PYG{l+m+mi}{1}\PYG{p}{]}\PYG{p}{)}\PYG{p}{)}\PYG{p}{,} \PYG{n}{axis}\PYG{o}{=}\PYG{l+m+mi}{1}\PYG{p}{)}
            \PYG{n}{mean} \PYG{o}{=} \PYG{n}{np}\PYG{o}{.}\PYG{n}{mean}\PYG{p}{(}\PYG{n+nb}{map}\PYG{p}{(}\PYG{n+nb}{list}\PYG{p}{,} \PYG{n+nb}{zip}\PYG{p}{(}\PYG{o}{*}\PYG{n}{Probability}\PYG{p}{[}\PYG{l+m+mi}{0}\PYG{p}{:}\PYG{n}{i} \PYG{o}{+} \PYG{l+m+mi}{1}\PYG{p}{]}\PYG{p}{)}\PYG{p}{)}\PYG{p}{,} \PYG{n}{axis}\PYG{o}{=}\PYG{l+m+mi}{1}\PYG{p}{)}
        \PYG{c+c1}{\PYGZsh{} only the last seen\PYGZus{}forests values will be taken into account}
        \PYG{k}{else}\PYG{p}{:}
            \PYG{c+c1}{\PYGZsh{} print \PYGZdq{}consider only last \PYGZdq{} + str(seen\PYGZus{}forests) + \PYGZdq{} forests for calculation of probability\PYGZdq{}}
            \PYG{n}{gradients} \PYG{o}{=} \PYG{n}{np}\PYG{o}{.}\PYG{n}{gradient}\PYG{p}{(}\PYG{n+nb}{map}\PYG{p}{(}\PYG{n+nb}{list}\PYG{p}{,} \PYG{n+nb}{zip}\PYG{p}{(}\PYG{o}{*}\PYG{n}{Probability}\PYG{p}{[}\PYG{n}{i} \PYG{o}{\PYGZhy{}} \PYG{n}{seen\PYGZus{}forests}\PYG{p}{:}\PYG{n}{i} \PYG{o}{+} \PYG{l+m+mi}{1}\PYG{p}{]}\PYG{p}{)}\PYG{p}{)}\PYG{p}{,} \PYG{n}{axis}\PYG{o}{=}\PYG{l+m+mi}{1}\PYG{p}{)}
            \PYG{n}{mean} \PYG{o}{=} \PYG{n}{np}\PYG{o}{.}\PYG{n}{mean}\PYG{p}{(}\PYG{n+nb}{map}\PYG{p}{(}\PYG{n+nb}{list}\PYG{p}{,} \PYG{n+nb}{zip}\PYG{p}{(}\PYG{o}{*}\PYG{n}{Probability}\PYG{p}{[}\PYG{n}{i} \PYG{o}{\PYGZhy{}} \PYG{n}{seen\PYGZus{}forests}\PYG{p}{:}\PYG{n}{i} \PYG{o}{+} \PYG{l+m+mi}{1}\PYG{p}{]}\PYG{p}{)}\PYG{p}{)}\PYG{p}{,} \PYG{n}{axis}\PYG{o}{=}\PYG{l+m+mi}{1}\PYG{p}{)}

        \PYG{c+c1}{\PYGZsh{} print \PYGZdq{}gradients: \PYGZdq{} + str(gradients) \PYGZsh{} Debugging Line}

        \PYG{c+c1}{\PYGZsh{} calculate the mean of the gradient for each feature}
        \PYG{n}{gradients\PYGZus{}mean} \PYG{o}{=} \PYG{n+nb}{map}\PYG{p}{(}\PYG{n}{np}\PYG{o}{.}\PYG{n}{mean}\PYG{p}{,} \PYG{n}{gradients}\PYG{p}{)}
        \PYG{c+c1}{\PYGZsh{} print \PYGZdq{}gradients\PYGZus{}mean: \PYGZdq{} + str(gradients\PYGZus{}mean) \PYGZsh{} Debugging Line}

        \PYG{c+c1}{\PYGZsh{} calculate the norm of the gradient for each feature}
        \PYG{n}{gradients\PYGZus{}norm} \PYG{o}{=} \PYG{n+nb}{map}\PYG{p}{(}\PYG{n}{np}\PYG{o}{.}\PYG{n}{linalg}\PYG{o}{.}\PYG{n}{norm}\PYG{p}{,} \PYG{n}{gradients}\PYG{p}{)}
        \PYG{c+c1}{\PYGZsh{} print \PYGZdq{}gradients\PYGZus{}norm: \PYGZdq{} + str(gradients\PYGZus{}norm) \PYGZsh{} Debugging Line}

        \PYG{c+c1}{\PYGZsh{} divide the mean by the norm(=length)}
        \PYG{c+c1}{\PYGZsh{} (to punish strongly fluctuating values and to reward values that change only slightly over time)}
        \PYG{n}{gradients} \PYG{o}{=} \PYG{n}{np}\PYG{o}{.}\PYG{n}{nan\PYGZus{}to\PYGZus{}num}\PYG{p}{(}\PYG{n}{np}\PYG{o}{.}\PYG{n}{divide}\PYG{p}{(}\PYG{n}{gradients\PYGZus{}mean}\PYG{p}{,} \PYG{n}{gradients\PYGZus{}norm}\PYG{p}{)}\PYG{p}{)}  \PYG{c+c1}{\PYGZsh{} nan\PYGZus{}to\PYGZus{}num: because division by zero leaves NaN}
        \PYG{c+c1}{\PYGZsh{} print \PYGZdq{}gradients mean / norm: \PYGZdq{} + str(gradients) \PYGZsh{} Debugging Line}

        \PYG{c+c1}{\PYGZsh{} scale values}
        \PYG{n}{min\PYGZus{}max\PYGZus{}scaler} \PYG{o}{=} \PYG{n}{preprocessing}\PYG{o}{.}\PYG{n}{MinMaxScaler}\PYG{p}{(}\PYG{n}{feature\PYGZus{}range}\PYG{o}{=}\PYG{p}{(}\PYG{l+m+mi}{1}\PYG{p}{,} \PYG{l+m+mi}{100}\PYG{p}{)}\PYG{p}{)}
        \PYG{n}{gradients} \PYG{o}{=} \PYG{n}{min\PYGZus{}max\PYGZus{}scaler}\PYG{o}{.}\PYG{n}{fit\PYGZus{}transform}\PYG{p}{(}\PYG{n}{gradients}\PYG{o}{.}\PYG{n}{reshape}\PYG{p}{(}\PYG{o}{\PYGZhy{}}\PYG{l+m+mi}{1}\PYG{p}{,} \PYG{l+m+mi}{1}\PYG{p}{)}\PYG{p}{)}  \PYG{c+c1}{\PYGZsh{} reshape: otherwise deprecation warning}
        \PYG{n}{mean} \PYG{o}{=} \PYG{n}{min\PYGZus{}max\PYGZus{}scaler}\PYG{o}{.}\PYG{n}{fit\PYGZus{}transform}\PYG{p}{(}\PYG{n}{mean}\PYG{o}{.}\PYG{n}{reshape}\PYG{p}{(}\PYG{o}{\PYGZhy{}}\PYG{l+m+mi}{1}\PYG{p}{,} \PYG{l+m+mi}{1}\PYG{p}{)}\PYG{p}{)}  \PYG{c+c1}{\PYGZsh{} reshape: otherwise deprecation warning}
        \PYG{c+c1}{\PYGZsh{} print \PYGZdq{}gradients rescaled: \PYGZdq{} + str(gradients) \PYGZsh{} Debugging Line}
        \PYG{c+c1}{\PYGZsh{} print \PYGZdq{}mean rescaled: \PYGZdq{} + str(mean) \PYGZsh{} Debugging Line}

        \PYG{c+c1}{\PYGZsh{} calculate new probability for selection of new feature sets}
        \PYG{c+c1}{\PYGZsh{} weight\PYGZus{}mean, weight\PYGZus{}gradient and multiplier are hyperparameters}
        \PYG{n}{prob\PYGZus{}current} \PYG{o}{=} \PYG{p}{(}\PYG{n}{mean} \PYG{o}{*} \PYG{n}{weight\PYGZus{}mean} \PYG{o}{+} \PYG{n}{gradients} \PYG{o}{*} \PYG{n}{weight\PYGZus{}gradient}\PYG{p}{)}\PYG{o}{*}\PYG{o}{*}\PYG{n}{multiplier}
        \PYG{c+c1}{\PYGZsh{} print \PYGZdq{}prob\PYGZus{}current: \PYGZdq{} + str(prob\PYGZus{}current) \PYGZsh{} Debugging Line}
        \PYG{c+c1}{\PYGZsh{} print \PYGZdq{}gradients + mean: \PYGZdq{} + str(gradients) \PYGZsh{} Debugging Line}

        \PYG{c+c1}{\PYGZsh{} express values as percentage (because sum(prob) must equal 1)}
        \PYG{n}{prob\PYGZus{}current} \PYG{o}{=} \PYG{n}{np}\PYG{o}{.}\PYG{n}{divide}\PYG{p}{(}\PYG{n}{prob\PYGZus{}current}\PYG{p}{,} \PYG{n}{np}\PYG{o}{.}\PYG{n}{sum}\PYG{p}{(}\PYG{n}{prob\PYGZus{}current}\PYG{p}{)}\PYG{p}{)}
        \PYG{c+c1}{\PYGZsh{} print \PYGZdq{}gradients percent: \PYGZdq{} + str(gradients)}
        \PYG{n}{prob\PYGZus{}current} \PYG{o}{=} \PYG{n}{np}\PYG{o}{.}\PYG{n}{array}\PYG{p}{(}\PYG{p}{[}\PYG{n}{item} \PYG{k}{for} \PYG{n}{sublist} \PYG{o+ow}{in} \PYG{n}{prob\PYGZus{}current} \PYG{k}{for} \PYG{n}{item} \PYG{o+ow}{in} \PYG{n}{sublist}\PYG{p}{]}\PYG{p}{)}  \PYG{c+c1}{\PYGZsh{} convert nested list into usual list}
        \PYG{c+c1}{\PYGZsh{} print \PYGZdq{}prob\PYGZus{}current: \PYGZdq{} + str(prob\PYGZus{}current) \PYGZsh{} Debugging Line}

        \PYG{c+c1}{\PYGZsh{} in the last run print out the gradients}
        \PYG{k}{if} \PYG{n}{i} \PYG{o}{+} \PYG{l+m+mi}{1} \PYG{o}{==} \PYG{n+nb}{len}\PYG{p}{(}\PYG{n}{Probability}\PYG{p}{)}\PYG{p}{:}
            \PYG{n+nb}{print} \PYG{l+s+s2}{\PYGZdq{}}\PYG{l+s+s2}{ }\PYG{l+s+s2}{\PYGZdq{}}
            \PYG{c+c1}{\PYGZsh{} print \PYGZdq{}gradients mean: \PYGZdq{} + str(gradients\PYGZus{}mean) \PYGZsh{} Debugging Line}
            \PYG{c+c1}{\PYGZsh{} print \PYGZdq{} \PYGZdq{} \PYGZsh{} Debugging Line}
            \PYG{c+c1}{\PYGZsh{} print \PYGZdq{}prob\PYGZus{}current: \PYGZdq{} + str(prob\PYGZus{}current) \PYGZsh{} Debugging Line}
    \PYG{k}{return} \PYG{n}{prob\PYGZus{}current}


\PYG{k}{def} \PYG{n+nf}{update\PYGZus{}database}\PYG{p}{(}\PYG{n}{X}\PYG{p}{,} \PYG{n}{y}\PYG{p}{,} \PYG{n}{data}\PYG{p}{,} \PYG{n}{mask\PYGZus{}best\PYGZus{}featureset}\PYG{p}{,} \PYG{n}{X\PYGZus{}test}\PYG{p}{,} \PYG{n}{y\PYGZus{}test}\PYG{p}{)}\PYG{p}{:}
    \PYG{l+s+sd}{\PYGZdq{}\PYGZdq{}\PYGZdq{}Appends newly tested feature sets and their result to the already calculated feature sets}

\PYG{l+s+sd}{    Arguments:}
\PYG{l+s+sd}{        * X \PYGZob{}numpy array\PYGZcb{} \PYGZhy{}\PYGZhy{} X rat data sets}
\PYG{l+s+sd}{        * y \PYGZob{}numpy array\PYGZcb{} \PYGZhy{}\PYGZhy{} y raw data sets}
\PYG{l+s+sd}{        * data \PYGZob{}[type]\PYGZcb{} \PYGZhy{}\PYGZhy{} data set the Forest is built upon}
\PYG{l+s+sd}{        * mask\PYGZus{}best\PYGZus{}featureset \PYGZob{}bool\PYGZcb{} \PYGZhy{}\PYGZhy{} feature set (1: feature contained, 0: feature not contained)}
\PYG{l+s+sd}{        * X\PYGZus{}test \PYGZob{}numpy array\PYGZcb{} \PYGZhy{}\PYGZhy{} test data set}
\PYG{l+s+sd}{        * y\PYGZus{}test \PYGZob{}numpy array\PYGZcb{} \PYGZhy{}\PYGZhy{} test data set}

\PYG{l+s+sd}{    Returns:}
\PYG{l+s+sd}{        data \PYGZhy{}\PYGZhy{} updated data base}
\PYG{l+s+sd}{    \PYGZdq{}\PYGZdq{}\PYGZdq{}}
    \PYG{c+c1}{\PYGZsh{} create the best mean feature set}
    \PYG{n}{X\PYGZus{}sub} \PYG{o}{=} \PYG{n}{X}\PYG{p}{[}\PYG{p}{:}\PYG{p}{,} \PYG{n}{mask\PYGZus{}best\PYGZus{}featureset}\PYG{p}{]}
    \PYG{c+c1}{\PYGZsh{} compute the corresponding y values}
    \PYG{n}{y\PYGZus{}new} \PYG{o}{=} \PYG{n}{compute}\PYG{p}{(}\PYG{n}{X\PYGZus{}sub}\PYG{p}{,} \PYG{n}{y}\PYG{p}{,} \PYG{n}{mask\PYGZus{}best\PYGZus{}featureset}\PYG{p}{,} \PYG{n}{X\PYGZus{}test}\PYG{p}{,} \PYG{n}{y\PYGZus{}test}\PYG{p}{)}
    \PYG{c+c1}{\PYGZsh{} combine feature set and new y (result)}
    \PYG{n}{new\PYGZus{}dataset} \PYG{o}{=} \PYG{n}{np}\PYG{o}{.}\PYG{n}{append}\PYG{p}{(}\PYG{n}{mask\PYGZus{}best\PYGZus{}featureset}\PYG{p}{,} \PYG{n}{y\PYGZus{}new}\PYG{p}{)}
    \PYG{c+c1}{\PYGZsh{} print \PYGZdq{}new\PYGZus{}dataset\PYGZus{}mean: \PYGZdq{} + str(new\PYGZus{}dataset\PYGZus{}mean) \PYGZsh{} Debugging Line}
    \PYG{c+c1}{\PYGZsh{} print new\PYGZus{}dataset\PYGZus{}mean.shape \PYGZsh{} Debugging Line}

    \PYG{c+c1}{\PYGZsh{} append new feature sets and according result to dataset}
    \PYG{n}{data} \PYG{o}{=} \PYG{n}{np}\PYG{o}{.}\PYG{n}{append}\PYG{p}{(}\PYG{n}{data}\PYG{p}{,} \PYG{p}{[}\PYG{n}{new\PYGZus{}dataset}\PYG{p}{]}\PYG{p}{,} \PYG{n}{axis}\PYG{o}{=}\PYG{l+m+mi}{0}\PYG{p}{)}
    \PYG{k}{return} \PYG{n}{data}


\PYG{c+c1}{\PYGZsh{} This is the main part of the program which uses the above made definitions}
\PYG{k}{def} \PYG{n+nf}{main\PYGZus{}loop}\PYG{p}{(}\PYG{n}{n\PYGZus{}start}\PYG{p}{,} \PYG{n}{pruning}\PYG{p}{,} \PYG{n}{min\PYGZus{}data}\PYG{p}{,} \PYG{n}{n\PYGZus{}forests}\PYG{p}{,} \PYG{n}{n\PYGZus{}trees}\PYG{p}{,} \PYG{n}{n\PYGZus{}configs\PYGZus{}biased}\PYG{p}{,} \PYG{n}{n\PYGZus{}configs\PYGZus{}unbiased}\PYG{p}{,} \PYG{n}{multiplier\PYGZus{}stepup}\PYG{p}{,} \PYG{n}{seen\PYGZus{}forests}\PYG{p}{,}
              \PYG{n}{weight\PYGZus{}mean}\PYG{p}{,} \PYG{n}{weight\PYGZus{}gradient}\PYG{p}{,} \PYG{n}{scoref}\PYG{p}{,} \PYG{n}{demo\PYGZus{}mode}\PYG{p}{,} \PYG{n}{plot\PYGZus{}enable}\PYG{p}{)}\PYG{p}{:}
    \PYG{l+s+sd}{\PYGZdq{}\PYGZdq{}\PYGZdq{}Load raw data and Generate database for Random Forest. Iteratively build and burn down new Random Forests, predict the performance of new feature sets and compute two new feature sets per round.}

\PYG{l+s+sd}{    Arguments:}

\PYG{l+s+sd}{        * n\PYGZus{}start \PYGZob{}int\PYGZcb{} \PYGZhy{}\PYGZhy{} number of runs before building first RF = number of data points in first RF; minimum = 4, default = 50}
\PYG{l+s+sd}{        * pruning \PYGZob{}float\PYGZcb{} \PYGZhy{}\PYGZhy{} if greater than zero, branches of a Decision Tree will be pruned proportional to pruning value; default = 0}
\PYG{l+s+sd}{        * min\PYGZus{}data \PYGZob{}float\PYGZcb{} \PYGZhy{}\PYGZhy{} minimum percentage of Datasets that is used in RF generation; default = 0.2}
\PYG{l+s+sd}{        * n\PYGZus{}forests \PYGZob{}int\PYGZcb{} \PYGZhy{}\PYGZhy{} number of forests; minimum=1;  default = 25}
\PYG{l+s+sd}{        * n\PYGZus{}trees \PYGZob{}int\PYGZcb{} \PYGZhy{}\PYGZhy{} \PYGZsh{} number of trees that stand in a forest; min = 3; default = number of features x 3 x}
\PYG{l+s+sd}{        * n\PYGZus{}configs\PYGZus{}biased \PYGZob{}int\PYGZcb{} \PYGZhy{}\PYGZhy{} \PYGZsh{} number of deliberately chosen feature sets that get predicted in each forest; default = n\PYGZus{}trees x 5}
\PYG{l+s+sd}{        * n\PYGZus{}configs\PYGZus{}unbiased \PYGZob{}int\PYGZcb{} \PYGZhy{}\PYGZhy{} \PYGZsh{} number of randomly chosen feature sets that get predicted in each forest; default = n\PYGZus{}configs\PYGZus{}biased x0.2}
\PYG{l+s+sd}{        * multiplier\PYGZus{}stepup \PYGZob{}float\PYGZcb{} \PYGZhy{}\PYGZhy{} \PYGZsh{} sets how aggressively the feature importance changes; default = 0.25}
\PYG{l+s+sd}{        * seen\PYGZus{}forests \PYGZob{}int\PYGZcb{} \PYGZhy{}\PYGZhy{} \PYGZsh{} number of recent forests that are taken into acount for generating probability of the chosen feature sets default = 4}
\PYG{l+s+sd}{        * weight\PYGZus{}mean \PYGZob{}float\PYGZcb{} \PYGZhy{}\PYGZhy{} \PYGZsh{} weight of the mean in calculating the new probability for selecting future feature sets; default = 0.2}
\PYG{l+s+sd}{        * weight\PYGZus{}gradient \PYGZob{}bool\PYGZcb{} \PYGZhy{}\PYGZhy{} \PYGZsh{} weight of the gradient in calculating the new probability for selecting future feature sets; default = 0.8}
\PYG{l+s+sd}{        * scoref \PYGZob{}function\PYGZcb{} \PYGZhy{}\PYGZhy{} \PYGZsh{} which scoring metric should be used in the Decision Tree (available: entropy and giniimpurity); default = entropy}
\PYG{l+s+sd}{        * demo\PYGZus{}mode bool \PYGZhy{}\PYGZhy{} \PYGZsh{} if true a comparison between the Random Forest driven Search and a random search is done}
\PYG{l+s+sd}{        * plot\PYGZus{}enable bool \PYGZhy{}\PYGZhy{} \PYGZsh{} decide if at the end a plot should be generated , only possible in demo mode}

\PYG{l+s+sd}{    \PYGZdq{}\PYGZdq{}\PYGZdq{}}
    \PYG{n+nb}{print} \PYG{l+s+s2}{\PYGZdq{}}\PYG{l+s+s2}{Starting script}\PYG{l+s+s2}{\PYGZdq{}}
    \PYG{c+c1}{\PYGZsh{} Generate Test Data}
    \PYG{n+nb}{print} \PYG{l+s+s2}{\PYGZdq{}}\PYG{l+s+s2}{Loading Raw Data}\PYG{l+s+s2}{\PYGZdq{}}
    \PYG{n}{X\PYGZus{}test}\PYG{p}{,} \PYG{n}{X}\PYG{p}{,} \PYG{n}{y\PYGZus{}test}\PYG{p}{,} \PYG{n}{y}\PYG{p}{,} \PYG{n}{n\PYGZus{}feat} \PYG{o}{=} \PYG{n}{import\PYGZus{}data}\PYG{p}{(}\PYG{p}{)}
    \PYG{c+c1}{\PYGZsh{} set default hyperparameters}
    \PYG{n+nb}{print} \PYG{l+s+s2}{\PYGZdq{}}\PYG{l+s+s2}{Setting Hyperparameters}\PYG{l+s+s2}{\PYGZdq{}}
    \PYG{k}{if} \PYG{n}{n\PYGZus{}trees} \PYG{o+ow}{is} \PYG{l+s+s1}{\PYGZsq{}}\PYG{l+s+s1}{default}\PYG{l+s+s1}{\PYGZsq{}}\PYG{p}{:}
        \PYG{n}{n\PYGZus{}trees} \PYG{o}{=} \PYG{n}{n\PYGZus{}feat} \PYG{o}{*} \PYG{l+m+mi}{3}
    \PYG{k}{if} \PYG{n}{seen\PYGZus{}forests} \PYG{o+ow}{is} \PYG{l+s+s1}{\PYGZsq{}}\PYG{l+s+s1}{default}\PYG{l+s+s1}{\PYGZsq{}}\PYG{p}{:}
        \PYG{n}{seen\PYGZus{}forests} \PYG{o}{=} \PYG{l+m+mi}{4}
    \PYG{k}{if} \PYG{n}{n\PYGZus{}configs\PYGZus{}biased} \PYG{o+ow}{is} \PYG{l+s+s1}{\PYGZsq{}}\PYG{l+s+s1}{default}\PYG{l+s+s1}{\PYGZsq{}}\PYG{p}{:}
        \PYG{n}{n\PYGZus{}configs\PYGZus{}biased} \PYG{o}{=} \PYG{n}{n\PYGZus{}trees} \PYG{o}{*} \PYG{l+m+mi}{5}  \PYG{c+c1}{\PYGZsh{} number of biased configs that get predicted in each forest}
    \PYG{k}{if} \PYG{n}{n\PYGZus{}configs\PYGZus{}unbiased} \PYG{o+ow}{is} \PYG{l+s+s1}{\PYGZsq{}}\PYG{l+s+s1}{default}\PYG{l+s+s1}{\PYGZsq{}}\PYG{p}{:}
        \PYG{n}{n\PYGZus{}configs\PYGZus{}unbiased} \PYG{o}{=} \PYG{n+nb}{int}\PYG{p}{(}\PYG{n+nb}{round}\PYG{p}{(}\PYG{n}{n\PYGZus{}configs\PYGZus{}biased} \PYG{o}{*} \PYG{l+m+mf}{0.2}\PYG{p}{)}\PYG{p}{)}  \PYG{c+c1}{\PYGZsh{} number of unbiased configs that get predicted in each forest}
    \PYG{k}{if} \PYG{n}{multiplier\PYGZus{}stepup} \PYG{o+ow}{is} \PYG{l+s+s1}{\PYGZsq{}}\PYG{l+s+s1}{default}\PYG{l+s+s1}{\PYGZsq{}}\PYG{p}{:}
        \PYG{n}{multiplier\PYGZus{}stepup} \PYG{o}{=} \PYG{l+m+mf}{0.01}
    \PYG{k}{if} \PYG{n}{weight\PYGZus{}mean} \PYG{o+ow}{is} \PYG{l+s+s1}{\PYGZsq{}}\PYG{l+s+s1}{default}\PYG{l+s+s1}{\PYGZsq{}}\PYG{p}{:}
        \PYG{n}{weight\PYGZus{}mean} \PYG{o}{=} \PYG{l+m+mf}{0.1}
    \PYG{k}{if} \PYG{n}{weight\PYGZus{}gradient} \PYG{o+ow}{is} \PYG{l+s+s1}{\PYGZsq{}}\PYG{l+s+s1}{default}\PYG{l+s+s1}{\PYGZsq{}}\PYG{p}{:}
        \PYG{n}{weight\PYGZus{}gradient} \PYG{o}{=} \PYG{l+m+mf}{0.9}
    \PYG{k}{if} \PYG{n}{scoref} \PYG{o+ow}{is} \PYG{l+s+s1}{\PYGZsq{}}\PYG{l+s+s1}{default}\PYG{l+s+s1}{\PYGZsq{}}\PYG{p}{:}
        \PYG{n}{scoref} \PYG{o}{=} \PYG{n}{entropy}
    \PYG{k}{elif} \PYG{n}{scoref} \PYG{o+ow}{is} \PYG{l+s+s1}{\PYGZsq{}}\PYG{l+s+s1}{entropy}\PYG{l+s+s1}{\PYGZsq{}}\PYG{p}{:}
        \PYG{n}{scoref} \PYG{o}{=} \PYG{n}{entropy}
    \PYG{k}{elif} \PYG{n}{scoref} \PYG{o+ow}{is} \PYG{l+s+s1}{\PYGZsq{}}\PYG{l+s+s1}{giniimpurity}\PYG{l+s+s1}{\PYGZsq{}}\PYG{p}{:}
        \PYG{n}{scoref} \PYG{o}{=} \PYG{n}{giniimpurity}
    \PYG{k}{elif} \PYG{n}{scoref} \PYG{o+ow}{is} \PYG{l+s+s1}{\PYGZsq{}}\PYG{l+s+s1}{variance}\PYG{l+s+s1}{\PYGZsq{}}\PYG{p}{:}
        \PYG{n}{scoref} \PYG{o}{=} \PYG{n}{variance}
    \PYG{k}{if} \PYG{n}{pruning} \PYG{o}{\PYGZgt{}} \PYG{l+m+mi}{0}\PYG{p}{:}
        \PYG{n+nb}{print} \PYG{l+s+s2}{\PYGZdq{}}\PYG{l+s+s2}{Pruning enabled}\PYG{l+s+s2}{\PYGZdq{}}

    \PYG{n}{multiplier} \PYG{o}{=} \PYG{l+m+mi}{1}  \PYG{c+c1}{\PYGZsh{} initialize value for multiplier}

    \PYG{n}{Probability} \PYG{o}{=} \PYG{n}{np}\PYG{o}{.}\PYG{n}{zeros}\PYG{p}{(}\PYG{n}{shape}\PYG{o}{=}\PYG{p}{[}\PYG{n}{n\PYGZus{}forests}\PYG{p}{,} \PYG{n}{n\PYGZus{}feat}\PYG{p}{]}\PYG{p}{)}  \PYG{c+c1}{\PYGZsh{} Prelocate Memory: probabilites for selecting features in svm}

    \PYG{c+c1}{\PYGZsh{} Generate database for RF}
    \PYG{n+nb}{print} \PYG{l+s+s2}{\PYGZdq{}}\PYG{l+s+s2}{Generate Data Base for Random Forest}\PYG{l+s+s2}{\PYGZdq{}}
    \PYG{n}{data} \PYG{o}{=} \PYG{n}{gen\PYGZus{}database}\PYG{p}{(}\PYG{n}{n\PYGZus{}start}\PYG{p}{,} \PYG{n}{X}\PYG{p}{,} \PYG{n}{y}\PYG{p}{,} \PYG{n}{X\PYGZus{}test}\PYG{p}{,} \PYG{n}{y\PYGZus{}test}\PYG{p}{)}

    \PYG{k}{if} \PYG{n}{demo\PYGZus{}mode}\PYG{p}{:}
        \PYG{n}{data\PYGZus{}start} \PYG{o}{=} \PYG{n}{data}  \PYG{c+c1}{\PYGZsh{} save starting data for later comparison with random feature set selection}
    \PYG{c+c1}{\PYGZsh{} print \PYGZdq{}len(data): \PYGZdq{} + str(len(data)) \PYGZsh{} Debugging Line}

    \PYG{c+c1}{\PYGZsh{} \PYGZsh{}\PYGZsh{}\PYGZsh{} Start of ForestFire \PYGZsh{}\PYGZsh{}\PYGZsh{}}
    \PYG{n+nb}{print} \PYG{l+s+s2}{\PYGZdq{}}\PYG{l+s+s2}{Starting ForestFire}\PYG{l+s+s2}{\PYGZdq{}}

    \PYG{c+c1}{\PYGZsh{} Creating Random Forests: build n\PYGZus{}trees, each sees only subs\PYGZsh{}et of data points and subset of features of data}
    \PYG{k}{for} \PYG{n}{i} \PYG{o+ow}{in} \PYG{n+nb}{range}\PYG{p}{(}\PYG{n}{n\PYGZus{}forests}\PYG{p}{)}\PYG{p}{:}

        \PYG{c+c1}{\PYGZsh{} create the forest}
        \PYG{n+nb}{print} \PYG{l+s+s2}{\PYGZdq{}}\PYG{l+s+s2}{ }\PYG{l+s+s2}{\PYGZdq{}}
        \PYG{n+nb}{print} \PYG{l+s+s2}{\PYGZdq{}}\PYG{l+s+s2}{Building Random Forest Nr. }\PYG{l+s+s2}{\PYGZdq{}} \PYG{o}{+} \PYG{n+nb}{str}\PYG{p}{(}\PYG{n}{i} \PYG{o}{+} \PYG{l+m+mi}{1}\PYG{p}{)}
        \PYG{n}{RF}\PYG{p}{,} \PYG{n}{Probability}\PYG{p}{[}\PYG{n}{i}\PYG{p}{]}\PYG{p}{,} \PYG{n}{trees} \PYG{o}{=} \PYG{n}{buildforest}\PYG{p}{(}\PYG{n}{data}\PYG{p}{,} \PYG{n}{n\PYGZus{}trees}\PYG{p}{,} \PYG{n}{scoref}\PYG{p}{,} \PYG{n}{n\PYGZus{}feat}\PYG{p}{,} \PYG{n}{min\PYGZus{}data}\PYG{p}{,} \PYG{n}{pruning}\PYG{p}{)}
        \PYG{c+c1}{\PYGZsh{} print \PYGZdq{}RF: \PYGZdq{} + str(RF) \PYGZsh{} Debugging Line}

        \PYG{c+c1}{\PYGZsh{} Update probability}
        \PYG{n}{prob\PYGZus{}current} \PYG{o}{=} \PYG{n}{update\PYGZus{}prob}\PYG{p}{(}\PYG{n}{Probability}\PYG{p}{,} \PYG{n}{i}\PYG{p}{,} \PYG{n}{weight\PYGZus{}mean}\PYG{p}{,} \PYG{n}{weight\PYGZus{}gradient}\PYG{p}{,} \PYG{n}{multiplier}\PYG{p}{,} \PYG{n}{seen\PYGZus{}forests}\PYG{p}{)}
        \PYG{n+nb}{print} \PYG{l+s+s2}{\PYGZdq{}}\PYG{l+s+s2}{max Probability: }\PYG{l+s+s2}{\PYGZdq{}} \PYG{o}{+} \PYG{n+nb}{str}\PYG{p}{(}\PYG{n}{np}\PYG{o}{.}\PYG{n}{max}\PYG{p}{(}\PYG{n}{prob\PYGZus{}current}\PYG{p}{)}\PYG{p}{)}
        \PYG{c+c1}{\PYGZsh{} print np.multiply(np.divide(1.0, n\PYGZus{}feat), 2)}
        \PYG{k}{if} \PYG{n}{i} \PYG{o}{\PYGZgt{}} \PYG{l+m+mi}{1} \PYG{o+ow}{and} \PYG{n}{np}\PYG{o}{.}\PYG{n}{max}\PYG{p}{(}\PYG{n}{prob\PYGZus{}current}\PYG{p}{)} \PYG{o}{\PYGZlt{}} \PYG{n}{np}\PYG{o}{.}\PYG{n}{multiply}\PYG{p}{(}\PYG{n}{np}\PYG{o}{.}\PYG{n}{divide}\PYG{p}{(}\PYG{l+m+mf}{1.0}\PYG{p}{,} \PYG{n}{n\PYGZus{}feat}\PYG{p}{)}\PYG{p}{,} \PYG{l+m+mi}{2}\PYG{p}{)}\PYG{p}{:}
            \PYG{n}{multiplier} \PYG{o}{+}\PYG{o}{=} \PYG{n}{multiplier\PYGZus{}stepup}
            \PYG{n+nb}{print} \PYG{l+s+s2}{\PYGZdq{}}\PYG{l+s+s2}{raised multiplier to }\PYG{l+s+s2}{\PYGZdq{}} \PYG{o}{+} \PYG{n+nb}{str}\PYG{p}{(}\PYG{n}{multiplier}\PYG{p}{)}
        \PYG{c+c1}{\PYGZsh{} print RF \PYGZsh{} Debugging Line}
        \PYG{c+c1}{\PYGZsh{} print \PYGZdq{} \PYGZdq{} \PYGZsh{} Debugging Line}
        \PYG{c+c1}{\PYGZsh{} print \PYGZdq{}Predicting new possible configs\PYGZdq{} \PYGZsh{} Debugging Line}
        \PYG{c+c1}{\PYGZsh{} print \PYGZdq{}biased configs\PYGZdq{} \PYGZsh{} Debugging Line}

        \PYG{c+c1}{\PYGZsh{} test new biased and unbiased feature sets and extract the best feature sets}
        \PYG{n}{best\PYGZus{}mean\PYGZus{}biased}\PYG{p}{,} \PYG{n}{best\PYGZus{}var\PYGZus{}biased}\PYG{p}{,} \PYG{n}{best\PYGZus{}featureset\PYGZus{}mean\PYGZus{}biased}\PYG{p}{,} \PYG{n}{best\PYGZus{}featureset\PYGZus{}var\PYGZus{}biased} \PYG{o}{=} \PYG{n}{forest\PYGZus{}predict}\PYG{p}{(}
            \PYG{n}{data}\PYG{p}{,} \PYG{n}{trees}\PYG{p}{,} \PYG{n}{prob\PYGZus{}current}\PYG{p}{,} \PYG{n}{n\PYGZus{}configs\PYGZus{}biased}\PYG{p}{,} \PYG{n}{biased}\PYG{o}{=}\PYG{k+kc}{True}\PYG{p}{)}
        \PYG{c+c1}{\PYGZsh{} print \PYGZdq{} \PYGZdq{} \PYGZsh{} Debugging Line}
        \PYG{c+c1}{\PYGZsh{} print \PYGZdq{}unbiased configs\PYGZdq{} \PYGZsh{} Debugging Line}
        \PYG{n}{best\PYGZus{}mean\PYGZus{}unbiased}\PYG{p}{,} \PYG{n}{best\PYGZus{}var\PYGZus{}unbiased}\PYG{p}{,} \PYG{n}{best\PYGZus{}featureset\PYGZus{}mean\PYGZus{}unbiased}\PYG{p}{,} \PYG{n}{best\PYGZus{}featureset\PYGZus{}var\PYGZus{}unbiased} \PYG{o}{=} \PYG{n}{forest\PYGZus{}predict}\PYG{p}{(}
            \PYG{n}{data}\PYG{p}{,} \PYG{n}{trees}\PYG{p}{,} \PYG{n}{prob\PYGZus{}current}\PYG{p}{,} \PYG{n}{n\PYGZus{}configs\PYGZus{}unbiased}\PYG{p}{,} \PYG{n}{biased}\PYG{o}{=}\PYG{k+kc}{False}\PYG{p}{)}
        \PYG{c+c1}{\PYGZsh{} print \PYGZdq{}best mean\PYGZus{}biased: \PYGZdq{} + str(best\PYGZus{}mean\PYGZus{}biased) \PYGZsh{} Debugging Line}
        \PYG{c+c1}{\PYGZsh{} print \PYGZdq{}best mean\PYGZus{}unbiased: \PYGZdq{} + str(best\PYGZus{}mean\PYGZus{}unbiased) \PYGZsh{} Debugging Line}
        \PYG{c+c1}{\PYGZsh{} print \PYGZdq{} \PYGZdq{} \PYGZsh{} Debugging Line}
        \PYG{n}{best\PYGZus{}mean} \PYG{o}{=} \PYG{n}{np}\PYG{o}{.}\PYG{n}{max}\PYG{p}{(}\PYG{p}{(}\PYG{n}{best\PYGZus{}mean\PYGZus{}biased}\PYG{p}{,} \PYG{n}{best\PYGZus{}mean\PYGZus{}unbiased}\PYG{p}{)}\PYG{p}{)}
        \PYG{k}{if} \PYG{n}{best\PYGZus{}mean} \PYG{o}{==} \PYG{n}{best\PYGZus{}mean\PYGZus{}biased}\PYG{p}{:}
            \PYG{n}{best\PYGZus{}featureset\PYGZus{}mean} \PYG{o}{=} \PYG{n}{best\PYGZus{}featureset\PYGZus{}mean\PYGZus{}biased}
            \PYG{n+nb}{print} \PYG{l+s+s2}{\PYGZdq{}}\PYG{l+s+s2}{picked biased feature set for mean}\PYG{l+s+s2}{\PYGZdq{}}
        \PYG{k}{elif} \PYG{n}{best\PYGZus{}mean} \PYG{o}{==} \PYG{n}{best\PYGZus{}mean\PYGZus{}unbiased}\PYG{p}{:}
            \PYG{n}{best\PYGZus{}featureset\PYGZus{}mean} \PYG{o}{=} \PYG{n}{best\PYGZus{}featureset\PYGZus{}mean\PYGZus{}unbiased}
            \PYG{n+nb}{print} \PYG{l+s+s2}{\PYGZdq{}}\PYG{l+s+s2}{picked unbiased feature set for mean}\PYG{l+s+s2}{\PYGZdq{}}
        \PYG{c+c1}{\PYGZsh{} print best\PYGZus{}mean \PYGZsh{} Debugging Line}
        \PYG{c+c1}{\PYGZsh{} print best\PYGZus{}featureset\PYGZus{}mean \PYGZsh{} Debugging Line}
        \PYG{c+c1}{\PYGZsh{} print \PYGZdq{}best\PYGZus{}var\PYGZus{}biased: \PYGZdq{} + str(best\PYGZus{}var\PYGZus{}biased) \PYGZsh{} Debugging Line}
        \PYG{c+c1}{\PYGZsh{} print \PYGZdq{}best\PYGZus{}var\PYGZus{}unbiased: \PYGZdq{} + str(best\PYGZus{}var\PYGZus{}unbiased) \PYGZsh{} Debugging Line}
        \PYG{n}{best\PYGZus{}var} \PYG{o}{=} \PYG{n}{np}\PYG{o}{.}\PYG{n}{max}\PYG{p}{(}\PYG{p}{(}\PYG{n}{best\PYGZus{}var\PYGZus{}biased}\PYG{p}{,} \PYG{n}{best\PYGZus{}var\PYGZus{}unbiased}\PYG{p}{)}\PYG{p}{)}
        \PYG{k}{if} \PYG{n}{best\PYGZus{}var} \PYG{o}{==} \PYG{n}{best\PYGZus{}var\PYGZus{}biased}\PYG{p}{:}
            \PYG{n}{best\PYGZus{}featureset\PYGZus{}var} \PYG{o}{=} \PYG{n}{best\PYGZus{}featureset\PYGZus{}var\PYGZus{}biased}
            \PYG{n+nb}{print} \PYG{l+s+s2}{\PYGZdq{}}\PYG{l+s+s2}{picked biased feature set for var}\PYG{l+s+s2}{\PYGZdq{}}
        \PYG{k}{elif} \PYG{n}{best\PYGZus{}var} \PYG{o}{==} \PYG{n}{best\PYGZus{}var\PYGZus{}unbiased}\PYG{p}{:}
            \PYG{n}{best\PYGZus{}featureset\PYGZus{}var} \PYG{o}{=} \PYG{n}{best\PYGZus{}featureset\PYGZus{}var\PYGZus{}unbiased}
            \PYG{n+nb}{print} \PYG{l+s+s2}{\PYGZdq{}}\PYG{l+s+s2}{picked unbiased feature set for var}\PYG{l+s+s2}{\PYGZdq{}}

        \PYG{c+c1}{\PYGZsh{} update database with two new feature sets}
        \PYG{c+c1}{\PYGZsh{} print \PYGZdq{}current feature sets:\PYGZdq{} + str(data[:, :\PYGZhy{}1]) \PYGZsh{} Debugging Line}
        \PYG{c+c1}{\PYGZsh{} print \PYGZdq{}best\PYGZus{}var feature set:\PYGZdq{} + str(best\PYGZus{}featureset\PYGZus{}var) \PYGZsh{} Debugging Line}
        \PYG{c+c1}{\PYGZsh{} print \PYGZdq{}best\PYGZus{}mean feature set:\PYGZdq{} + str(best\PYGZus{}featureset\PYGZus{}mean) \PYGZsh{} Debugging Line}

        \PYG{c+c1}{\PYGZsh{} check if newly selected feature sets  are already in data. if so, there is no need to compute again}
        \PYG{n}{check\PYGZus{}mean} \PYG{o}{=} \PYG{n+nb}{any}\PYG{p}{(}\PYG{n}{check} \PYG{k}{for} \PYG{n}{check} \PYG{o+ow}{in} \PYG{p}{(}\PYG{n}{np}\PYG{o}{.}\PYG{n}{array\PYGZus{}equal}\PYG{p}{(}\PYG{n}{data}\PYG{p}{[}\PYG{n}{entry}\PYG{p}{,} \PYG{p}{:}\PYG{o}{\PYGZhy{}}\PYG{l+m+mi}{1}\PYG{p}{]}\PYG{p}{,} \PYG{n}{best\PYGZus{}featureset\PYGZus{}mean}\PYG{p}{)} \PYG{k}{for} \PYG{n}{entry} \PYG{o+ow}{in} \PYG{n+nb}{range}\PYG{p}{(}\PYG{n+nb}{len}\PYG{p}{(}\PYG{n}{data}\PYG{p}{)}\PYG{p}{)}\PYG{p}{)}\PYG{p}{)}
        \PYG{n}{check\PYGZus{}var} \PYG{o}{=} \PYG{n+nb}{any}\PYG{p}{(}\PYG{n}{check} \PYG{k}{for} \PYG{n}{check} \PYG{o+ow}{in} \PYG{p}{(}\PYG{n}{np}\PYG{o}{.}\PYG{n}{array\PYGZus{}equal}\PYG{p}{(}\PYG{n}{data}\PYG{p}{[}\PYG{n}{entry}\PYG{p}{,} \PYG{p}{:}\PYG{o}{\PYGZhy{}}\PYG{l+m+mi}{1}\PYG{p}{]}\PYG{p}{,} \PYG{n}{best\PYGZus{}featureset\PYGZus{}var}\PYG{p}{)} \PYG{k}{for} \PYG{n}{entry} \PYG{o+ow}{in} \PYG{n+nb}{range}\PYG{p}{(}\PYG{n+nb}{len}\PYG{p}{(}\PYG{n}{data}\PYG{p}{)}\PYG{p}{)}\PYG{p}{)}\PYG{p}{)}

        \PYG{c+c1}{\PYGZsh{} print \PYGZdq{}data len: \PYGZdq{} + str(len(data)) \PYGZsh{} Debugging Line}

        \PYG{n}{double\PYGZus{}var} \PYG{o}{=} \PYG{n}{np}\PYG{o}{.}\PYG{n}{all}\PYG{p}{(}\PYG{n}{np}\PYG{o}{.}\PYG{n}{all}\PYG{p}{(}\PYG{n}{data}\PYG{p}{[}\PYG{n}{x}\PYG{p}{,} \PYG{p}{:}\PYG{o}{\PYGZhy{}}\PYG{l+m+mi}{1}\PYG{p}{]} \PYG{o}{==} \PYG{n}{best\PYGZus{}featureset\PYGZus{}var} \PYG{k}{for} \PYG{n}{x} \PYG{o+ow}{in} \PYG{n+nb}{range}\PYG{p}{(}\PYG{n+nb}{len}\PYG{p}{(}\PYG{n}{data}\PYG{p}{[}\PYG{p}{:}\PYG{p}{,} \PYG{o}{\PYGZhy{}}\PYG{l+m+mi}{1}\PYG{p}{]}\PYG{p}{)}\PYG{p}{)}\PYG{p}{)}\PYG{p}{)}
        \PYG{n}{double\PYGZus{}mean} \PYG{o}{=} \PYG{n}{np}\PYG{o}{.}\PYG{n}{all}\PYG{p}{(}\PYG{n}{np}\PYG{o}{.}\PYG{n}{all}\PYG{p}{(}\PYG{n}{data}\PYG{p}{[}\PYG{n}{x}\PYG{p}{,} \PYG{p}{:}\PYG{o}{\PYGZhy{}}\PYG{l+m+mi}{1}\PYG{p}{]} \PYG{o}{==} \PYG{n}{best\PYGZus{}featureset\PYGZus{}mean} \PYG{k}{for} \PYG{n}{x} \PYG{o+ow}{in} \PYG{n+nb}{range}\PYG{p}{(}\PYG{n+nb}{len}\PYG{p}{(}\PYG{n}{data}\PYG{p}{[}\PYG{p}{:}\PYG{p}{,} \PYG{o}{\PYGZhy{}}\PYG{l+m+mi}{1}\PYG{p}{]}\PYG{p}{)}\PYG{p}{)}\PYG{p}{)}\PYG{p}{)}

        \PYG{k}{if} \PYG{n}{check\PYGZus{}var}\PYG{p}{:}
            \PYG{n}{z} \PYG{o}{=} \PYG{l+m+mi}{0}
            \PYG{n}{stopper} \PYG{o}{=} \PYG{k+kc}{False}
            \PYG{k}{for} \PYG{n}{x} \PYG{o+ow}{in} \PYG{n}{double\PYGZus{}var}\PYG{p}{:}
                \PYG{k}{if} \PYG{n}{x}\PYG{o}{.}\PYG{n}{all}\PYG{p}{(}\PYG{p}{)} \PYG{o+ow}{and} \PYG{o+ow}{not} \PYG{n}{stopper}\PYG{p}{:}
                    \PYG{c+c1}{\PYGZsh{} print \PYGZdq{}Stopper: \PYGZdq{} + str(stopper) \PYGZsh{} Debugging Line}
                    \PYG{n+nb}{print} \PYG{l+s+s2}{\PYGZdq{}}\PYG{l+s+s2}{Variance feature set already computed. No need to do it again}\PYG{l+s+s2}{\PYGZdq{}}
                    \PYG{n}{data} \PYG{o}{=} \PYG{n}{np}\PYG{o}{.}\PYG{n}{append}\PYG{p}{(}\PYG{n}{data}\PYG{p}{,} \PYG{p}{[}\PYG{n}{data}\PYG{p}{[}\PYG{n}{z}\PYG{p}{]}\PYG{p}{]}\PYG{p}{,} \PYG{n}{axis}\PYG{o}{=}\PYG{l+m+mi}{0}\PYG{p}{)}
                    \PYG{n}{stopper} \PYG{o}{=} \PYG{k+kc}{True}
                \PYG{n}{z} \PYG{o}{+}\PYG{o}{=} \PYG{l+m+mi}{1}
        \PYG{k}{else}\PYG{p}{:}
            \PYG{n}{data} \PYG{o}{=} \PYG{n}{update\PYGZus{}database}\PYG{p}{(}\PYG{n}{X}\PYG{p}{,} \PYG{n}{y}\PYG{p}{,} \PYG{n}{data}\PYG{p}{,} \PYG{n}{best\PYGZus{}featureset\PYGZus{}var}\PYG{p}{,} \PYG{n}{X\PYGZus{}test}\PYG{p}{,} \PYG{n}{y\PYGZus{}test}\PYG{p}{)}

        \PYG{k}{if} \PYG{n}{check\PYGZus{}mean}\PYG{p}{:}
            \PYG{n}{z} \PYG{o}{=} \PYG{l+m+mi}{0}
            \PYG{n}{stopper} \PYG{o}{=} \PYG{k+kc}{False}
            \PYG{k}{for} \PYG{n}{x} \PYG{o+ow}{in} \PYG{n}{double\PYGZus{}mean}\PYG{p}{:}
                \PYG{k}{if} \PYG{n}{x}\PYG{o}{.}\PYG{n}{all}\PYG{p}{(}\PYG{p}{)} \PYG{o+ow}{and} \PYG{o+ow}{not} \PYG{n}{stopper}\PYG{p}{:}
                    \PYG{c+c1}{\PYGZsh{} print \PYGZdq{}Stopper: \PYGZdq{} + str(stopper) \PYGZsh{} Debugging Line}
                    \PYG{n+nb}{print} \PYG{l+s+s2}{\PYGZdq{}}\PYG{l+s+s2}{Mean feature set already computed. No need to do it again!}\PYG{l+s+s2}{\PYGZdq{}}
                    \PYG{n}{data} \PYG{o}{=} \PYG{n}{np}\PYG{o}{.}\PYG{n}{append}\PYG{p}{(}\PYG{n}{data}\PYG{p}{,} \PYG{p}{[}\PYG{n}{data}\PYG{p}{[}\PYG{n}{z}\PYG{p}{]}\PYG{p}{]}\PYG{p}{,} \PYG{n}{axis}\PYG{o}{=}\PYG{l+m+mi}{0}\PYG{p}{)}
                    \PYG{n}{stopper} \PYG{o}{=} \PYG{k+kc}{True}
                \PYG{n}{z} \PYG{o}{+}\PYG{o}{=} \PYG{l+m+mi}{1}
        \PYG{k}{else}\PYG{p}{:}
            \PYG{n}{data} \PYG{o}{=} \PYG{n}{update\PYGZus{}database}\PYG{p}{(}\PYG{n}{X}\PYG{p}{,} \PYG{n}{y}\PYG{p}{,} \PYG{n}{data}\PYG{p}{,} \PYG{n}{best\PYGZus{}featureset\PYGZus{}mean}\PYG{p}{,} \PYG{n}{X\PYGZus{}test}\PYG{p}{,} \PYG{n}{y\PYGZus{}test}\PYG{p}{)}

        \PYG{c+c1}{\PYGZsh{} check for current best feature sets}
        \PYG{n}{best\PYGZus{}featuresets\PYGZus{}sorted} \PYG{o}{=} \PYG{n}{data}\PYG{p}{[}\PYG{n}{np}\PYG{o}{.}\PYG{n}{argsort}\PYG{p}{(}\PYG{o}{\PYGZhy{}}\PYG{n}{data}\PYG{p}{[}\PYG{p}{:}\PYG{p}{,} \PYG{o}{\PYGZhy{}}\PYG{l+m+mi}{1}\PYG{p}{]}\PYG{p}{)}\PYG{p}{]}
        \PYG{k}{if} \PYG{n}{i} \PYG{o}{==} \PYG{l+m+mi}{0}\PYG{p}{:}
            \PYG{n}{best\PYGZus{}featuresets\PYGZus{}sorted\PYGZus{}old} \PYG{o}{=} \PYG{n}{best\PYGZus{}featuresets\PYGZus{}sorted}  \PYG{c+c1}{\PYGZsh{} initialize storage value}
        \PYG{c+c1}{\PYGZsh{} if the best 5 feature sets have improved, update the current best feature sets}
        \PYG{k}{if} \PYG{n+nb}{sum}\PYG{p}{(}\PYG{n}{best\PYGZus{}featuresets\PYGZus{}sorted}\PYG{p}{[}\PYG{p}{:}\PYG{l+m+mi}{5}\PYG{p}{,} \PYG{o}{\PYGZhy{}}\PYG{l+m+mi}{1}\PYG{p}{]}\PYG{p}{)} \PYG{o}{\PYGZgt{}} \PYG{n+nb}{sum}\PYG{p}{(}\PYG{n}{best\PYGZus{}featuresets\PYGZus{}sorted\PYGZus{}old}\PYG{p}{[}\PYG{p}{:}\PYG{l+m+mi}{5}\PYG{p}{,} \PYG{o}{\PYGZhy{}}\PYG{l+m+mi}{1}\PYG{p}{]}\PYG{p}{)} \PYG{o+ow}{or} \PYG{n}{i} \PYG{o}{==} \PYG{l+m+mi}{0}\PYG{p}{:}
            \PYG{n+nb}{print} \PYG{l+s+s2}{\PYGZdq{}}\PYG{l+s+s2}{found new best 5 feature sets: }\PYG{l+s+s2}{\PYGZdq{}} \PYG{o}{+} \PYG{n+nb}{str}\PYG{p}{(}\PYG{n}{best\PYGZus{}featuresets\PYGZus{}sorted}\PYG{p}{[}\PYG{p}{:}\PYG{l+m+mi}{5}\PYG{p}{]}\PYG{p}{)}
        \PYG{c+c1}{\PYGZsh{} store values for comparison to later results}
        \PYG{n}{best\PYGZus{}featuresets\PYGZus{}sorted\PYGZus{}old} \PYG{o}{=} \PYG{n}{best\PYGZus{}featuresets\PYGZus{}sorted}

    \PYG{c+c1}{\PYGZsh{} \PYGZsh{}\PYGZsh{}\PYGZsh{} End of ForestFire \PYGZsh{}\PYGZsh{}\PYGZsh{}}

    \PYG{c+c1}{\PYGZsh{} store results}
    \PYG{n+nb}{print} \PYG{l+s+s2}{\PYGZdq{}}\PYG{l+s+s2}{Storing results}\PYG{l+s+s2}{\PYGZdq{}}
    \PYG{n}{np}\PYG{o}{.}\PYG{n}{savetxt}\PYG{p}{(}\PYG{l+s+s1}{\PYGZsq{}}\PYG{l+s+s1}{results.txt}\PYG{l+s+s1}{\PYGZsq{}}\PYG{p}{,} \PYG{n}{best\PYGZus{}featuresets\PYGZus{}sorted}\PYG{p}{)}
    \PYG{n}{np}\PYG{o}{.}\PYG{n}{save}\PYG{p}{(}\PYG{l+s+s1}{\PYGZsq{}}\PYG{l+s+s1}{results}\PYG{l+s+s1}{\PYGZsq{}}\PYG{p}{,} \PYG{n}{best\PYGZus{}featuresets\PYGZus{}sorted}\PYG{p}{)}

    \PYG{n+nb}{print} \PYG{l+s+s2}{\PYGZdq{}}\PYG{l+s+s2}{ }\PYG{l+s+s2}{\PYGZdq{}}
    \PYG{n+nb}{print} \PYG{l+s+s2}{\PYGZdq{}}\PYG{l+s+s2}{ForestFire finished}\PYG{l+s+s2}{\PYGZdq{}}
    \PYG{n+nb}{print} \PYG{l+s+s2}{\PYGZdq{}}\PYG{l+s+s2}{ }\PYG{l+s+s2}{\PYGZdq{}}

    \PYG{k}{if} \PYG{n}{demo\PYGZus{}mode}\PYG{p}{:}
        \PYG{c+c1}{\PYGZsh{} Generate additional data set to compare performance of RF to random selection of feature sets}
        \PYG{n+nb}{print} \PYG{l+s+s2}{\PYGZdq{}}\PYG{l+s+s2}{Generating more randomly selected feature sets for comparison}\PYG{l+s+s2}{\PYGZdq{}}
        \PYG{n}{data\PYGZus{}compare} \PYG{o}{=} \PYG{n}{np}\PYG{o}{.}\PYG{n}{append}\PYG{p}{(}\PYG{n}{data\PYGZus{}start}\PYG{p}{,} \PYG{n}{gen\PYGZus{}database}\PYG{p}{(}\PYG{l+m+mi}{2} \PYG{o}{*} \PYG{n}{n\PYGZus{}forests}\PYG{p}{,} \PYG{n}{X}\PYG{p}{,} \PYG{n}{y}\PYG{p}{,} \PYG{n}{X\PYGZus{}test}\PYG{p}{,} \PYG{n}{y\PYGZus{}test}\PYG{p}{)}\PYG{p}{,} \PYG{n}{axis}\PYG{o}{=}\PYG{l+m+mi}{0}\PYG{p}{)}
        \PYG{c+c1}{\PYGZsh{} print \PYGZdq{}len(data\PYGZus{}compare): \PYGZdq{} + str(len(data\PYGZus{}compare))}

        \PYG{c+c1}{\PYGZsh{} sort according to lowest MSE}
        \PYG{n}{best\PYGZus{}featuresets\PYGZus{}sorted\PYGZus{}compare} \PYG{o}{=} \PYG{n}{data\PYGZus{}compare}\PYG{p}{[}\PYG{n}{np}\PYG{o}{.}\PYG{n}{argsort}\PYG{p}{(}\PYG{o}{\PYGZhy{}}\PYG{n}{data\PYGZus{}compare}\PYG{p}{[}\PYG{p}{:}\PYG{p}{,} \PYG{o}{\PYGZhy{}}\PYG{l+m+mi}{1}\PYG{p}{]}\PYG{p}{)}\PYG{p}{]}

        \PYG{c+c1}{\PYGZsh{} print out some of the results}
        \PYG{n+nb}{print} \PYG{l+s+s2}{\PYGZdq{}}\PYG{l+s+s2}{best 5 feature sets of random selection: }\PYG{l+s+s2}{\PYGZdq{}} \PYG{o}{+} \PYG{n+nb}{str}\PYG{p}{(}\PYG{n}{best\PYGZus{}featuresets\PYGZus{}sorted\PYGZus{}compare}\PYG{p}{[}\PYG{p}{:}\PYG{l+m+mi}{5}\PYG{p}{]}\PYG{p}{)}
        \PYG{n+nb}{print} \PYG{l+s+s2}{\PYGZdq{}}\PYG{l+s+s2}{ }\PYG{l+s+s2}{\PYGZdq{}}
        \PYG{n+nb}{print} \PYG{l+s+s2}{\PYGZdq{}}\PYG{l+s+s2}{Best result after }\PYG{l+s+s2}{\PYGZdq{}} \PYG{o}{+} \PYG{n+nb}{str}\PYG{p}{(}\PYG{n}{n\PYGZus{}start} \PYG{o}{+} \PYG{l+m+mi}{2} \PYG{o}{*} \PYG{n}{n\PYGZus{}forests}\PYG{p}{)} \PYG{o}{+} \PYG{l+s+s2}{\PYGZdq{}}\PYG{l+s+s2}{ random SVM runs: }\PYG{l+s+s2}{\PYGZdq{}} \PYG{o}{+} \PYG{n+nb}{str}\PYG{p}{(}\PYG{n}{best\PYGZus{}featuresets\PYGZus{}sorted\PYGZus{}compare}\PYG{p}{[}\PYG{l+m+mi}{0}\PYG{p}{,} \PYG{o}{\PYGZhy{}}\PYG{l+m+mi}{1}\PYG{p}{]}\PYG{p}{)}
        \PYG{n+nb}{print} \PYG{l+s+s2}{\PYGZdq{}}\PYG{l+s+s2}{Best result of ForestFire after }\PYG{l+s+s2}{\PYGZdq{}} \PYG{o}{+} \PYG{n+nb}{str}\PYG{p}{(}\PYG{n}{n\PYGZus{}start}\PYG{p}{)} \PYG{o}{+} \PYG{l+s+s2}{\PYGZdq{}}\PYG{l+s+s2}{ initial random runs and }\PYG{l+s+s2}{\PYGZdq{}} \PYG{o}{+} \PYG{n+nb}{str}\PYG{p}{(}\PYG{l+m+mi}{2} \PYG{o}{*} \PYG{n}{n\PYGZus{}forests}\PYG{p}{)} \PYG{o}{+} \PYG{l+s+s2}{\PYGZdq{}}\PYG{l+s+s2}{ guided runs: }\PYG{l+s+s2}{\PYGZdq{}} \PYG{o}{+} \PYG{n+nb}{str}\PYG{p}{(}\PYG{n}{best\PYGZus{}featuresets\PYGZus{}sorted}\PYG{p}{[}\PYG{l+m+mi}{0}\PYG{p}{,} \PYG{o}{\PYGZhy{}}\PYG{l+m+mi}{1}\PYG{p}{]}\PYG{p}{)}
        \PYG{k}{if} \PYG{n}{best\PYGZus{}featuresets\PYGZus{}sorted}\PYG{p}{[}\PYG{l+m+mi}{0}\PYG{p}{,} \PYG{o}{\PYGZhy{}}\PYG{l+m+mi}{1}\PYG{p}{]} \PYG{o}{\PYGZgt{}} \PYG{n}{best\PYGZus{}featuresets\PYGZus{}sorted\PYGZus{}compare}\PYG{p}{[}\PYG{l+m+mi}{0}\PYG{p}{,} \PYG{o}{\PYGZhy{}}\PYG{l+m+mi}{1}\PYG{p}{]}\PYG{p}{:}
            \PYG{n+nb}{print} \PYG{l+s+s2}{\PYGZdq{}}\PYG{l+s+s2}{Performance with ForestFire improved by }\PYG{l+s+s2}{\PYGZdq{}} \PYG{o}{+} \PYG{n+nb}{str}\PYG{p}{(}\PYG{o}{\PYGZhy{}}\PYG{l+m+mi}{100} \PYG{o}{*} \PYG{p}{(}\PYG{l+m+mi}{1} \PYG{o}{\PYGZhy{}} \PYG{n}{np}\PYG{o}{.}\PYG{n}{divide}\PYG{p}{(}\PYG{n}{best\PYGZus{}featuresets\PYGZus{}sorted}\PYG{p}{[}\PYG{l+m+mi}{0}\PYG{p}{,} \PYG{o}{\PYGZhy{}}\PYG{l+m+mi}{1}\PYG{p}{]}\PYG{p}{,} \PYG{n}{best\PYGZus{}featuresets\PYGZus{}sorted\PYGZus{}compare}\PYG{p}{[}\PYG{l+m+mi}{0}\PYG{p}{,} \PYG{o}{\PYGZhy{}}\PYG{l+m+mi}{1}\PYG{p}{]}\PYG{p}{)}\PYG{p}{)}\PYG{p}{)} \PYG{o}{+} \PYG{l+s+s2}{\PYGZdq{}}\PYG{l+s+s2}{\PYGZpc{}}\PYG{l+s+s2}{\PYGZdq{}}
        \PYG{k}{if} \PYG{n}{best\PYGZus{}featuresets\PYGZus{}sorted}\PYG{p}{[}\PYG{l+m+mi}{0}\PYG{p}{,} \PYG{o}{\PYGZhy{}}\PYG{l+m+mi}{1}\PYG{p}{]} \PYG{o}{==} \PYG{n}{best\PYGZus{}featuresets\PYGZus{}sorted\PYGZus{}compare}\PYG{p}{[}\PYG{l+m+mi}{0}\PYG{p}{,} \PYG{o}{\PYGZhy{}}\PYG{l+m+mi}{1}\PYG{p}{]}\PYG{p}{:}
            \PYG{n+nb}{print} \PYG{l+s+s2}{\PYGZdq{}}\PYG{l+s+s2}{Performance could not be improved (same MSE as in random selection)}\PYG{l+s+s2}{\PYGZdq{}}
        \PYG{k}{if} \PYG{n}{best\PYGZus{}featuresets\PYGZus{}sorted}\PYG{p}{[}\PYG{l+m+mi}{0}\PYG{p}{,} \PYG{o}{\PYGZhy{}}\PYG{l+m+mi}{1}\PYG{p}{]} \PYG{o}{\PYGZlt{}} \PYG{n}{best\PYGZus{}featuresets\PYGZus{}sorted\PYGZus{}compare}\PYG{p}{[}\PYG{l+m+mi}{0}\PYG{p}{,} \PYG{o}{\PYGZhy{}}\PYG{l+m+mi}{1}\PYG{p}{]}\PYG{p}{:}
            \PYG{n+nb}{print} \PYG{l+s+s2}{\PYGZdq{}}\PYG{l+s+s2}{Performance deteriorated, ForestFire is not suitable :(}\PYG{l+s+s2}{\PYGZdq{}}
        \PYG{n+nb}{print} \PYG{l+s+s2}{\PYGZdq{}}\PYG{l+s+s2}{Execution finished}\PYG{l+s+s2}{\PYGZdq{}}

        \PYG{c+c1}{\PYGZsh{} Compare Random Search VS Random Forest Search}
        \PYG{n+nb}{print} \PYG{l+s+s2}{\PYGZdq{}}\PYG{l+s+s2}{ }\PYG{l+s+s2}{\PYGZdq{}}
        \PYG{n+nb}{print} \PYG{l+s+s2}{\PYGZdq{}}\PYG{l+s+s2}{Found best value for ForestFire Search after }\PYG{l+s+s2}{\PYGZdq{}} \PYG{o}{+} \PYG{n+nb}{str}\PYG{p}{(}\PYG{n}{n\PYGZus{}start}\PYG{p}{)} \PYG{o}{+} \PYG{l+s+s2}{\PYGZdq{}}\PYG{l+s+s2}{ initial runs and }\PYG{l+s+s2}{\PYGZdq{}} \PYG{o}{+} \PYG{n+nb}{str}\PYG{p}{(}\PYG{n}{np}\PYG{o}{.}\PYG{n}{argmax}\PYG{p}{(}\PYG{n}{data}\PYG{p}{[}\PYG{p}{:}\PYG{p}{,} \PYG{o}{\PYGZhy{}}\PYG{l+m+mi}{1}\PYG{p}{]} \PYG{o}{+} \PYG{l+m+mi}{1}\PYG{p}{)} \PYG{o}{\PYGZhy{}} \PYG{n}{n\PYGZus{}start}\PYG{p}{)} \PYG{o}{+} \PYG{l+s+s2}{\PYGZdq{}}\PYG{l+s+s2}{/}\PYG{l+s+s2}{\PYGZdq{}} \PYG{o}{+} \PYG{n+nb}{str}\PYG{p}{(}\PYG{n+nb}{len}\PYG{p}{(}\PYG{n}{data}\PYG{p}{)} \PYG{o}{\PYGZhy{}} \PYG{n}{n\PYGZus{}start}\PYG{p}{)} \PYG{o}{+} \PYG{l+s+s2}{\PYGZdq{}}\PYG{l+s+s2}{ smart runs}\PYG{l+s+s2}{\PYGZdq{}}
        \PYG{n+nb}{print} \PYG{l+s+s2}{\PYGZdq{}}\PYG{l+s+s2}{Best value with ForestFire: }\PYG{l+s+s2}{\PYGZdq{}} \PYG{o}{+} \PYG{n+nb}{str}\PYG{p}{(}\PYG{n}{np}\PYG{o}{.}\PYG{n}{max}\PYG{p}{(}\PYG{n}{data}\PYG{p}{[}\PYG{p}{:}\PYG{p}{,} \PYG{o}{\PYGZhy{}}\PYG{l+m+mi}{1}\PYG{p}{]}\PYG{p}{)}\PYG{p}{)}
        \PYG{n+nb}{print} \PYG{l+s+s2}{\PYGZdq{}}\PYG{l+s+s2}{ }\PYG{l+s+s2}{\PYGZdq{}}
        \PYG{n+nb}{print} \PYG{l+s+s2}{\PYGZdq{}}\PYG{l+s+s2}{Found best value for Random Search after }\PYG{l+s+s2}{\PYGZdq{}} \PYG{o}{+} \PYG{n+nb}{str}\PYG{p}{(}\PYG{n}{np}\PYG{o}{.}\PYG{n}{argmax}\PYG{p}{(}\PYG{n}{data\PYGZus{}compare}\PYG{p}{[}\PYG{p}{:}\PYG{p}{,} \PYG{o}{\PYGZhy{}}\PYG{l+m+mi}{1}\PYG{p}{]}\PYG{p}{)}\PYG{p}{)} \PYG{o}{+} \PYG{l+s+s2}{\PYGZdq{}}\PYG{l+s+s2}{ random runs}\PYG{l+s+s2}{\PYGZdq{}}
        \PYG{n+nb}{print} \PYG{l+s+s2}{\PYGZdq{}}\PYG{l+s+s2}{Best value with Random Search: }\PYG{l+s+s2}{\PYGZdq{}} \PYG{o}{+} \PYG{n+nb}{str}\PYG{p}{(}\PYG{n}{np}\PYG{o}{.}\PYG{n}{max}\PYG{p}{(}\PYG{n}{data\PYGZus{}compare}\PYG{p}{[}\PYG{p}{:}\PYG{p}{,} \PYG{o}{\PYGZhy{}}\PYG{l+m+mi}{1}\PYG{p}{]}\PYG{p}{)}\PYG{p}{)}

        \PYG{n+nb}{print} \PYG{l+s+s2}{\PYGZdq{}}\PYG{l+s+s2}{ }\PYG{l+s+s2}{\PYGZdq{}}
        \PYG{n+nb}{print} \PYG{l+s+s2}{\PYGZdq{}}\PYG{l+s+s2}{Creating Plots}\PYG{l+s+s2}{\PYGZdq{}}

        \PYG{c+c1}{\PYGZsh{} plots}
        \PYG{k}{if} \PYG{n}{plot\PYGZus{}enable} \PYG{o+ow}{and} \PYG{n}{demo\PYGZus{}mode}\PYG{p}{:}
            \PYG{c+c1}{\PYGZsh{} first plot}
            \PYG{n}{plt}\PYG{o}{.}\PYG{n}{figure}\PYG{p}{(}\PYG{l+m+mi}{1}\PYG{p}{,} \PYG{n}{figsize}\PYG{o}{=}\PYG{p}{(}\PYG{l+m+mi}{25}\PYG{p}{,} \PYG{l+m+mi}{12}\PYG{p}{)}\PYG{p}{)}
            \PYG{n}{plt}\PYG{o}{.}\PYG{n}{plot}\PYG{p}{(}\PYG{n}{np}\PYG{o}{.}\PYG{n}{array}\PYG{p}{(}\PYG{n+nb}{range}\PYG{p}{(}\PYG{n+nb}{len}\PYG{p}{(}\PYG{n}{data}\PYG{p}{[}\PYG{p}{:}\PYG{p}{,} \PYG{o}{\PYGZhy{}}\PYG{l+m+mi}{1}\PYG{p}{]}\PYG{p}{)}\PYG{p}{)}\PYG{p}{)}\PYG{p}{,} \PYG{n}{data}\PYG{p}{[}\PYG{p}{:}\PYG{p}{,} \PYG{o}{\PYGZhy{}}\PYG{l+m+mi}{1}\PYG{p}{]}\PYG{p}{,} \PYG{n}{label}\PYG{o}{=}\PYG{l+s+s1}{\PYGZsq{}}\PYG{l+s+s1}{ForestFire}\PYG{l+s+s1}{\PYGZsq{}}\PYG{p}{)}
            \PYG{n}{plt}\PYG{o}{.}\PYG{n}{plot}\PYG{p}{(}\PYG{n}{np}\PYG{o}{.}\PYG{n}{array}\PYG{p}{(}\PYG{n+nb}{range}\PYG{p}{(}\PYG{n+nb}{len}\PYG{p}{(}\PYG{n}{data\PYGZus{}compare}\PYG{p}{[}\PYG{p}{:}\PYG{p}{,} \PYG{o}{\PYGZhy{}}\PYG{l+m+mi}{1}\PYG{p}{]}\PYG{p}{)}\PYG{p}{)}\PYG{p}{)}\PYG{p}{,} \PYG{n}{data\PYGZus{}compare}\PYG{p}{[}\PYG{p}{:}\PYG{p}{,} \PYG{o}{\PYGZhy{}}\PYG{l+m+mi}{1}\PYG{p}{]}\PYG{p}{,} \PYG{n}{label}\PYG{o}{=}\PYG{l+s+s1}{\PYGZsq{}}\PYG{l+s+s1}{Random Search}\PYG{l+s+s1}{\PYGZsq{}}\PYG{p}{)}

            \PYG{n}{plt}\PYG{o}{.}\PYG{n}{xlabel}\PYG{p}{(}\PYG{l+s+s1}{\PYGZsq{}}\PYG{l+s+s1}{No. tested Feature Sets}\PYG{l+s+s1}{\PYGZsq{}}\PYG{p}{)}
            \PYG{n}{plt}\PYG{o}{.}\PYG{n}{ylabel}\PYG{p}{(}\PYG{l+s+s1}{\PYGZsq{}}\PYG{l+s+s1}{Score}\PYG{l+s+s1}{\PYGZsq{}}\PYG{p}{)}
            \PYG{n}{plt}\PYG{o}{.}\PYG{n}{title}\PYG{p}{(}\PYG{l+s+s1}{\PYGZsq{}}\PYG{l+s+s1}{Results current best score}\PYG{l+s+s1}{\PYGZsq{}}\PYG{p}{)}
            \PYG{n}{plt}\PYG{o}{.}\PYG{n}{legend}\PYG{p}{(}\PYG{n}{loc}\PYG{o}{=}\PYG{l+m+mi}{2}\PYG{p}{)}
            \PYG{n}{plt}\PYG{o}{.}\PYG{n}{annotate}\PYG{p}{(}\PYG{l+s+s1}{\PYGZsq{}}\PYG{l+s+s1}{Highest Score ForestFire}\PYG{l+s+s1}{\PYGZsq{}}\PYG{p}{,} \PYG{n}{xycoords}\PYG{o}{=}\PYG{l+s+s1}{\PYGZsq{}}\PYG{l+s+s1}{data}\PYG{l+s+s1}{\PYGZsq{}}\PYG{p}{,}
                         \PYG{n}{xy}\PYG{o}{=}\PYG{p}{(}\PYG{n}{np}\PYG{o}{.}\PYG{n}{argmax}\PYG{p}{(}\PYG{n}{data}\PYG{p}{[}\PYG{p}{:}\PYG{p}{,} \PYG{o}{\PYGZhy{}}\PYG{l+m+mi}{1}\PYG{p}{]}\PYG{p}{)}\PYG{p}{,} \PYG{n}{np}\PYG{o}{.}\PYG{n}{max}\PYG{p}{(}\PYG{n}{data}\PYG{p}{[}\PYG{p}{:}\PYG{p}{,} \PYG{o}{\PYGZhy{}}\PYG{l+m+mi}{1}\PYG{p}{]}\PYG{p}{)}\PYG{p}{)}\PYG{p}{,}
                         \PYG{n}{xytext}\PYG{o}{=}\PYG{p}{(}\PYG{n}{np}\PYG{o}{.}\PYG{n}{argmax}\PYG{p}{(}\PYG{n}{data}\PYG{p}{[}\PYG{p}{:}\PYG{p}{,} \PYG{o}{\PYGZhy{}}\PYG{l+m+mi}{1}\PYG{p}{]}\PYG{p}{)} \PYG{o}{*} \PYG{l+m+mf}{1.05}\PYG{p}{,} \PYG{n}{np}\PYG{o}{.}\PYG{n}{max}\PYG{p}{(}\PYG{n}{data}\PYG{p}{[}\PYG{p}{:}\PYG{p}{,} \PYG{o}{\PYGZhy{}}\PYG{l+m+mi}{1}\PYG{p}{]}\PYG{p}{)} \PYG{o}{*} \PYG{l+m+mf}{1.01}\PYG{p}{)}\PYG{p}{,}
                         \PYG{n}{arrowprops}\PYG{o}{=}\PYG{n+nb}{dict}\PYG{p}{(}\PYG{n}{facecolor}\PYG{o}{=}\PYG{l+s+s1}{\PYGZsq{}}\PYG{l+s+s1}{black}\PYG{l+s+s1}{\PYGZsq{}}\PYG{p}{,} \PYG{n}{shrink}\PYG{o}{=}\PYG{l+m+mi}{1}\PYG{p}{)}\PYG{p}{,}
                         \PYG{p}{)}
            \PYG{n}{plt}\PYG{o}{.}\PYG{n}{annotate}\PYG{p}{(}\PYG{l+s+s1}{\PYGZsq{}}\PYG{l+s+s1}{Highest Score Random Search}\PYG{l+s+s1}{\PYGZsq{}}\PYG{p}{,} \PYG{n}{xycoords}\PYG{o}{=}\PYG{l+s+s1}{\PYGZsq{}}\PYG{l+s+s1}{data}\PYG{l+s+s1}{\PYGZsq{}}\PYG{p}{,}
                         \PYG{n}{xy}\PYG{o}{=}\PYG{p}{(}\PYG{n}{np}\PYG{o}{.}\PYG{n}{argmax}\PYG{p}{(}\PYG{n}{data\PYGZus{}compare}\PYG{p}{[}\PYG{p}{:}\PYG{p}{,} \PYG{o}{\PYGZhy{}}\PYG{l+m+mi}{1}\PYG{p}{]}\PYG{p}{)}\PYG{p}{,} \PYG{n}{np}\PYG{o}{.}\PYG{n}{max}\PYG{p}{(}\PYG{n}{data\PYGZus{}compare}\PYG{p}{[}\PYG{p}{:}\PYG{p}{,} \PYG{o}{\PYGZhy{}}\PYG{l+m+mi}{1}\PYG{p}{]}\PYG{p}{)}\PYG{p}{)}\PYG{p}{,}
                         \PYG{n}{xytext}\PYG{o}{=}\PYG{p}{(}\PYG{n}{np}\PYG{o}{.}\PYG{n}{argmax}\PYG{p}{(}\PYG{n}{data\PYGZus{}compare}\PYG{p}{[}\PYG{p}{:}\PYG{p}{,} \PYG{o}{\PYGZhy{}}\PYG{l+m+mi}{1}\PYG{p}{]}\PYG{p}{)} \PYG{o}{*} \PYG{l+m+mf}{1.05}\PYG{p}{,} \PYG{n}{np}\PYG{o}{.}\PYG{n}{max}\PYG{p}{(}\PYG{n}{data\PYGZus{}compare}\PYG{p}{[}\PYG{p}{:}\PYG{p}{,} \PYG{o}{\PYGZhy{}}\PYG{l+m+mi}{1}\PYG{p}{]}\PYG{p}{)} \PYG{o}{*} \PYG{l+m+mf}{0.95}\PYG{p}{)}\PYG{p}{,}
                         \PYG{n}{arrowprops}\PYG{o}{=}\PYG{n+nb}{dict}\PYG{p}{(}\PYG{n}{facecolor}\PYG{o}{=}\PYG{l+s+s1}{\PYGZsq{}}\PYG{l+s+s1}{black}\PYG{l+s+s1}{\PYGZsq{}}\PYG{p}{,} \PYG{n}{shrink}\PYG{o}{=}\PYG{l+m+mi}{1}\PYG{p}{)}\PYG{p}{,}
                         \PYG{p}{)}

            \PYG{c+c1}{\PYGZsh{} second plot}
            \PYG{n}{data\PYGZus{}high} \PYG{o}{=} \PYG{n}{data}
            \PYG{k}{for} \PYG{n}{x} \PYG{o+ow}{in} \PYG{n+nb}{range}\PYG{p}{(}\PYG{n+nb}{len}\PYG{p}{(}\PYG{n}{data\PYGZus{}high}\PYG{p}{)} \PYG{o}{\PYGZhy{}} \PYG{l+m+mi}{1}\PYG{p}{)}\PYG{p}{:}
                \PYG{k}{if} \PYG{n}{data\PYGZus{}high}\PYG{p}{[}\PYG{n}{x}\PYG{p}{,} \PYG{o}{\PYGZhy{}}\PYG{l+m+mi}{1}\PYG{p}{]} \PYG{o}{\PYGZgt{}} \PYG{n}{data\PYGZus{}high}\PYG{p}{[}\PYG{n}{x} \PYG{o}{+} \PYG{l+m+mi}{1}\PYG{p}{,} \PYG{o}{\PYGZhy{}}\PYG{l+m+mi}{1}\PYG{p}{]}\PYG{p}{:}
                    \PYG{n}{data\PYGZus{}high}\PYG{p}{[}\PYG{n}{x} \PYG{o}{+} \PYG{l+m+mi}{1}\PYG{p}{,} \PYG{o}{\PYGZhy{}}\PYG{l+m+mi}{1}\PYG{p}{]} \PYG{o}{=} \PYG{n}{data\PYGZus{}high}\PYG{p}{[}\PYG{n}{x}\PYG{p}{,} \PYG{o}{\PYGZhy{}}\PYG{l+m+mi}{1}\PYG{p}{]}

            \PYG{n}{data\PYGZus{}compare\PYGZus{}high} \PYG{o}{=} \PYG{n}{data\PYGZus{}compare}
            \PYG{k}{for} \PYG{n}{x} \PYG{o+ow}{in} \PYG{n+nb}{range}\PYG{p}{(}\PYG{n+nb}{len}\PYG{p}{(}\PYG{n}{data\PYGZus{}compare\PYGZus{}high}\PYG{p}{)} \PYG{o}{\PYGZhy{}} \PYG{l+m+mi}{1}\PYG{p}{)}\PYG{p}{:}
                \PYG{k}{if} \PYG{n}{data\PYGZus{}compare\PYGZus{}high}\PYG{p}{[}\PYG{n}{x}\PYG{p}{,} \PYG{o}{\PYGZhy{}}\PYG{l+m+mi}{1}\PYG{p}{]} \PYG{o}{\PYGZgt{}} \PYG{n}{data\PYGZus{}compare\PYGZus{}high}\PYG{p}{[}\PYG{n}{x} \PYG{o}{+} \PYG{l+m+mi}{1}\PYG{p}{,} \PYG{o}{\PYGZhy{}}\PYG{l+m+mi}{1}\PYG{p}{]}\PYG{p}{:}
                    \PYG{n}{data\PYGZus{}compare\PYGZus{}high}\PYG{p}{[}\PYG{n}{x} \PYG{o}{+} \PYG{l+m+mi}{1}\PYG{p}{,} \PYG{o}{\PYGZhy{}}\PYG{l+m+mi}{1}\PYG{p}{]} \PYG{o}{=} \PYG{n}{data\PYGZus{}compare\PYGZus{}high}\PYG{p}{[}\PYG{n}{x}\PYG{p}{,} \PYG{o}{\PYGZhy{}}\PYG{l+m+mi}{1}\PYG{p}{]}

            \PYG{n}{plt}\PYG{o}{.}\PYG{n}{figure}\PYG{p}{(}\PYG{l+m+mi}{2}\PYG{p}{,} \PYG{n}{figsize}\PYG{o}{=}\PYG{p}{(}\PYG{l+m+mi}{25}\PYG{p}{,} \PYG{l+m+mi}{12}\PYG{p}{)}\PYG{p}{)}
            \PYG{n}{plt}\PYG{o}{.}\PYG{n}{plot}\PYG{p}{(}\PYG{n}{np}\PYG{o}{.}\PYG{n}{array}\PYG{p}{(}\PYG{n+nb}{range}\PYG{p}{(}\PYG{n+nb}{len}\PYG{p}{(}\PYG{n}{data}\PYG{p}{[}\PYG{p}{:}\PYG{p}{,} \PYG{o}{\PYGZhy{}}\PYG{l+m+mi}{1}\PYG{p}{]}\PYG{p}{)}\PYG{p}{)}\PYG{p}{)}\PYG{p}{,} \PYG{n}{data\PYGZus{}high}\PYG{p}{[}\PYG{p}{:}\PYG{p}{,} \PYG{o}{\PYGZhy{}}\PYG{l+m+mi}{1}\PYG{p}{]}\PYG{p}{,} \PYG{n}{label}\PYG{o}{=}\PYG{l+s+s1}{\PYGZsq{}}\PYG{l+s+s1}{ForestFire}\PYG{l+s+s1}{\PYGZsq{}}\PYG{p}{)}
            \PYG{n}{plt}\PYG{o}{.}\PYG{n}{plot}\PYG{p}{(}\PYG{n}{np}\PYG{o}{.}\PYG{n}{array}\PYG{p}{(}\PYG{n+nb}{range}\PYG{p}{(}\PYG{n+nb}{len}\PYG{p}{(}\PYG{n}{data\PYGZus{}compare}\PYG{p}{[}\PYG{p}{:}\PYG{p}{,} \PYG{o}{\PYGZhy{}}\PYG{l+m+mi}{1}\PYG{p}{]}\PYG{p}{)}\PYG{p}{)}\PYG{p}{)}\PYG{p}{,} \PYG{n}{data\PYGZus{}compare}\PYG{p}{[}\PYG{p}{:}\PYG{p}{,} \PYG{o}{\PYGZhy{}}\PYG{l+m+mi}{1}\PYG{p}{]}\PYG{p}{,} \PYG{n}{label}\PYG{o}{=}\PYG{l+s+s1}{\PYGZsq{}}\PYG{l+s+s1}{Random Search}\PYG{l+s+s1}{\PYGZsq{}}\PYG{p}{)}

            \PYG{n}{plt}\PYG{o}{.}\PYG{n}{xlabel}\PYG{p}{(}\PYG{l+s+s1}{\PYGZsq{}}\PYG{l+s+s1}{No. tested Feature Sets}\PYG{l+s+s1}{\PYGZsq{}}\PYG{p}{)}
            \PYG{n}{plt}\PYG{o}{.}\PYG{n}{ylabel}\PYG{p}{(}\PYG{l+s+s1}{\PYGZsq{}}\PYG{l+s+s1}{Score}\PYG{l+s+s1}{\PYGZsq{}}\PYG{p}{)}
            \PYG{n}{plt}\PYG{o}{.}\PYG{n}{title}\PYG{p}{(}\PYG{l+s+s1}{\PYGZsq{}}\PYG{l+s+s1}{Results all time best score}\PYG{l+s+s1}{\PYGZsq{}}\PYG{p}{)}
            \PYG{n}{plt}\PYG{o}{.}\PYG{n}{legend}\PYG{p}{(}\PYG{n}{loc}\PYG{o}{=}\PYG{l+m+mi}{2}\PYG{p}{)}

            \PYG{n}{plt}\PYG{o}{.}\PYG{n}{show}\PYG{p}{(}\PYG{p}{)}
\end{sphinxVerbatim}


\chapter{Indices and tables}
\label{\detokenize{index:indices-and-tables}}\begin{itemize}
\item {} 
\DUrole{xref,std,std-ref}{genindex}

\item {} 
\DUrole{xref,std,std-ref}{modindex}

\item {} 
\DUrole{xref,std,std-ref}{search}

\end{itemize}
\phantomsection\label{\detokenize{index:blank}}\begin{figure}[htbp]
\centering

\noindent\sphinxincludegraphics[scale=0.8]{{blank}.jpg}
\label{\detokenize{index:blank}}\end{figure}

\begin{sphinxthebibliography}{Collective_Intelligence}
\bibitem[Collective\_Intelligence]{\detokenize{Collective_Intelligence}}{\phantomsection\label{\detokenize{Overview:collective-intelligence}} 
Collective Intelligence, O'Reilly, ISBN: 978-0-596-52932-1
}
\end{sphinxthebibliography}


\renewcommand{\indexname}{Python Module Index}
\begin{sphinxtheindex}
\def\bigletter#1{{\Large\sffamily#1}\nopagebreak\vspace{1mm}}
\bigletter{f}
\item {\sphinxstyleindexentry{ForestFire}}\sphinxstyleindexpageref{index:\detokenize{module-ForestFire}}
\end{sphinxtheindex}

\renewcommand{\indexname}{Index}
\printindex
\end{document}